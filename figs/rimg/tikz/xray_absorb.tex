% Principle of X-ray photoelectron spectroscopy (XPS)
% Author: Mathias Laurin
\documentclass{article}
\usepackage{tikz}

\usepackage{verbatim}
\usepackage[active,tightpage]{preview}
\PreviewEnvironment{tikzpicture}
\setlength\PreviewBorder{5pt}%
%%

\usetikzlibrary{decorations.pathmorphing, arrows, intersections, snakes}


\begin{document}
\begin{tikzpicture}[scale=0.25]

     \begin{scope}[color=red!25!yellow!50]
        \draw[->,very thick, >=latex] (8.5, 0.75) to [out=115, in=245] (10, 25) ;
    \end{scope}


    \begin{scope} % Energy levels
        \draw (1,   0) node[left] {$1s$}      --  ++(18, 0) node[right] {$K$};
        \draw (1,   7) node[left] {$2s$}     --  ++(18, 0) node[right] {$L_1$};
        \draw (1, 8.5) node[left] {$2p$}     --  ++(18 ,0) node[right] {$L_2,L_3$};

        \draw (1,  12) node[left] {$3s$}     --  ++(18, 0) node[right] {$M_1$};
        \draw (1, 13.5) node[left] {$3p$}     --  ++(18 ,0) node[right] {$M_2,M_3$};
        \draw (1,   15) node[left] {$3d$}     --  ++(18 ,0) node[right] {$M_4,M_5$};

        \draw (1,   20) node[right=30, below=-1] {{\small{valence band}}} -- ++(18,0) node[right] {$E_{\rm Fermi}$};
        \draw (1,   22) node[right=38, above=-1] {{\small{conduction band}}} -- ++(18,0) node[right] {$E_{\rm vacuum}$};

       \draw[->,>=latex,  thick] (26, 0) -- ++(0, 30) node[sloped,right=0,below=2, midway] {Energy};

        \end{scope}

    \begin{scope} % core electrons
        \foreach \x in {1,3,5, 7, 9, 11, 13, 15, 17,19}  \filldraw (\x, 15) circle (.25);
        \foreach \x in {5, 7, 9, 11, 13, 15}  \filldraw (\x, 13.5) circle (.25);
        \foreach \x in {9, 11}       \filldraw (\x, 12) circle (.25);
        \foreach \x in {5, 7, 9, 11, 13, 15}  \filldraw (\x, 8.5) circle (.25);
        \foreach \x in {9, 11}       \filldraw (\x, 7) circle (.25);
        \filldraw (11, 0) circle (.25);
        \draw   (9, 0) circle (.25);
    \end{scope}
     \begin{scope}[color=blue!70!black]
        \draw[->, thick, >=stealth', shorten >=-5pt, decorate, decoration={snake, amplitude=2.5}] (-2, 3)  --  node[left=3, above=4, midway] {{\small{X-ray}}} (8, 0.5);
    \end{scope}
    \begin{scope}[color=red!50!black]
        \draw[->, thick, ,>=stealth', shorten >=-5pt, snake=snake, segment length=16, segment amplitude=3.5, line after snake=1] (10., 25)  --  node[above=2, midway]{photoelectron} ++(10, 0) ;
    \end{scope}

    \end{tikzpicture}
\end{document}

