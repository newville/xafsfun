% Lambert-Beer law parameters drawing
% Author: Michele Muccioli
% Compile with LuaLaTeX
\documentclass{article}
\usepackage{tikz}
\usetikzlibrary{calc,fadings,decorations.markings}
\usepackage{amsmath}
%\usepackage[active, tightpage]{preview}
%\setlength\PreviewBorder{5pt}%
\usepackage{verbatim}
\usepackage[active,tightpage]{preview}
\PreviewEnvironment{tikzpicture}
\setlength\PreviewBorder{5pt}


\makeatletter
\pgfkeys{/pgf/decoration/.cd,
         start color/.store in = \startcolor,
         end color/.store in   = \endcolor
        }

\pgfdeclaredecoration{color change}{initial}{
% Initial state
\state{initial}[%
    width                     = 0pt,
    next state                = line,
    persistent precomputation = {\pgfmathdivide{50}{\pgfdecoratedpathlength}%
                                 \let\increment=\pgfmathresult%
                                 \def\x{0}}]%
{}%

% Line state
\state{line}[%
    width                      = .5pt,
    persistent postcomputation = {\pgfmathadd@{\x}{\increment}%
                                  \let\x=\pgfmathresult}]%
{%
  \pgfsetlinewidth{\pgflinewidth}%
  \pgfsetarrows{-}%
  \pgfpathmoveto{\pgfpointorigin}%
  \pgfpathlineto{\pgfqpoint{.75pt}{0pt}}%
  \pgfsetstrokecolor{\endcolor!\x!\startcolor}%
  \pgfusepath{stroke}%
}%

% Final state
\state{final}{%
  \pgfsetlinewidth{\pgflinewidth}%
  \pgfpathmoveto{\pgfpointorigin}%
  \color{\endcolor!\x!\startcolor}%
  \pgfusepath{stroke}%
}
}
\makeatother

\begin{document}


\begin{tikzpicture}[scale=0.25, line width=2pt]

\draw[gray!10!brown!10,fill=gray!10!brown!10, line width=0.25] (0, 7) rectangle +(7, 6);


\draw[black, decoration = {markings,
        mark = at position 0.5 with {\arrow[]{latex}}},
        postaction = {decorate}] (-5 , 10)--++ (5,0)node[midway, above, black]{$I_0$};

\draw[decoration = {color change,  start color = black, end color   = black!15!white}, decorate] (0, 10)--++ (7, 0);

\draw[black!15!white, decoration = {markings,
        mark = at position 0.5 with {\arrow[]{latex}}},
        postaction = {decorate}] (7, 10)--++(5,  0)node[midway, above, black]{$I$};


\draw[|<->|, >= latex, line width=0.8] ($(0, 12.75)+(0,1)$)--++ (7, 0)node[midway, above]{$t$};



%%%%%%%%%%%%%%%%%%%%
\end{tikzpicture}
\end{document}
