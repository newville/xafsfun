\usepackage{color,fancybox,alltt,graphicx}

\usepackage{pgf,pgfarrows,pgfnodes,pgfautomata,pgfheaps,pgfshade}

\usepackage[latin1]{inputenc}
\usepackage{colortbl}
\usepackage[english]{babel}
\usepackage{multimedia}
\usepackage{amssymb,amsmath}
\usepackage{ragged2e}
\usepackage{animate}
\usepackage{listings}

\newif\ifpdf
\ifx\pdfoutput\undefined
\pdffalse % we are not running PDFLaTeX
\else
\pdfoutput=1 % we are running PDFLaTeX
\pdftrue
\fi

\mode<article>{ \usepackage{fullpage}  \usepackage{pgf}  \usepackage{hyperref} }


\mode<presentation>{
  \usetheme{XAFS}
  \setbeamercovered{transparent}
}

\usepackage{amsmath}
\usepackage{tikz}

\usepackage[customcolors,shade]{hf-tikz}

\usetikzlibrary{calc}

% put color to \boxed math command
\newcommand*{\cmboxcolor}{orange}
\makeatletter
\newcommand{\cmbox}[1]{\textcolor{\cmboxcolor}{%
 \makeatother
\tikz[baseline={([yshift=-1ex]current bounding box.center)}] \node [rectangle, minimum width=1ex,rounded corners,draw] {\normalcolor\m@th$\displaystyle#1$};}}



\usepackage[latin1]{inputenc}
\usepackage[english]{babel}
\setbeamertemplate{navigation symbols}{}

\setlength{\fboxrule}{1pt}


\newcommand{\vmm}{{\vspace{2mm}}}
\newcommand{\hmm}{{\hspace{1mm}}}
\newcommand{\Justify}{\justify\vspace{-\baselineskip}}

\newcommand{\Program}[1]{\scshape{#1}}
\newcommand{\atoms}{{\Program{atoms}}}
\newcommand{\feffit}{{\Program{feffit}}}
\newcommand{\ifeffit}{{\Program{ifeffit}}}
\newcommand{\larch}{{\Program{larch}}}
\newcommand{\xasviewer}{{\Program{XAS Viewer}}}

\newcommand{\ease}{{\Program{EASE}}}
\newcommand{\autobk}{{\Program{autobk}}}
\newcommand{\ffchi}{{\Program{ff2chi}}}
\newcommand{\diffkk}{{\Program{diffkk}}}
\newcommand{\sixpack}{{\Program{sixpack}}}
\newcommand{\hephaestus}{{\Program{hephaestus}}}
\newcommand{\athena}{{\Program{athena}}}
\newcommand{\artemis}{{\Program{artemis}}}
\newcommand{\feff}{{\Program{feff}}}
\newcommand{\mxan}{{\Program{mxan}}}
\newcommand{\fdmnes}{{\Program{fdmnes}}}
\newcommand{\gnuplot}{{\Program{gnuplot}}}

\newcommand{\file}[1]{{{\slshape\ttfamily{#1}}}}
\newcommand{\feffndat}{\file{feffnnnn.dat}}
\newcommand{\feffbin}{\file{feff.bin}}


\newcommand{\bmu}{{\mu}}
\newcommand{\bepsilon}{{\epsilon}}
\newcommand{\bDelta}{{\Delta}}
\newcommand{\bOmega}{{\Omega}}
\newcommand{\bdelta}{{\delta}}
\newcommand{\bsigma}{{\sigma}}
\newcommand{\bln}{{\ln}}
\newcommand{\bsum}{{\sum}}
\newcommand{\bsim}{{\sim}}
\newcommand{\bsin}{{\sin}}
\newcommand{\bexp}{{\exp}}
\newcommand{\bint}{{\int}}
\newcommand{\bpsi}{{\psi}}
\newcommand{\bpropto}{{\propto}}
\newcommand{\bapprox}{{\approx}}
\newcommand{\bchi}{{\chi}}
\newcommand{\brho}{{\rho}}
\newcommand{\bpi}{{\pi}}
\newcommand{\balpha}{{\alpha}}
\newcommand{\bbeta}{{\beta}}
\newcommand{\blambda}{{\lambda}}
\newcommand{\blesssim}{{\lesssim}}
\newcommand{\brightarrow}{{\rightarrow}}
\newcommand{\bAA}{{\rm\AA}}
\newcommand{\mbf}[1]{{\ensuremath{\mathbf\mathit{#1}}}}

\newcommand{\chie}{{\ensuremath{\chi(E)}}}
\newcommand{\chik}{{\ensuremath{\chi(k)}}}
\newcommand{\chir}{{\ensuremath{\chi(R)}}}
\newcommand{\mue}{{\ensuremath{\mu(E)}}}
\newcommand{\bkg}{{\ensuremath{\mu_0(E)}}}

\definecolor{lightyellow}{rgb}{1.0,1.0,0.8}
\definecolor{lightyellow2}{rgb}{1.0,1.0,0.97}
\definecolor{golden}{rgb}{0.75,0.75,0.37}
\definecolor{lightpink}{rgb}{1.0,0.9,0.9}
\definecolor{nearwhite}{rgb}{0.95,0.94,0.94}
\definecolor{verywhite}{rgb}{0.99,0.99,0.99}
\definecolor{white}{rgb}{1.0,1.0,1.0}
\definecolor{DarkBlue}{rgb}{0,0,0.3}
\definecolor{BrightBlue}{rgb}{0,0,0.7}
\definecolor{DarkRed}{rgb}{0.65,0,0}
\definecolor{BrightRed}{rgb}{0.8,0,0}
\definecolor{RRed}{rgb}{0.95,0,0}
\definecolor{BlueGrey}{rgb}{0.2,0.1,0.1}

\definecolor{VBlue}{rgb}{0,0,0.9}
\definecolor{BrightGreen}{rgb}{0,0.6,0.0}
\definecolor{DarkGreen}{rgb}{0,0.3,0}


\newcommand{\Color}[2]{{\textcolor{#1}{#2}}}
\newcommand{\Red}[1]{{\Color{BrightRed}{#1}}}
\newcommand{\RRed}[1]{{\Color{RRed}{#1}}}
\newcommand{\DarkRed}[1]{{\Color{DarkRed}{#1}}}
\newcommand{\Blue}[1]{{\Color{BrightBlue}{#1}}}
\newcommand{\BrightBlue}[1]{{\Color{BrightBlue}{#1}}}
\newcommand{\Black}[1]{{\Color{black}{#1}}}
\newcommand{\RedM}[1]{{\Color{red}{\mbf{#1}}}}
\newcommand{\BlueM}[1]{{\Color{blue}{\mbf{#1}}}}
\newcommand{\BlackM}[1]{{\Color{black}{\mbf{#1}}}}


\newcommand{\DarkGreen}[1]{{\Color{DarkGreen}{#1}}}
\newcommand{\DarkBlue}[1]{{\Color{DarkBlue}{#1}}}


\newcommand{\RedEmph}[1]{{\Color{BrightRed}{\emph{#1}}}}
\newcommand{\RedSl}[1]{{\Color{BrightRed}{\slshape{#1}}}}
\newcommand{\BlueSl}[1]{{\Color{BrightBlue}{\slshape{#1}}}}
\newcommand{\BlueEmph}[1]{{\Color{BrightBlue}{\emph{#1}}}}


\newcommand{\LString}[1]{{\Color{BrightGreen}{{#1}}}}
\newcommand{\LKeyword}[1]{{\Color{DarkRed}{{#1}}}}
\newcommand{\LFunc}[1]{{\Color{VBlue}{{#1}}}}
\newcommand{\LComment}[1]{{\Color{BrightRed}{{#1}}}}


\newcommand{\pthpar}[1]{{\ensuremath{{\tt{\Blue{#1}}}}}}

\newcommand{\feffc}[1]{{{\ensuremath{{\Red{#1}}}}}}
\newcommand{\reff}{{{\feffc{R_{\rm eff}}}}}
\newcommand{\twothirds}{{{\textstyle{2 \over 3}}}}
\newcommand{\fourthirds}{{{\textstyle{2 \over 3}}}}
\newcommand{\masse}{{({{2m_e} / {\hbar^2}})}}

\newcommand{\highlightbox}[1]{{ \fcolorbox{black}{lightyellow}{#1}}}

\newenvironment{VerbSBox}[1]%
{\VerbatimEnvironment\begin{Sbox}%
\begin{minipage}{#1}\begin{alltt}}%
{\end{alltt}\end{minipage}\end{Sbox}\setlength{\fboxsep}{2mm}{%
\begin{flushright}\shadowbox{\TheSbox}\end{flushright}}}
%%

\newenvironment{VerbSSBox}[1]%
{\VerbatimEnvironment\begin{Sbox}%
\begin{minipage}{#1}\begin{alltt}}%
{\end{alltt}\end{minipage}\end{Sbox}\setlength{\fboxsep}{2mm}{%
\begin{center}\shadowbox{\TheSbox}\end{center}}}
%%

% \newenvironment{VerbBox}[1]%
% {\VerbatimEnvironment\begin{Sbox}%
% \begin{minipage}{#1}\begin{alltt}}%
% {\end{alltt}\end{minipage}\end{Sbox}\setlength{\fboxsep}{2mm}{%
% \shadowbox{\TheSbox}
% %%

\newenvironment{CodeBlock}[2]%
{\VerbatimEnvironment\begin{minipage}{#1}\begin{block}{\small{#2}}\begin{semiverbatim}\tiny}%
{\end{semiverbatim}\end{block}\end{minipage}}%%

\definecolor{DeepGrey}{rgb}{0.15,0.05,0.05}
\newcommand{\GreyLine}{{\color{DeepGrey}{\rule{\linewidth}{1.00pt}}}}

\newcommand{\STitle}[1]{{\hspace{2mm}{\bfseries\sl\Large%
      \BrightBlue{#1}}}\hfill\par%
  \vspace{-3.5mm}\GreyLine\vspace{-0.1mm}}

\newenvironment{ListingBlock}[2]%
{\begin{minipage}{#1}\begin{block}{\small{#2}}\begin{lstlisting}}
{\end{lstlisting}\end{block}\end{minipage}}%%

\newenvironment{figblock}[3]%
{\begin{minipage}{#1}\begin{exampleblock}{#2}{\wgraph{#1}{#3}}}%
{\end{exampleblock}\end{minipage}}%

\newenvironment{MFrame}[1]%%
{\subsection{#1}\begin{frame}\frametitle{#1}}%
{\end{frame}}%

\newenvironment{Boxedminipage}%
    {\begin{Sbox}\begin{minipage}}%
    {\end{minipage}\end{Sbox}\shadowbox{\TheSbox}}


\setbeamercolor{postit}{fg=black,bg=lightyellow}

\newenvironment{postitbox}[1]%%
{\begin{center}\begin{minipage}{#1}\begin{beamercolorbox}[shadow=true,rounded=true]{postit}}%
{\end{beamercolorbox}\end{minipage}\end{center}}%


\newenvironment{postitboxC}%%
{\begin{center}\begin{beamercolorbox}[center,shadow=true,rounded=true]{postit}}%
{\end{beamercolorbox}\end{center}}%

\newcommand{\entrylabel}[1]{ {\Blue{#1:}}}
%%\newcommand{\entrylabel}[1]{{\parbox[b]{10mm}{%
%%      \makebox[10mm][l]{{\Red{#1:}}}\\}}}

\newenvironment{entry}
{\begin{list}{}{\renewcommand{\makelabel}{\entrylabel}%
      \setlength{\labelwidth}{7mm}
      \setlength{\leftmargin}{6mm}}}{\end{list}}

\newcommand{\redlabel}[1]{ {\Color{DarkRed}{#1}}}
\newenvironment{redlist}[1]{\begin{list}{}{\renewcommand{\makelabel}{\redlabel}%
      \setlength{\leftmargin}{#1}}}{\end{list}}


\newcommand{\bluelabel}[1]{ {\Color{BrightBlue}{#1}}}
\newenvironment{bluelist}[1]{\begin{list}{}{\renewcommand{\makelabel}{\bluelabel}%
      \setlength{\leftmargin}{#1}}}{\end{list}}

\newcommand{\xbluelabel}[1]{{\Color{BrightBlue}{#1\hspace{2mm}}}}
\newenvironment{xbluelist}[1]{\begin{list}{}{\renewcommand{\makelabel}{\xbluelabel}%
      \setlength{\leftmargin}{#1}}}{\end{list}}

\newenvironment{cenpage}[1]%%  centered minipage
{\begin{center}\begin{minipage}{#1}}%
    {\end{minipage}\end{center}}%%

\newenvironment{slide}[1]%%  begin named slide
{\begin{frame}\frametitle{#1}}%
{\end{frame}}%%

\newenvironment{fslide}[1]%%  begin named slide
{\subsection{#1}\begin{frame}[fragile]\frametitle{#1}}%
{\end{frame}}%%


\newcommand{\xdgraph}[2]{{%
        \ifpdf  \includegraphics[width={#1}]{figs/#2.png}%
        \else   \includegraphics[width={#1}]{figs/#2.eps}\fi}}

\newcommand{\rgraph}[2]{\includegraphics[width={#1}]{figs/rimg/#2}}
\newcommand{\wgraph}[2]{\includegraphics[width={#1}]{figs/#2.png}}
\newcommand{\hgraph}[2]{\includegraphics[height={#1}]{figs/#2.png}}
\newcommand{\wpdf}[2]{\includegraphics[width={#1}]{#2.pdf}}

\newcommand{\webpage}[1]{{{\Blue{{#1}}}}}

%% \pgfdeclareimage[interpolate=true,width=45mm]{xafscartoon}{figs/xafsabsorb}
%% \pgfdeclareimage[interpolate=true,width=50mm]{xafsxanes}{figs/xafsxanes}
%% \pgfdeclareimage[interpolate=true,width=50mm]{feff}{figs/scattamp}
%% \pgfdeclareimage[interpolate=true,width=55mm]{tdlplot}{figs/tdlplot}

%% \pgfdeclareimage[interpolate=true,width=15mm]{ravel}{figs/RavelHead}
%% \pgfdeclareimage[interpolate=true,width=15mm]{calvin}{figs/CalvinHead}
%% \pgfdeclareimage[interpolate=true,width=15mm]{frenkel}{figs/FrenkelHead}
%% \pgfdeclareimage[interpolate=true,width=15mm]{haskel}{figs/HaskelHead}
%% \pgfdeclareimage[interpolate=true,width=15mm]{jox}{figs/JoxHead}
%% \pgfdeclareimage[interpolate=true,width=15mm]{newville}{figs/NewvilleHead}
%% \pgfdeclareimage[interpolate=true,width=15mm]{kelly}{figs/KellyHead}

%% \pgfdeclareimage[interpolate=true,width=15mm]{rehr}{figs/RehrHead}
%% \pgfdeclareimage[interpolate=true,width=15mm]{fons}{figs/FonsHead}
%% \pgfdeclareimage[interpolate=true,width=15mm]{webb}{figs/WebbHead}
%% \pgfdeclareimage[interpolate=true,width=15mm]{glover}{figs/GloverHead}
%% \pgfdeclareimage[interpolate=true,width=15mm]{trainor}{figs/TrainorHead}

%% \pgfdeclareimage[interpolate=true,width=85mm]{APS2005_Photo}{figs/APS2005_Photo}

%% \pgfdeclareimage[interpolate=true,width=68mm]{EASE}{figs/EASE}
