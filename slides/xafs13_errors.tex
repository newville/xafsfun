\begin{slide}{The Status of EXAFS Analysis (summary so far...)}
    
   For analyzing EXAFS data we can:
    
    \begin{itemize}
    \item use {\Blue{Constraints}} to build physical models for our data.
      
    \item use {\Blue{Bayesian Principles}} and {\Blue{Restraints}} to
      express imperfect Prior Knowledge of our system.
      
    \item select which $k$- and $R$-components to use in the analysis.

    \item simultaneously refine multiple data sets, multiple $k$-weights,
      etc. 

    \end{itemize}

    so many clever things we can do in the analysis \ldots

\vspace{1mm}

\vfill
\end{slide} 

\begin{slide}{The Status of EXAFS Analysis (summary so far...)}
    
   For analyzing EXAFS data we can:
    
    \begin{itemize}
    \item use {\Blue{Constraints}} to build physical models for our data.
      
    \item use {\Blue{Bayesian Principles}} and {\Blue{Restraints}} to
      express imperfect Prior Knowledge of our system.
      
    \item select which $k$- and $R$-components to use in the analysis.

    \item simultaneously refine multiple data sets, multiple $k$-weights,
      etc. 

    \end{itemize}

    so many clever things we can do in the analysis \ldots and
    
    {\begin{center} \large \highlightbox{ {\RedEmph{Our fits are
              TERRIBLE!}} } 
      \end{center} }

    \vmm
    
    The ``goodness-of-fit statistic'' reduced chi-square, $\chi^2_\nu =
    \chi^2 / (N_{\rm idp}-N_{\rm varys})$ should be $\sim$ 1 for a ``Good
    Fit'':

    \[    \begin{array}{ll}
      \chi^2_\nu & = \displaystyle \frac{N_{\rm idp}}{\epsilon^2 N_{\rm fit} (N_{\rm 
          idp}-N_{\rm varys})} 
      \sum_i^{N_{\rm fit}} [\chi_i^{\rm data} - \chi_i^{\rm
      model}(\vec{x})]^2 \\
    \end{array}
    \]
    
    \vmm
    
    We're lucky if $\chi_\nu^2$ is below 20!  \hspace{4mm}
    What are we doing wrong?

\vspace{1mm}

\vfill
\end{slide} 


%%%%%%%%%%%%%%%%%%%%%%
\begin{slide}{Noise Levels in Data }
    
    Is the estimated noise in the data, $\epsilon$, way off?  Here are some
    typical EXAFS spectra (both transmission, 1 sec/point):

    \vmm

      \begin{tabular}{lcl}\setlength{\baselineskip}{2pt}
        0.2 mM Zn nitrate solution & & Cu foil Room Temperature   \\
        \hspace{-7mm} \wgraph{48mm}{errors/noise_data01} & \hspace{2mm} & 
        \hspace{-5mm} \wgraph{48mm}{errors/noise_data02} \\
        $\epsilon \approx 6.6 \times 10^{-4}$ ``Normal Data''& &
        $\epsilon \approx 1.6 \times 10^{-4}$ ``Good Data''\\
      \end{tabular}

      \vmm

      For the Zn solution, this noise level is consistent with counting
      statistics for ``Number of Photons in Ion Chambers''! 
      
      \vmm A first shell fit to this Cu data gives $\chi^2_\nu \approx 200 $.  
      We multiply the uncertainties by $\sqrt(\chi^2_\nu) \approx 14$,
      as if $\epsilon$ was $14\times$ larger than our estimate.

      
      \begin{center}
        \highlightbox{That's $3\times$ data worse than the noise in the Zn
        data!}
      \end{center}

\vfill
\end{slide} 



%%%%%%%%%%%%%%%%%%%%%%
\begin{slide}{Room Temperature Cu Fit}

    Simple fit to first shell of Cu foil (300K): $k = [2,16] \rm\,
    \AA^{-1}$, $R = [1.7,2.6] \rm\, \AA$, $k$-weight=2, $N_{\rm idp} = 8.4
    $.  Fit results and statistics:
    

    {
      \hspace{0.1mm}\begin{tabular}{lll}
        $R = 2.548(0.007) \, \rm\AA$ 
        &     
        $\Delta E_0 = 4.5(0.6)$ 
        &  
        $C_3      = 9(9) \times10^{-5} \rm\, \AA^3$ 
        \\
        $\epsilon_k = 1.6 \times 10^{-4}$ 
        &
        $S_0^2 = 0.96(0.04)$  
        &
        $\sigma^2 = 8.5(0.3) \times10^{-3} \rm\, \AA^2$ 
        \\
        $\chi^2 = 678$ &
        $\chi^2_\nu = 196.7$   & ${\cal{R}} = 0.00107 $\\
      \end{tabular}
    }

    \vmm
      \begin{tabular}{lcl}
        \hspace{-10mm} \wgraph{49mm}{errors/cufit02} & \hspace{2mm} & 
        \hspace{-3mm}  \wgraph{49mm}{errors/cufit01} \\
      \end{tabular}

      \begin{itemize}
      \item ${\cal{R}} = 0.1\% $ -- a good fit!  But like $\chi^2_\nu$,
        ${\cal{R}}$ is larger than the $\epsilon_k$ suggests.
      \item These error bars account for correlations.  They increase
        $\chi^2$ by $\chi^2_\nu$ (not 1), which scales them by
        $\sqrt{\chi^2_\nu}\approx 14$ over ``increase $\chi^2$ by 1''.
      \end{itemize}

      \vmm

\vfill
\end{slide} 

%%%%%%%%%%%%%%%%%%%%%%
\begin{slide}{$\epsilon$: Noise Levels in Data}
    
    $\epsilon$ is estimated from the high-$R$ (15
    to 25 \AA) components of the data.  
    
    \vmm
      \begin{tabular}{lcl}
        \hspace{-10mm} \wgraph{52mm}{errors/znnoise01} &      & 
        \hspace{-5mm}  \wgraph{52mm}{errors/znnoise02} \\
        $|\chi(R)|$ &    &  $ \hspace{4mm} \log_{10}(|\chi(R)|)$ \\
      \end{tabular}
      
      \begin{itemize}
        \item The Zn data really shows ``white noise''.
          
        \item The Zn data is much noisier than the Cu data.  You need to
          look above $R \sim 12\rm \, \AA$ to see this.

      
      \item The Cu data has signal well above the noise level past 10\AA.

      \item Using the range $R=[15,25]\rm\, \AA$ may 
        {\RedEmph{overestimate}} $\epsilon$ for room temperature Cu.  
         This will be {\RedEmph{worse}} for low temperature data!

         \vmm
        \hspace{5mm} That's the wrong way: a smaller $\epsilon$  will
        make $\chi^2_\nu$ bigger! 

      \end{itemize}

      

\vfill
\end{slide} 
