
\begin{slide}{Modeling XANES}

    \vmm

    Ideally, we could model XANES by refining a structural model to best
    match data.

    \vmm

    This requires calculations that are good enough.  \ldots  Maybe.

    \vmm

    It also means  considering XANES as a multivariate function of a
    \begin{equation*}
      \mu(E,x_i,\alpha_j,\beta_k)
    \end{equation*}

    where

    \begin{tabular}{ll}
      $x_i$      & atomic positions \\
      $\alpha_j$ & parameters of the potentials (muffin tin radii, loss terms,\ldots) \\
      $\beta_k$  & empirical parameters (broadening, $E_0$, background  function, \ldots)
    \end{tabular}

  Then we could fit to XANES data:

  \begin{equation*}
  %     \mu_0(E, x_i, \alpha_j, \beta_k) @>>>
  %     \mathrm{compare~to~data} @>>> \mathrm{adjust}~ x_i, \alpha_j, \beta_k \\
  %     && @AAA @VVV \\
  %     &&  \mu(E, x_i, \alpha_j, \beta_k) @<<< \mathrm{converged?}
  % \end{equation*}

  \vmm

  This is still an ongoing development, but shows promise. See

  \hspace{4mm} Benfatto  {\emph{et al}}, Phys.\ Rev.\ \textbf{B65} 174205 (2002).

  \hspace{4mm} Smolentsev and Soldatov, J. Synch. Rad \textbf{13}, p19 (2006).


\vfill
\end{slide}
