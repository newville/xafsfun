%% Slide
\subsection{X-Ray Fluorescence}
\begin{frame} \frametitle{X-ray Fluorescence and Auger emission}

\begin{cenpage}{110mm}
  After X-ray absorption, the excited atom relaxes to the
  ground state.  A higher level electron fills the core hole, and a
  {\RedEmph{fluorescent X-ray}} or {\RedEmph{Auger electron}} is emitted.
\end{cenpage}

\vmm

  \begin{columns}[T]
      \begin{column}{55mm}
        {\RedEmph{X-ray Fluorescence}}:
        Emit an X-ray with energy given by core-levels energies.

       \hspace{4mm} \rgraph{38mm}{xray_emission_fluor}

        \begin{columns}
          \begin{column}{8mm}
            \rgraph{8mm}{Barkla}
          \end{column}
          \begin{column}{40mm}

            {\tiny{Charles Barkla, Nobel Prize, 1917 ``discovery of the characteristic R\"ontgen radiation of the elements"
                }}
          \end{column}
        \end{columns}

      \end{column}
      \begin{column}{55mm}
        \onslide+<2->

        {\RedEmph{Auger Effect}}:
        Promote an electron from another core-level to the continuum.

       \hspace{4mm} \rgraph{38mm}{xray_emission_auger}

        \begin{columns}
          \begin{column}{8mm}
            \rgraph{8mm}{Meitner}
          \end{column}
          \begin{column}{40mm}

            {\tiny{Lise Meitner, no Nobel Prize,  first to discover Auger effect, explained nuclear fission}}.

          \end{column}
        \end{columns}

      \end{column}
    \end{columns}

    \onslide+<3->

    \begin{postitbox}{105mm}
        X-ray fluorescence and Auger emission have discrete energies,
        characteristic of the absorbing atom -- very useful for identifying atoms!
      \end{postitbox}

\vfill
\end{frame}
