%% Slide
\begin{slide}{The EXAFS Equation: Sum over Paths (vs. ``Shells'')}
    
  We have an equation we can use to model and interpret EXAFS:
    
  \begin{center}
      \begin{minipage}{100mm} 
        \[ \chi(k) = \sum_j \frac{ {\Blue{N_j}} S_0^2 {\Red{f_j(k)}}
            e^{-2{\Blue{R_j}}/{\Red{\lambda(k)}}}\, 
            e^{-2k^2{\Blue{\sigma_j^2}}}}{k{\Blue{R_j}}^2}
          \sin{ [ 2k{\Blue{R_j}} + \Red{\delta_j(k)} ]} \] \end{minipage}
    \end{center}

    \vmm \pause

    where the sum could be over {\RedEmph{shells}} of atoms (Fe-O, Fe-Fe)
    or over {\RedEmph{scattering paths}} that the photo-electron might
    take:
    \[ \rm Fe_{\rm absorber} \Rightarrow O_{\rm 1st} \Rightarrow Fe_{\rm absorber} \]
    \[ \rm Fe_{\rm absorber}\Rightarrow Fe_{\rm 2nd} \Rightarrow Fe_{\rm absorber} \]

    \pause\vmm
    These are nearly the same concept.\par
    
    \vmm
    Using {\RedEmph{scattering path}} more easily allows
    multiple-scattering:

    \[ \rm Fe_{\rm absorber}\Rightarrow Fe_{\rm 2nd} \Rightarrow O_{\rm
      1st} \Rightarrow Fe_{\rm absorber} \]    
      

%     \vmm
%     We're done! \onslide+<6->  Except for filling in some details.

\end{slide} 
