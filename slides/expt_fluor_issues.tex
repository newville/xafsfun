
\begin{slide}{Fluorescence Measurements}

  There are two main considerations for getting a good fluorescence
  measurement:

    \begin{center}
      \begin{description}
        \settowidth{\labelwidth}{30mm}
        \setlength{\itemindent}{0mm}\setlength{\baselineskip}{11pt}
      \item[{\Blue{Energy Discrimination}}] either physically or
        electronically filter the unwanted portions of the
        fluorescence/scatter spectra.
      \item[{\Blue{Solid Angle}}] Fluorescence is emitted isotropically,
        so we'd like to collect as much of the ${\rm 4\pi}$
        of solid angle as possible.
      \end{description}
    \end{center}

    \vmm
    \pause

    \begin{tabular}{ll}
    \begin{minipage}{45mm}\setlength{\baselineskip}{10pt}
      A simple method of energy discrimination uses a {\RedEmph{filter}} of
      ``Z-1'' from the element of interest.  \vspace{2mm}
    
      For Fe, a Mn filter absorbs the scatter, while passing the
      Fe ${K_{\alpha}}$ line.

      \vspace{2mm} 
      This can be used with a detector with no energy resolution.
    \end{minipage}
    &
    \begin{minipage}{60mm}
      \scalebox{1}{\wgraph{61mm}{experiment/filter_spectra}}
    \end{minipage}
  \end{tabular}

\vfill
\end{slide} 
