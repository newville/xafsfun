%% Slide
\begin{slide}{The EXAFS Equation: What we left out}

    \vmm
    \begin{cenpage}{125mm}
      
    This simple description so far is qualitatively right, but for quantitative
    EXAFS calculations, it's important to consider several other points:

    \vspace{2mm}

    \begin{minipage}{102mm}
    \begin{description}
    \item[\RedEmph{Inelastic Losses}]  Processes that alter the absorbing
      atom or photo-electron before the photo-electron scatters back home.
      \begin{itemize}
      \item[\BlueEmph{Extrinsic Losses}] photo-electron mean-free path,
        including complex self-energy and finite core-hole lifetime.
      \item[\BlueEmph{Intrinsic Losses}] relaxation of absorbing atom due to
        the  presence of the core hole.
      \end{itemize}

    \item[\RedEmph{Multiple Scattering}] the photo-electron can scatter
      multiple times, which is important at low $k$, and can be important
      at high $k$ for some systems.

    \end{description}
    \end{minipage}

    \pause


    \vspace{3mm}
    Caclulations {\emph{should}} include these effects, and mostly do.


    \vspace{3mm}
    We'll discuss these in more detail \ldots.
\vfill

\end{cenpage}

\vfill
\end{slide}

