
\begin{frame} \frametitle{XANES: Conclusions}
  \small
  \begin{cenpage}{110mm}\setlength{\baselineskip}{10pt}

    \vspace{3mm}

    \begin{center}
      \fcolorbox{black}{lightyellow}{
        \begin{minipage}{85mm}  \setlength{\baselineskip}{10pt}
          {\BlueEmph{XANES is a much larger signal than  EXAFS}} 
          \vspace{2mm}
        
          XANES can be done at lower concentrations, and 
          less-than-perfect sample conditions.    
        \end{minipage}
        }
    \end{center}

    \vspace{3mm}

    \begin{center}
      \fcolorbox{black}{lightyellow}{
        \begin{minipage}{85mm}  \setlength{\baselineskip}{10pt}
          {\BlueEmph{XANES is easier to crudely interpret than EXAFS}}
          \vspace{2mm}

          For many systems, the XANES analysis based on linear
          combinations of known spectra from ``model compounds'' is
          sufficient.
        
        \end{minipage}
        }
    \end{center}

    \vspace{3mm}

    \begin{center}
      \fcolorbox{black}{lightyellow}{
        \begin{minipage}{85mm}  \setlength{\baselineskip}{10pt}
          {\BlueEmph{XANES is harder to fully interpret than EXAFS}}
          \vspace{2mm}
        
          The exact physical and chemical interpretation of all spectral
          features is still difficult to do accurately, precisely, and
          reliably.

          \vspace{1mm}
        
          This situation is improving, so stay tuned to the progress in
          XANES calculations \ldots.
        
        \end{minipage}
        }
    \end{center}

  \end{cenpage}
\vfill
\end{frame} 
