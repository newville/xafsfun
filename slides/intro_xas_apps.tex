
\begin{frame}\frametitle{XAS uses: a variety of scientific fields}

XAS can de done with a variety of sample environments, and so can
compliment other measurements.

\vmm
{\Blue{Main XAS Application Fields:}}

\begin{itemize}
\item {\RedEmph{Material Science / Physics}}:   role of  metal-oxygen coordination on material properties.
\item {\RedEmph{Biology}}:  active site coordination chemistry in metalo-proteins.
\item {\RedEmph{Earth Sciences}}: trace element speciation in minerals,
  meteorites.
\item {\RedEmph{Environmental Sciences}}: fate and speciation of metal
  contaminants in soils.
\item {\RedEmph{Catalysis/nanoscience}}:  structure, chemistry of
  industrial important catalysts, fundamentals of nanoparticles, {\emph{in
      situ}} and/or {\emph{in operando}}.
\item {\RedEmph{Energy Sciences}}:  changes in speciation and structure
  during battery cycling.
\end{itemize}

\vmm\vmm
{\Blue{Other XAS variations and related techniques:}}

\begin{itemize}
\item {\RedEmph{Magnetism}}:   X-ray Magnetic Circular Dichroism uses the
  spin-state dependence of XAS to probe magnetism of particular atoms.
\item {\RedEmph{Dynamics}}:  Using the timing structure of the synchrotron
  X-ray pulses,  XAS can probe changes in chemical state at many time
  scales: nano-seconds to seconds.
\end{itemize}

\end{frame}
