%% Slide
\begin{slide}{The EXAFS Equation: simple description}

   \begin{center}
     \begin{postitbox}{98mm}
       The XAFS -- the change in $\mu$ -- is the {\BlueEmph{overlap}} of the
       returning, scattered photo-electron with the tightly bound core electron
      \end{postitbox}
    \end{center}

 \begin{tabular}{ll}

   \begin{minipage}{60mm}
     With a spherical wave for the
     photo-electron:
     \[ \psi(k,r) = {{e^{ikr}}/{kr}} \]

     $\chi(k)$ is due to the photo-electron:
     \begin{enumerate}
       {\onslide+<2-> \item leaves the absorbing atom}
       {\onslide+<3-> \item scatters from the neighbor atom}
       {{\onslide+<4-> \item returns to the absorbing atom}}
     \end{enumerate}
   \end{minipage}
   &
   {\onslide+<1->
     \begin{minipage}{50mm}
       \rgraph{48mm}{xafscartoon_scatter}
     \end{minipage}
   }
\end{tabular}

\vmm

     \[ {\displaystyle  {\onslide+<2->\chi(k) = {{e^{ikR}}\over{kR}} \> }
       {\onslide+<3-> [{\Red{2k f(k)e^{i\delta(k)}}} ] \> }
       {\onslide+<4-> {{e^{ikR}}\over{kR}}}
       {\onslide+<5-> + C.C. }}
     \]

  \onslide+<3->

  \begin{cenpage}{80mm}

  \begin{description} \settowidth{\labelwidth}{5mm} \setlength{\itemindent}{-5mm}
  \item[{{\Red{ $f(k)$}} \hspace{1mm}}] the scattering amplitude for the atom.
  \item[{ {\Red{$\delta(k)$}} \hspace{1mm}}] the scattering phase-shift for the atom.
  \end{description}
  \end{cenpage}

\vfill
\end{slide}

%% Slide
\begin{slide}{Development of the EXAFS Equation: Coordination Sphere}

  Including the complex conjugate, and insisting on a real result, we get

   \[ \chi(k) = \frac{f(k)}{kR^2} {\sin[{2kR + \delta(k)}]} \]

    The EXAFS Equation for 1 scattering atom.

    \vmm
    \onslide+<2->

    For $N$ neighboring atoms, with (thermal and static) disorder in the
    distribution of $R$  described by  $\sigma^2$, we have

     \[ \chi(k) = \frac{N f(k) e^{-2k^2\sigma^2}}{kR^2} \sin{[2kR + \delta(k)]} \]

     \onslide+<3->

     \begin{columns}
       \begin{column}{45mm}
         In general, we would integrate over the {\RedEmph{Partial}} Pair
         distribution function, $g(R)$.  Using $N$, $R$, and  $\sigma^2$,
         is a common simplification.

       \end{column}
       \begin{column}{55mm}
         \rgraph{52mm}{gnxas}
       \end{column}

     \end{columns}


 \vfill

 \end{slide}
