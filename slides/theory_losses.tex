%% Slide
\begin{slide}{What we left out: Extrinsic Loss Terms}

    \vmm
    
    The photo-electrons are quasiparticles, and can inelastically scatter.

    \vmm 

    Extrinsic Losses include:
    
    \begin{description}
    \item[{\RedEmph{self-energy}}] $\Sigma$.  The potential for a one-particle Dyson
      equation becomes 
      
      \[ {\cal{H}}\psi = \Bigl[ -{\frac{1}{2}}\bigtriangledown^2 + V_{\rm
        coul} + V_{\rm core-hole} + \Sigma(E) \Bigr] \psi = E\psi
      \]

      
      The self-energy for the {\RedEmph{excited atom}}, goes beyond
      ``normal'' ground state DFT, and follows work of L. Hedin and S.
      Lundqvist J. Phys C {\bf{4}}, 2064 (1971).

      
    \item[{\RedEmph{core-hole width}}] $\Gamma$.  This ranges from ~0.1 to
      10 eV for most deep core-levels, corresponding to a
      {\RedEmph{core-hole lifetime}} $\tau$ on the order of 1 fs.  ($\Gamma
      \tau \approx \hbar$).

    \end{description}

    \vmm \hrule \vmm

    Conceptually, these two terms can be combined into a {\BlueEmph{mean
        free path}}, $\lambda(k)$,  and the photo-electron wave function be changed from
    
    \begin{equation*} 
      \psi(k,r)  =  {\frac{e^{ikr}}{kr}} \Rightarrow   {\frac{e^{ikr}e^{-r/\lambda(k)}}{kr}}
      \Rightarrow {\frac{e^{ipr}}{pr}}
    \end{equation*} 

    or we can use a complex wavenumber of the photo-electron:

    \[p(k) = k + i / \lambda(k) \] 


\vfill
\end{slide} 
