
\begin{slide}{Fourier Filtering}
  \small \setlength{\baselineskip}{10pt}
  \begin{center}
  \begin{tabular}{ll}
    \begin{minipage}{65mm}  {\wgraph{65mm}{reduction/chir_win}}  \end{minipage}
    &
    \begin{minipage}{37mm}  \setlength{\baselineskip}{10pt}
      {\Red{${\chi(R)}$}} often has well-separated peaks for 
      different ``shells''. \vspace{0.5mm}      

      This shell can be isolated by a Filtered Back-Fourier Transform,
      using the window shown for the first shell of FeO.
       \vspace{1mm}

    \end{minipage}\\
    \begin{minipage}{65mm}  
      \onslide+<2->
      {\wgraph{65mm}{reduction/chiq}}
    \end{minipage}
    &
    \begin{minipage}{37mm} \setlength{\baselineskip}{10pt}
      \onslide+<2->{
      \vspace{1mm} 
      
      This results in the filtered ${\chi(k)}$ for the selected
      shell.  

      \vspace{1mm}
      
      Many analysis programs use such filtering to remove shells at higher
      $R$.
      
      Beyond the first shell, isolating a shell in this way can be
      difficult.

      \vspace{3mm}
    }
    \end{minipage}
  \end{tabular}
  \end{center}    

\vfill
\end{slide} 
