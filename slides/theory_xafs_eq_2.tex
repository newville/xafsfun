%% Slide
\begin{slide}{The EXAFS Equation}

    
  We have an equation we can use to model and interpret EXAFS:
    
  \begin{center}
    \highlightbox{
      \begin{minipage}{100mm} 
        \[ \chi(k) = \sum_j \frac{ {\Blue{N_j}} S_0^2 {\Red{f_j(k)}}
            e^{-2{\Blue{R_j}}/{\Red{\lambda(k)}}}\, 
            e^{-2k^2{\Blue{\sigma_j^2}}}}{k{\Blue{R_j}}^2}
          \sin{ [ 2k{\Blue{R_j}} + \Red{\delta_j(k)} ]} \] \end{minipage}
      }
    \end{center}

    
    \vmm
    where the sum is over ``shells'' of atoms or ``scattering paths'' for
    the photo-electron -- nearly the same concept.

    \vmm If we know the {\RedEmph{scattering}} properties of the
    neighboring atom: ${{\Red{f(k)}}}$ and $\Red{\delta(k)}$, and the
    mean-free-path $\Red{\lambda(k)}$ we can determine:
   
      
    \vmm
    \begin{description}  \settowidth{\labelwidth}{10mm}
      \setlength{\itemindent}{5mm}
    \item[$\Blue{R}$]  distance to neighboring atom.  
    \item[$\Blue{N}$]  coordination number of neighboring atom.
    \item[$\Blue{\sigma^2}$] mean-square disorder of    neighbor distance.
    \end{description}

    
    The scattering amplitude $\Red{f(k)}$ and phase-shift $\Red{\delta(k)}$
    depend on atomic number, so that XAFS is also sensitive to
    {\BlueEmph{Z}} of the neighboring atom.
    
\end{slide} 
