
\begin{slide}{XAFS Data Reduction}
  We begin with {\BlueEmph{Data Reduction}}:

  \onslide+<2->
  \begin{center}
    Converting measured data to ${\mu(E)}$ and then to ${\chi(k)}$
  \end{center}
  
  \vmm
  \onslide+<3->
  This won't tell us $R$, $N$, and neighbor species, but it will:
  
  \vmm
  \begin{enumerate}
  \item help us determine data quality
    
    \onslide+<4->
  \item can be useful for ``spectroscopic'' analysis: \par fingerprinting,
    linear combinations of spectra, etc.

    \onslide+<5->
  \item needed to get to ${\chi(k)}$ for further modeling.
  \end{enumerate}

  \vmm \vfill
\end{slide}

\begin{slide}{XAFS Data Reduction: Strategy}

  Step for reducing measured data to ${\mu(E)}$ and then to ${\chi(k)}$:

  \begin{enumerate}
  \item convert measured intensities to ${\mu(E)}$
    \pause
  \item subtract a smooth pre-edge function, to get rid of any instrumental
    background, and absorption from other edges. 
    \pause
  \item normalize ${\mu(E)}$ to go from 0 to 1, so that it represents the
    absorption of 1 x-ray.
    \pause
  \item remove a smooth post-edge background function to approximate
    ${\mu_0(E)}$ to isolate the XAFS ${\chi}$.
    \pause
  \item identify the threshold energy ${E_0}$, and convert from
    ${E}$ to ${k}$ space: ${ k=
      \sqrt{{{2m(E-E_0)}\over{\hbar^2}}} }$
    \pause
  \item (optional) weight the XAFS ${\chi(k)}$ and Fourier transform from
    ${k}$ to ${R}$ space.
    \pause
  \item  (optional) isolate the ${\chi(k)}$ for an individual
  ``shell''  by Fourier filtering.
  \end{enumerate}


  \vmm
%   After we get this far, we'll model ${f(k)}$ and
%   ${\delta(k)}$ and analyze ${\chi(k)}$ to get distance
%   ${R}$, coordination number ${N}$.


\vfill
\end{slide} 
