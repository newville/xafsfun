\begin{slide}{The EXAFS Equation: Even more stuff we left out \ldots}

  These effects also need to be considered for  quantitative  EXAFS
  calculations:

    \vspace{2mm}

    \begin{minipage}{102mm}
    \begin{description}
    \item[\RedEmph{Curved Wave Effects}] scattering calculation needs
      a partial wave expansion.

    \item[\RedEmph{Muffin-Tin Approximation:}] scattering calculation needs
      a real-space potential, and a muffin-tin approximation is  tractable,
      but not perfect.

    % \item[\RedEmph{Polarization Effects}] synchrotron beams are highly
    %   polarized, which needs to be taken into account in anisotropic
    %   samples.  This is simple for $K$-edges ($s\rightarrow p$ is dipole),
    %   and less so for $L$-edges (where both $p \rightarrow d$ and $p
    %   \rightarrow s$ contribute).

    \item[\RedEmph{Disorder Terms}]  thermal and static disorder in real
      systems should be properly considered: A topic of its own.

    \end{description}

    \end{minipage}

    \vspace{3mm}

    We'll skip the details on these for now.

\vfill
\end{slide}

\begin{slide}{The EXAFS Equation (one last time!)}

  Even with all those complications and caveats, we still use
  the {\BlueEmph{EXAFS Equation}}:
  \vspace{-1mm}

  \begin{center}
    \[ \chi(k) = \sum_j {\frac{{\Blue{N_j}} {\Red{f_j(k)}}
        e^{-2{\Blue{R_j}}/{\Red{\lambda(k)}}}
        e^{-2k^2{\Blue{\sigma_j^2}}}}{k{\Blue{R_j}}^2}
      {\sin[{2k{\Blue{R_j}} + {\Red{\delta_j(k)}}] }}} \]
  \end{center}

  \vmm

  where $\Red{f(k)}$ and $\Red{\delta(k)}$ are
  {\RedEmph{photo-electron scattering properties}} of the neighboring
  atom and $ \Red{\lambda(k)} $ is the photo-electron mean-free-path.

  \vmm

  \begin{center}
    \begin{postitbox}{105mm}
      But now we will keep in mind that $\Red{f(k)}$, $\Red{\delta(k)}$,
      and $\Red{\lambda(k)}$       might include some of the complications
      {\BlueEmph{AND}} the sum over paths includes multiple-scattering
    \end{postitbox}
  \end{center}

  \vmm
Again, if we know these properties, we can determine:
  \onslide+<2->
    \begin{description}
      \settowidth{\labelwidth}{15mm}
      \setlength{\itemindent}{15mm}
      \setlength{\leftmargin}{15mm}
    \item[$R$] distance to neighboring atom.
    \item[$N$] coordination number of neighboring atom.
    \item[$\sigma^2$] mean-square disorder of neighbor distance.
    \end{description}

  \vmm
  \onslide+<3>
  ${\Red{f(k)}}$ and ${\Red{\delta(k)}}$ depend on atomic number
  {\BlueEmph{Z}} of the scattering atom, so we can also determine the
  species of the neighboring atom.

\end{slide}
