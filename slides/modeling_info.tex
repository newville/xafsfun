%%%%%%%%%%%%%%%%%%%%%%%%%%%%%%%%%%%%%%%%%
\subsection{Information Content}
\begin{frame}
  \frametitle{XAFS Analysis: Information Content in XAFS }

  The Number of Parameters we can measure from our data is limited:

  \vmm

  \begin{center}  {\Red{
        $\displaystyle  N \approx { { 2 \Delta k \Delta R} \over{ \pi}}  $
      }}
  \end{center}

  \vmm

  where $\Red{ \Delta k}$ and $\Red{ \Delta R}$ are the $k$- and
  $R$-ranges of the data.

  Typical:  $k = [2.0, 12.0] \rm\,\AA^{-1}$ and $R =
  [1.0, 3.0]\rm\,\AA$, gives  $\sim 12$ Parameters.

  \pause \vmm \hrule \vmm

  Fit statistics, and Error Bars  need to reflect this limit.

  \vmm
  \pause

  Need to {\Red{constrain}} Parameters $R$, $N$,
  $\sigma^2$ for different paths  and different data sets (different
  edge elements, temperatures, etc)

  \vmm

  \begin{postitbox}{90mm}
    Use as much outside information about the system as possible!
  \end{postitbox}

   \vmm \pause
   It's also possible to add {\Red{restraints}} to describe external
   knowledge of the system (crystallography, Bond Valence, etc).


\end{frame}


\subsection{Building Models}
\begin{frame}
 \frametitle{XAFS Analysis: Building Models }

  The basic difficulties in  EXAFS Analysis are

  \begin{enumerate}

  \item The scattering factors \feffc{f(k)}, \feffc{\delta(k)} are
    non-trivial (we use {\feff}).

 \item The basis functions (Paths) are not very well resolved, and their
    number grows exponentially with $R$.

  \item There's not much information in a real measurement:
    \[ N_{\rm idp} \approx \frac{2\Delta k\Delta R}{\pi} \]

  \end{enumerate}

    \hrule \vmm \pause

    We address these with methods to:

    \begin{enumerate}
    \item {reduce} the number of Paths to consider (Fourier analysis).

    \item parameterize {\emph{ab initio}} calculations of
      \feffc{f(k)}, \feffc{\delta(k)} (use {\feff})

    \item reduce the number of  variables in the fit, while keeping a
      meaningful analysis.

    \end{enumerate}

    \hrule \vmm

    We parameterize the EXAFS with a physical model and then
    put {\RedEmph{Constraints}} on the  %%% and {\RedEmph{Restraints}} on the
    parameters in a least-squares fit.

    \vmm
\end{frame}

\subsection{Constraints}
\begin{frame}[fragile]
  \frametitle{Constraints in {\larch} /  {\artemis} }

  All Path Parameters written as expressions of  Variables refined in Fit.
  \vmm

  \setbeamertemplate{blocks}[rounded][shadow=true]

  \begin{tabular}{ll}
    \begin{minipage}{45mm}
      \begin{block}
        {Parameter = Variable}
{\tiny{\begin{alltt}

{\Red{guess e0    = 1.0 }}
path(1,  e0 = e0)
path(2,  e0 = e0)
\end{alltt}}}
 \end{block}
   \vspace{1.75mm}
   \onslide+<2->
   \begin{block}
{mixed coordination shell}

{\tiny{\begin{alltt}
 set S02   = 0.80

 {\Red{guess x   = 0.5}}

 path(1,  Amp= S02 * x )
 path(2,  Amp= S02 * (1-x))

\end{alltt}}}
   \end{block}
   \end{minipage}
    & \hspace{0.2mm}
    \begin{minipage}{65mm}
      \onslide+<3->   \begin{block}
 { Fit Einstein Temperature }

{\tiny{\begin{alltt}

 set   factor  = 24.254337   {\Blue{#= (hbar*c)^2/(2 k_boltz)}}
{\Blue{# mass and reduced mass in amu}}
 set   mass1 = 63.54,  mass2 = 63.54
 set   r_mass =  1/ (1/mass1 +  1/mass2)

{\Blue{# the Einstein Temp will be adjusted in the fit!}}
 {\Red{guess thetaE = 200}}
{\Blue{# use for data set 1, T=77}}
 set   temp1 = 77
 {\Red{def ss2_path1 = factor*coth(thetaE/(2*temp1))/r_mass )}}
 path(101,  sigma2 = ss2_path1   )

{\Blue{# use for data set 2, T=300}}
 set   temp2 = 300
 {\Red{def ss2_path2 = factor*coth(thetaE/(2*temp2))/r_mass )}}
 path(201,  sigma2 = ss2_path2   )
\end{alltt}}}
        \end{block}
    \end{minipage}\\
  \end{tabular}

  \vmm \vmm

  \onslide+<4-> Other Examples:

\begin{itemize}
\item force one $R$ for the same bond for data taken from different  edges.
\item model complex distortions (height of a sorbed atom above a surface).
\end{itemize}

\end{frame}


\subsection{Fit statistics}
\begin{frame}\frametitle{Fitting with {\ifeffit} / {\artemis} }

  {\ifeffit} optimizes the Fitting Parameters with a least-squares fit to the
  Data

  \begin{postitbox}{90mm}
    Find the variables that make the Model best match the Data
  \end{postitbox}


  \pause \vmm


  $\chi^2$ (don't confuse with EXAFS $\chi$!!) describe the fit:

  \[
  \chi^2  =   \sum_i^{N_{\rm fit}} \frac{[\chi_i^{\rm data} - \chi_i^{\rm
      model}({x})]^2}{\epsilon^2}
  \]

  ${N_{\rm fit}} = $ number of data points, ${x} = $ set of variables,
  $\epsilon =$ noise level in the data. \pause

  We should consider only $ N_{\rm idp} $ data points:

  \begin{postitbox}{88mm}
    \[   \chi^2  =  \frac{ N_{\rm idp}}{\epsilon^2 N_{\rm fit}}
    \sum_i^{N_{\rm fit}} [\chi_i^{\rm data} - \chi_i^{\rm model}({x})]^2 \]
  \end{postitbox}

  \pause   Fitting is typically done in $R$-space to ignore higher shells.

\end{frame}

\subsection{More Fit Statistics}
\begin{frame}
\frametitle{Goodness of Fit and Error Bars}

  Goodness-of-Fit statistics:
  \begin{itemize}[<+->]
  \item{\Red{chi-square}}:     $  \chi^2  =  \frac{ N_{\rm idp}}{\epsilon^2 N_{\rm fit}}
    \sum_i^{N_{\rm fit}} [\chi_i^{\rm data} - \chi_i^{\rm model}({x})]^2   $

  \item{\Red{reduced chi-square}}:  scale $\chi^2$ by the  ``degrees of freedom''

    $ \chi^2_\nu =  \chi^2 / (N_{\rm idp}-N_{\rm varys}) $

    A Good Fit should have $\chi^2_\nu \sim 1$. This {\RedEmph{never}}
    happens!

    $ \chi^2_\nu \sim 10 $ or higher, typically.


  \item{\Red{R-factor}}:  Fractional misfit.
    \[
    {\cal{R}} =
    {\sum_i^{N_{\rm fit}}[\chi_i^{\rm data} - \chi_i^{\rm model}({x})]^2 }
    /
    { \displaystyle{\sum_i^{N_{\rm fit}} [{\chi_i^{\rm data}}]^2}}
    \]

  \end{itemize}

\hrule \vmm \onslide+<4->

  Error bars for Fitting Parameters are calculated, and
  increase $\chi^2$ by $\chi^2_\nu$.

  \vmm
  Correlations between parameters are also calculated.

\end{frame}
