%% Slide
\begin{slide}{Development of the EXAFS Equation}

      \vmm
    \begin{cenpage}{125mm}

      
    \[ \chi(k) = \frac{f(k)}{kR^2} {\sin[{2kR + \delta(k)}]} \]

    \vmm
    The EXAFS Equation for 1 scattering atom.

    \vmm\vmm \hrule \vmm \vmm

    \onslide+<2->

    For $N$ neighboring atoms, and with thermal and static disorder of
    $\sigma^2$, giving the {\BlueEmph{mean-square disorder}} in ${R}$, we
    have

    \[ \chi(k) = \frac{N f(k) e^{-2k^2\sigma^2}}{kR^2} \sin{[2kR + \delta(k)]} \]


    A real system has atoms at different distances and of different types.
    We add all these contributions to get a better version of the EXAFS
    equation:

      \vspace{2mm}

      \begin{center}
        \begin{postitbox}{77mm}
    \[ \chi(k) = \sum_j {\frac{{\Blue{N_j}} {\Red{f_j(k)}}
        e^{-2k^2{\Blue{\sigma_j^2}}}}{k{\Blue{R_j}}^2}
      {\sin[{2k{\Blue{R_j}} + {\Red{\delta_j(k)}}] }}} \]
  \end{postitbox}
      \end{center}

\vfill

\end{cenpage}

\end{slide}
