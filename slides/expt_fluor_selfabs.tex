\begin{slide}{Fluorescence Measurements: Self-Absorption}

    The fluorescent x-ray has to get out of the sample, and can be
    attenuated by the sample itself.  This {\RedEmph{self-absorption}} by
    can damp the XAFS -- even completely wiping it out for highly
    concentrated elements.  \vspace{2mm}

    The measured fluorescence intensity goes as:

    \[ {
      I_f = I_0 {{\bepsilon\bDelta\bOmega}\over{4\bpi}} \,
      {{\bmu_{\bchi}(E)}\over{\bmu_{\rm tot}(E)+\bmu_{\rm tot}(E_f)}}
      \,
      \big[ 1- e^{-[\bmu_{\rm tot}(E) + \bmu_{\rm tot}(E_f)]t}\big]
      }
    \]
 
    where 

    \begin{tabular}{ll}\setlength{\parskip}{1pt}
       {\Blue{$\bepsilon$}} &is the fluorescence efficiency.\\
       {\Blue{$\bDelta\bOmega$}} &is the solid angle of the detector.\\
       {\Blue{${E_f}$}} &is the energy of the fluorescent x-ray.\\
       {\Blue{${\bmu_{\bchi}(E)}$}} &is the absorption from the
       element of interest. \\ 
  
       {\Blue{${\bmu_{\rm tot}(E)}$}} & is the {\emph{total}}
       absorption in the sample:\\
   \end{tabular}

    \[{  \bmu_{\rm tot}(E) = \bmu_{\bchi}(E) + \bmu_{\rm other}(E)    }\]

  \vspace{2mm}

  \begin{center}
    \fcolorbox{black}{lightyellow}{
      \begin{minipage}{75mm}  
        When ${\mu_{\chi}(E)}$ dominates ${\mu_{\rm tot}(E)}$,
        the fluorescence intensity will be severely attenuated.
      \end{minipage}
      }
  \end{center}    

  \vfill

\end{slide} 
