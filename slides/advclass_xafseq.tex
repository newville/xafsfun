
%% Slide
\begin{slide}{The EXAFS Equation in IFEFFIT}
    
    The simplest useful version of the EXAFS Equation is

    {\large
      \begin{center}
        \fcolorbox{black}{lightyellow2}{
          $ \displaystyle {\bchi(k) = {{f(k)e^{-2R/\lambda(k)}}\over{kR^2}}
            {\bsin[{2kR + \bdelta(k)]}} }
          $ 
        }
      \end{center}  }
    
    \vmm
        
    which is the contribution for 1 scattering path with at atom at
    distance ${R}$.

    \vmm
    For real analysis we need to include  effects of

    \begin{itemize}\setlength{\itemindent}{3mm}
    \item paths at several distances (a ``shell'').
    \item having ${N}$ neighboring atoms in a given ``shell''.
    \item paths with different scattering atom types.
    \item thermal and static disorder: a range the distances within a ``shell''.
    \end{itemize}


    The simplest approach (ie, usually the best approach) is:


    \begin{enumerate}\setlength{\itemindent}{3mm}
    \item sum over independent paths.
    \item multiply each path's contribution by  ${N}$: the
      ``coordination number''.
    \item wrap thermal and static disorder into ${\sigma^2}$: a
      mean-square-displacement.
    \end{enumerate}
    
    This doesn't cover every case, but it's a start.

\vfill
\end{slide} 

\begin{slide}{The EXAFS Equation}
    
    With a sum over paths and model for disorder, we have the
    EXAFS Equation:
    
    {\large
      \begin{center}
        \fcolorbox{black}{lightyellow2}{
          ${\displaystyle \bchi(k) = \sum_j {{{\Blue{N_j}} S_0^2 {\Red{f_j(k)}}
                e^{-2{\Blue{R_j}}/{\Red{\blambda(k)}}}\, 
                e^{-2k^2{\Blue{\bsigma_j^2}}}}\over{k{\Blue{R_j}}^2}}
            {\sin[{2k{\Blue{R_j}} + {\Red{\bdelta_j(k)}}] }}}$
        }
      \end{center}
    }
    
    \vmm \vmm
    this sum over photo-electron ``scattering paths''  may include multiple scatterings.
    
    \vmm \hrule \vmm
    
    If we know the {\RedEmph{scattering}} properties of the neighboring
    atom: \feffc{f(k)}, \feffc{\delta(k)}, and the mean-free-path
    \feffc{\lambda(k)} (from {\feff}) we can determine:
   
      
    \vmm
    \begin{cenpage}{80mm}
      \begin{description}  \setlength{\itemindent}{3mm}
      \item[${{\Blue{R}}}$]  distance to neighboring atom.  
      \item[${{\Blue{N}}}$]  coordination number of neighboring atom.
      \item[${{\Blue{\bsigma^2}}}$] mean-square disorder of
      neighbor distance.
      \end{description}
    \end{cenpage}

    \vmm \vmm
  
    The scattering amplitude \feffc{f(k)} and phase-shift \feffc{\delta(k)}
    depend on atomic number, so that XAFS is also sensitive to
    {\BlueEmph{Z}} of the neighboring atom.
    
  \vfill
\end{slide} 



%% Slide
\begin{slide}{The EXAFS Equation in IFEFFIT }
    \vmm  
    {\ifeffit} uses a slightly more complex version of the EXAFS Equation:
    
    \[ 
    {\displaystyle{ 
        \bchi_{\rm model}(k) = {\rm Im} \Biggl\{ 
        \sum_{j=1}^{N_{\rm Paths}}
        \bchi_j(k, {\Red{\rm FEFF}}_j, {\Blue{\rm Parameters_j}}). }}
    \Biggr\}
    \]
    
    where \vmm
    
    \fcolorbox{black}{lightyellow2}{\begin{minipage}{103mm}
        \begin{eqnarray*}
          {\bchi_j(k)} = & 
          {\displaystyle{ \hspace{-2mm}
              \frac{\feffc{f(k)}\times \feffc{N} \times {\pthpar{S02}}}
              {k( \reff + \pthpar{delR} ) ^2}
            }
            \enskip\bexp(-2p''\reff - 2p^2\pthpar{sigma2} + \twothirds p^4 \pthpar{fourth}) } \\
          &  
          {\times \enskip\bexp \biggl\{ i\big[
            2k\reff + \feffc{\delta(k)}
            + 2p(\pthpar{delR} - 2{\textstyle{\pthpar{sigma2}\over\reff }})
            -\fourthirds p^3\pthpar{third}
            \big]  \biggr\}     } 
        \end{eqnarray*}
      \end{minipage} }
    
    \vmm 
    where the components from {\Red{\feff}} and the {\Blue{Path Parameters}} depend on Path
    $j$.  
    ${k}$ is the ``real'' photoelectron wavenumber:
    
    \vspace{-1mm}
    \[ {k} = {\sqrt{ \feffc{k_{\rm FEFF}}^2 - \pthpar{e0}\,
        \masse}}\hspace{24mm}
    \]
    
    and {{${p = p' + ip''}$}} is the {\RedEmph{complex
        photoelectron wavenumber}} 
    from {\feff}:
    
    \vspace{-1mm}
    \[
    {p} = {  \sqrt{ \Bigl[ {\feffc{real\_p(k)}} +
        { i / \feffc{\lambda(k)}} \Bigr]^2 - i \,  \pthpar{ei} \, \masse  }} 
    \]
    
  \vfill
\end{slide} 




%% Slide
\begin{slide}{The EXAFS Equation in IFEFFIT (part 2)}

    
    \fcolorbox{black}{lightyellow2}{
      \begin{minipage}{103mm}
        \begin{eqnarray*}
          {\bchi_j(k)} = & 
          {\displaystyle{
              \frac{\feffc{f(k)}\times \feffc{N} \times {\pthpar{S02}}}
              {k( \reff + \pthpar{delR} ) ^2}
            }
            \enskip\bexp(-2p''\reff - 2p^2\pthpar{sigma2} + \twothirds p^4 \pthpar{fourth}) } \\
          &  
          {\times \enskip\bexp \biggl\{ i\big[
            2k\reff + \feffc{\delta(k)}
            + 2p(\pthpar{delR} - 2{\textstyle{\pthpar{sigma2}\over\reff }})
            -\fourthirds p^3\pthpar{third}
            \big]  \biggr\}     } \\
          {p^2} = & {   \Bigl[ {\feffc{real\_p(k)}} +
            { i / \feffc{\lambda(k)}} \Bigr]^2 - i \,  \pthpar{ei} \, \masse } \hspace{35mm} \\
          {k^2} = & {\feffc{k_{\rm FEFF}}^2 -  \pthpar{e0}\,
          \masse} \hspace{56mm}  
      \end{eqnarray*} 
    \end{minipage}       
  }
  
  \vmm 
  A few notes on this mess:
    
  \begin{itemize}

  \item \feffc{f(k)}, \feffc{\delta(k)}, \feffc{\lambda(k)}, \feffc{N},
    \reff, \feffc{real\_p(k)}, and \feffc{k_{\rm FEFF}} all come from
    {\feff} (\feffndat).
    
    
  \item The Path Parameters \pthpar{S02}, \pthpar{e0}, \pthpar{ei}, 
    \pthpar{delR}, \pthpar{sigma2}, \pthpar{third}, and \pthpar{fourth}  are
    all simple real numbers with physical meaning, and can be adjusted in
    a fit.
    
  \item $p$ -- which includes the mean-free-path -- is used in the EXAFS
    equation for the cumulants ( \pthpar{delR}, \pthpar{sigma2},
    \pthpar{third}, and \pthpar{fourth})
   
  \item $k$ is used to reconstruct {\feff}'s ${\chi(k)}$ and to
    refer to the data's energy grid.
  \end{itemize}

  \vfill
\end{slide} 


%% Slide
\begin{slide}{An IFEFFIT Path }
    
    \vmm {\ifeffit} uses a {\tt{path()}} command like this:

    \vmm \vmm 

    \begin{tabular}{ll}
        \begin{CodeBlock}{40mm}{Path Command}
path({\Blue{index}}  = 1,  
     {\Blue{feff }}  = directory/feff0001.dat,
     {\Blue{degen}}  = 12, 
     {\Blue{e0   }}  = e0,
     {\Blue{s02  }}  = S02 * N\_1, 
     {\Blue{delr }}  = deltar, 
     {\Blue{sigma2}} = ss2,  
     {\Blue{third}}  = third\_expr, 
     {\Blue{fourth}} = fourth\_expr, 
     {\Blue{ei   }}  = ei\_expr ) 
   \end{CodeBlock}
   & 
   \begin{minipage}{49mm}

     Legend:

     \begin{description}
     \item[{\pthpar{index}}]: internal Path Number.
     \item[{\pthpar{feff}}]:  name of {\feffndat} file.
     \item[{\pthpar{degen}}]: override {\feff}'s degeneracy {\feffc{N}}
     \item[{\pthpar{e0}} \ldots {\pthpar{ei}}]:   numerical parameters to go into 
       {\ifeffit}'s EXAFS Equation.
     \end{description}
     
       \vmm \vmm
       

    \end{minipage}\\
    
  \end{tabular}    
    
  \vmm \vmm  \vmm 
  

  In {\ifeffit} (and {\Program{artemis}},{\Program{feffit}}, and
  {\Program{sixpack}}), the ``value'' (right-hand-side) for each Numerical  Path
  Parameter 
  {\pthpar{e0}},{\pthpar{s02}},{\pthpar{delr}}, \ldots {\pthpar{ei}}
  can be a {\RedEmph{formula}}, written in terms of 
  user-defined variables.

  \vmm \vmm  \vmm 

  More on this later \ldots.
  \vfill
\end{slide} 

