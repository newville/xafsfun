%% Slide
\begin{slide}{Photo-Electron Mean-Free Path}

 We used a spherical wave for the photo-electron, $ \psi(k,r) \sim
 {{e^{ikr}}/{kr}}  $ to get to

    \[ \chi(k) = \sum_j {\frac{{\Blue{N_j}} {\Red{f_j(k)}}
        e^{-2k^2{\Blue{\sigma_j^2}}}}{k{\Blue{R_j}}^2}
      {\sin[{2k{\Blue{R_j}} + {\Red{\delta_j(k)}}] }}} \]

 
  \begin{cenpage}{95mm}
    The photo-electron can also scatter {\BlueEmph{inelastically}}, and may
    not be able to get back the absorbing atom in tact (in phase, at energy).

    \vmm

    {\BlueEmph{Plus}}: the core-level has a {\RedEmph{finite lifetime}},
    before it is filled.   This also limits how far the photo-electron can
    go out.
  \end{cenpage}

  \vmm \pause

    A {\RedEmph{mean free path}} -- $\lambda$ -- describes how far the
    photo-electron can go before it loses energy/coherence.


    \begin{cenpage}{50mm}
      {$\displaystyle \psi(k,r) \sim \frac{e^{ikr} e^{-r/\lambda(k)}}{kr} $ }
    \end{cenpage}


\end{slide}

%% Slide
\begin{slide}{Photo-Electron Mean-Free Path}

Using  {$\displaystyle \psi(k,r) \sim \frac{e^{ikr} e^{-r/\lambda(k)}}{kr} $ }
adds a term to the EXAFS equation:
\vmm

    \[ \chi(k) = \sum_j {\frac{{\Blue{N_j}} {\Red{f_j(k)}}
        e^{-2{\Blue{R_j}}/{\Red{\lambda(k)}}}
        e^{-2k^2{\Blue{\sigma_j^2}}}}{k{\Blue{R_j}}^2}
      {\sin[{2k{\Blue{R_j}} + {\Red{\delta_j(k)}}] }}} \]

\vmm\pause

\begin{columns}
  \begin{column}{58mm}
    \rgraph{59mm}{lambda}
  \end{column}
  \begin{column}{58mm}

  \begin{itemize}
  \item  $\lambda \lesssim 30\,\rm\AA$  for $k > 3\rm\,\AA^{-1}$.

  \item This (and the $R^{-2}$) makes EXAFS a {\RedEmph{local
        atomic probe}}.
  \item  $\lambda $ increases at low $k$. \par 

  \item XANES is less a {\RedEmph{local probe}} than EXAFS.
  \end{itemize}
  \end{column}
\end{columns}

\vmm

$\lambda$ has a universal and strong $k$ dependence, but is mostly
independent of the material.


\end{slide}
