\begin{slide}{EXAFS Analysis Example: Modeling the 1st Shell of FeO}

    \begin{tabular}{ll}
      \begin{minipage}{85mm} 
        FeO has a rock-salt structure.
        \vmm\vmm
    
        To model the FeO EXAFS, we'll calculate the scattering amplitude
        ${{\Red{f(k)}}}$ and phase-shift ${{\Red{\delta(k)}}}$, based on a
        guess of the structure, with Fe-O distance ${R=2.14\rm\,\AA}$ (a
        regular octahedral coordination).
        
        \vspace{2mm}
        
        We'll use these functions to {\BlueEmph{refine}} the values
        {\Blue{${R}$}}, {\Blue{${N}$}},
        {\Blue{${\sigma^2}$}}, and {\Blue{${E_0}$}} so our
        model EXAFS function matches our data.
        \vspace{2mm}

      \end{minipage}
      &
    \begin{minipage}{22mm} {\wgraph{20mm}{molecules/feo}}
    \end{minipage}

    \end{tabular}

 
    \begin{tabular}{ll}
      \begin{minipage}{65mm} {\wgraph{60mm}{fits/feo_r_1sh_mag}}  
        ${|\chi(R)|}$ for FeO {\Blue{data}} and {\Red{${\rm 1^{st}}$ shell fit}}.
      \end{minipage}
      &
      \begin{minipage}{30mm}  \setlength{\baselineskip}{10pt}
        \vspace{1mm} 
        Fit results:   \vspace{2mm}
        \begin{tabbing}[ll]\= aaaaaaa\= aaaaaaaaaaaaaaaa\kill 
          \> ${N}$           \>= 5.8 ${\pm}$ 1.8\\
          \> ${R}$           \>= 2.10 ${\pm}$ 0.02\AA\\
          \> ${\Delta E_0}$ \>= -3.1 ${\pm}$ 2.5 eV\\
          \> ${\sigma^2}$   \>= 0.015 ${\pm}$ 0.005
          ${\rm\,\AA^2}$.\\
          \end{tabbing}

        \vfill
    \end{minipage}
  \end{tabular}
  
\vfill
\end{slide} 
