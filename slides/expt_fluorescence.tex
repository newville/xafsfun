
\begin{slide}{X-ray Absorption Measurements: Fluorescence}
  
    \begin{center}
      \scalebox{1}{\wgraph{50mm}{experiment/cartoon}}
    \end{center}
    
    \small For low concentrations (down to the ppm level), monitoring the
    x-ray fluorescence is the preferred measurement.
    
    \vspace{2mm}
    \pause

    \begin{tabular}{ll}
      \begin{minipage}{55mm}
        {\wgraph{55mm}{experiment/fluor_spectra}}
      \end{minipage}
      &
      \begin{minipage}{43mm}
        x-rays emitted from the sample l include the fluorescence
        line of interest (here, both ${\rm Fe\>K_{\balpha}}$ and
        ${\rm Fe\>K_{\bbeta}}$) as well as {\RedEmph{scattered}} (elastic
        and inelastic) x-rays, and other fluorescence lines.

        \vspace{5mm}
      \end{minipage}
    \end{tabular}

    \pause 
%     
%     There are both {\BlueEmph{elastically scattered}} (at the same energy
%     as the incident beam), and {\BlueEmph{inelastically scattered}} (Compton
%     effect) x-rays.

    \vspace{1mm}    
    \begin{center}
      \fcolorbox{black}{lightyellow}{
        \begin{minipage}{90mm}  
          In many cases, the scatter or other fluorescence lines will
          dominate the fluorescence line of interest.
        \end{minipage}
        }
    \end{center}

\vfill
\end{slide} 
