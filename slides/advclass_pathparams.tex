\section{Path Parameters,  Pair Distribution Function}

%% Slide
\begin{slide}{Path Parameters: The Pair Distribution Function}
  \small
  \begin{cenpage}{100mm}\setlength{\baselineskip}{11pt}
    
    \vmm
    
    An EXAFS measurement averages billions of 1 femto-second ``snapshots''
    of the absorber-scatterer.  We put this in the 
    EXAFS Equation as $\mathbf{N}$ and $\mathbf{\sigma^2}$.

    
    \vmm 
    
    More generally, we sample a {\RedEmph{Partial Pair Distribution
        Function}}, {\RedEmph{$\mathbf{g(R)}$}} : the probability that at
    atom is a distance ${\mathbf{R}}$ away from the absorbing atom:

    \[  \displaystyle{\mathbf{
        \bchi(k)  = \Biggl\langle \sum_j 
        \frac{f(k)e^{-2R/\lambda(k)}}{kR^2} \bsin[2kR + \bdelta(k)]
        \Biggr\rangle  }}
    \]

    
    where  $  \langle f(R) \rangle = \int dR\, f(R) \, g(R) / \int dR\, g(R) $.

    \vmm \hrule \vmm
    
    We'll go back to using {\RedEmph{complex notation}}, use {$\mathbf{p
        \approx k + i/\lambda(k)}$}, and ignore the $\mathbf{R}$ dependence
    of \feffc{f(k)} and \feffc{\delta(k)}.  

    For now, we'll neglect the $R$ dependence of the $1/R^2$ too, so that:
        
    \[  \displaystyle{\mathbf{
        \bchi_j(p)  = \biggl\langle
        {f(k)\over{kR^2}} e^{i[2pR + \bdelta(k)]}  \biggr\rangle
        \approx  {f(k)e^{i\bdelta(k)}\over{kR^2}} \biggl\langle e^{i2pR}  \biggr\rangle 
      }} 
    \]
    
    \vmm We can then use {\RedEmph{the cumulant expansion}} from
    intermediate statistics, which relates $\langle e^x \rangle$ to
    $\langle x \rangle$, the moments of $g(R)$.

  \end{cenpage}
  \vfill
\end{slide} 


%% Slide
\begin{slide}{The Cumulant Expansion}
  \small
  \begin{cenpage}{98mm}\setlength{\baselineskip}{11pt}
    
    Note that we're ignoring the $\mathbf{1/R^2}$ term -- (for now!):

    \[  \displaystyle{\mathbf{
        \bchi_j(p)  
        \approx  {f(k)e^{i\bdelta(k)}\over{kR^2}} \biggl\langle e^{i2pR}  \biggr\rangle 
      }} 
    \]
    

    The cumulant expansion gives the average  of $\exp{i2pR}$ as

    \[ \displaystyle{\mathbf{
        \biggl\langle e^{i2pR} \biggr\rangle
        = \exp \bigg[ \sum_{n=1}^{\infty} { {(2ip)^n}\over{n!}}C_n  \bigg].}}
    \]
    
    \vmm $C_n$ are the {\RedEmph{cumulants}} of $g(R)$, related to the
    moments of $g(R)$: $\langle r^n \rangle$,

    with
    $r= R - R_0$ and $R_0$ is the centroid of the distribution (we'll use {\reff}):
    
    \begin{tabular}{lll} 
      \hspace{4mm} & 
      $C_1 (\Delta R) $ & $ = \langle r \rangle    $ \\
      & $C_2 (\sigma^2) $ & $ = \langle r^2 \rangle - \langle r \rangle^2    $  \\
      & $ C_3 $ & $ = \langle r^3 \rangle - 3 \langle r^2 \rangle  
      \langle r \rangle  + 2 \langle r \rangle^3   $  \\
      & $ C_4 $ & $ =  \langle r^4 \rangle - 3 \langle r^2 \rangle^2 
      - 4\langle r^3 \rangle \langle r \rangle
      +12  \langle r^2 \rangle  \langle r \rangle^2 
      - 6\langle r \rangle^4  $   \\
    \end{tabular}

    \vmm
        
    These four cumulants are {\ifeffit}'s Path Parameters for adjusting
    $g(R)$ for a path: \pthpar{delR}, \pthpar{sigma2}, \pthpar{third}, and
    \pthpar{fourth}


    \vmm \begin{center}\highlightbox{\pthpar{third} is sometimes important!! }\end{center}
    

\vmm \vmm
    
\end{cenpage}
\vfill
\end{slide} 

%% Slide
\begin{slide}{Corrections to the cumulant expansion}
  \small
  \begin{cenpage}{105mm}\setlength{\baselineskip}{11pt}
   
    
    For the cumulants we ignored the $\mathbf{1/R^2}$ term.  A simple
    Taylor series expansion gives two corrections (with $C_1$ =
    \pthpar{delR} and $R_0$ = \reff):
    
    \begin{eqnarray*}
      {\displaystyle{\mathbf{\frac{1}{k\reff^2}}}} & 
      {\displaystyle{\mathbf{\rightarrow}}} & 
      {\displaystyle{\mathbf{\frac{1}{k(\reff + \pthpar{delR})^2} }}} \\
      {\displaystyle{\mathbf{2p \pthpar{delR}}}}  &
      {\displaystyle{\mathbf{\rightarrow}}} & 
      {\displaystyle{\mathbf{2p(\pthpar{delR}- 2 \pthpar{sigma2}/{\reff}) }}} \\
    \end{eqnarray*}       



    
      For any other corrections (ie, when more than 4 cumulants would be
    needed), we just add more paths to the distribution function.

    \vmm \hrule \vmm

    And now \vmm
    
    \fcolorbox{black}{lightyellow2}{\begin{minipage}{103mm}
        \begin{eqnarray*}
          \mathbf{\bchi_j(k)} = & 
          \mathbf{\displaystyle{ \hspace{-2mm}
              \frac{\feffc{f(k)}\times \feffc{N} \times {\pthpar{S02}}}
              {k( \reff + \pthpar{delR} ) ^2}
            }
            \enskip\bexp(-2p''\reff - 2p^2\pthpar{sigma2} + \twothirds p^4 \pthpar{fourth}) } \\
          &  
          \mathbf{\times \enskip\bexp \biggl\{ i\big[
            2k\reff + \feffc{\delta(k)}
            + 2p(\pthpar{delR} - 2{\textstyle{\pthpar{sigma2}\over\reff }})
            -\fourthirds p^3\pthpar{third}
            \big]  \biggr\}     } 
        \end{eqnarray*}
      \end{minipage} 

    }
    
    \vmm \vmm
    
    doesn't seem that bad, huh? 

\end{cenpage}
\vfill
\end{slide} 


