%% Slide
\begin{slide}{The EXAFS Equation: What we left out}

    \vmm
\begin{cenpage}{110mm}

  For {\BlueEmph{quantitative}} EXAFS calculations, it's important to
  also consider:

    \begin{description}
    \item[\RedEmph{Photo-electron mean free path}]  Accounts for inelastic
      scattering of the photo-electron and finite lifetime of the core-hole.

    \item[\RedEmph{Fast Relaxation of the Excited State}]   The other
      electrons in the excited atom can respond to the core-hole.

    \item[\RedEmph{Multiple Scattering}] the photo-electron can scatter
      from multiple atoms.  Most important at low $k$, and leads to a
      {\BlueEmph{Path Expansion}}.

    \item[\RedEmph{Polarization Effects}] synchrotron beams are highly
      polarized, which needs to be taken into account.  Simple for
      $K$-edges ($s\rightarrow p$ is dipole). Less simple for $L$-edges.

    \item[\RedEmph{Disorder Terms}]  thermal and static disorder in real
      systems should be properly considered: A topic of its own.

    \item[\RedEmph{Muffin-Tin Approximation:}] The scattering calculation needs
      a real-space potential, and a muffin-tin approximation is most
      tractable.  But this approximation is not so good for XANES.

    \end{description}

    {\BlueEmph{Good News}}:   The {\feff} calculations (or others) try to include these effects.


    \vmm \vmm
    We'll discuss of few of these in more detail \ldots.

\end{cenpage}


\vfill

\end{slide}
