
\begin{slide}{EXAFS: Extended X-ray Absorption Fine Structure}


\begin{cenpage}{108mm}
  Even far above the edge, there are oscillations in $\mu(E)$ that are
  sensitive to the positions and types of atoms neighboring the absorbing
  atom.
\vmm

 We define the EXAFS as:

  \[
  \mu(E) =   \mu_0(E) [1 + \chi(E)]  \hspace{15mm} \chi(E) =   \frac{ {\mu(E) - \mu_0(E)}}{\Delta \mu_0(E_0)}
  \]

  We subtract off a smooth {\BlueEmph{``bare atom'' background}}
  $\mu_0(E)$, and divide by the {\BlueEmph{``edge step''}}
  $\Delta \mu_0(E_0)$ to get $\chi$, the EXAFS oscillations:

\end{cenpage}

\begin{tabular}{ll}
  \onslide+<1->
  \begin{minipage}{55mm}
    \rgraph{55mm}{mu_with_mu0}
  \end{minipage}
  &
  \onslide+<1->
  \begin{minipage}{55mm}
    \rgraph{55mm}{chie}
  \end{minipage} \\
%%   \noalign{\smallskip}\\
  \onslide+<1->
  \hspace{3mm} $\mu(E)$ and $\mu_0(E)$ for FeO
  &
  \onslide+<1->
  \hspace{3mm} $\chi(E)$ for FeO, with $E_0$ = 7122 eV.
\end{tabular}

\vfill
\end{slide}
