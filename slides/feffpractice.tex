\section{Practical FEFF}

\begin{slide}{ Practical {\feff} }

    \begin{cenpage}{110mm}

{\RedEmph{Good News}}: you don't have to worry about the hard parts (mostly)!

\begin{enumerate}
\item Start with a structure close to the {\BlueEmph{local}} atomic structure
  of your sample, and generate x,y,z coordinates for the atoms. Often a
  crystal structure is close -- it does not have to be perfect!

\item Run {\feff}.  This creates a {\feffndat} files for each path.

\item Use these  Path Files in to model measured XAFS.
\end{enumerate}

  {\larch} and {\xasviewer} helps you find crystal structures, create and
  edit input filess, run {\feff}, and sort and use the results. 

\begin{columns}
      \begin{column}[T]{88mm}
        \begin{block}{ run {\feff} in a specified folder  }
          \begin{semiverbatim} {\footnotesize larch>
{\LFunc{feff8l}}({\LString{'feff8.inp'}}, folder={\LString{'CuS\_Feff'}}) }
          \end{semiverbatim}
        \end{block}

      \end{column}
      \begin{column}[T]{1mm} \end{column}
    \end{columns}


\end{cenpage}

  Having good starting structures can be important, but you do not need the
  exact structure to use {\feff}.

\vmm


  You can (may need to) mix structural models to model real data.

\end{slide}{

\subsection{Anatomy of {\file{feff.inp}}}
\begin{frame}[fragile] \frametitle{Anatomy of {\file{feff.inp}}}

\begin{cenpage}{100mm}
  {\feff} is a very old program that runs from an input file -- it
  {\bf{must}} be called {\file{feff.inp}}.   {\larch} can help with this.
  \vmm

  Each calculation should be in its own folder/subdirectory.
\end{cenpage}


\begin{tabular}{lr}
  \begin{CodeBlock}{48mm}{\file{feff.inp} file ({\feff} 8):}
{\Blue{TITLE    FeO, rock salt structure}}
{\Blue{EDGE K}}
{\Blue{S02   1.0}}
{\Blue{CONTROL  1 1 1 1 1 1}}  {\Red{\# which parts of code to run}}
{\Blue{PRINT    1 0 0 0 0 3}}  {\Red{\# which output files to write}}
{\Blue{RPATH   6.0}}   {\Red{\# How far in R to build paths}}

{\Blue{POTENTIALS }}      {\Red{\# list of Atomic Potentials}}
* potential  z  label
      0     26  Fe   {\Red{\# Absorbing Atom}}
      1      8  O    {\Red{\# 1 Potential for each Z}}
      2     26  Fe

{\Blue{ATOMS}}  {\Red{\# list of Atomic X, Y, Z, Potential }}
 0.00000     0.00000     0.00000    0   Fe
 0.00000     0.00000    -2.13870    1   O
-2.13870     0.00000     0.00000    1   O
 0.00000    -2.13870     0.00000    1   O
 0.00000     0.00000     2.13870    1   O
 2.13870     0.00000     0.00000    1   O
 0.00000     2.13870     0.00000    1   O
 0.00000     2.13870     2.13870    2   Fe
 0.00000    -2.13870    -2.13870    2   Fe
\end{CodeBlock} &
\begin{minipage}{67mm}

\onslide<2->

{\file{feff.inp}} includes:

\vmm

\begin{enumerate}
  \onslide<2->\item A list of unique Atomic Potentials:
  \begin{itemize}
  \item 1 Absorbing Atom, always Potential 0 
  \item 1 Potential per atomic species (Z)
  \end{itemize}

  \onslide<2->\item List of atomic coordinates (in {\AA}):

  {\hmm\hmm}  $x$, $y$, $z$, $Pot$

  for each atom in the cluster of atoms 

  \vmm
  The cluster can be non-crystalline.

  \vmm
  The absorber can be at (0, 0, 0), or not.

  \vmm
  Hint:  better off removing H atoms!

\end{enumerate}
\end{minipage}\\
\end{tabular}

  \onslide<2->
   \begin{postitbox}{90mm}
     \Red{You can edit the atomic positions, and add or change potentials.}
   \end{postitbox}

\end{frame}


\begin{slide}{Input Parameters for {\file{feff.inp}  ({\feff} 8)}}

\begin{cenpage}{100mm}
  {\feff} has many  inputs, but only a few of them are really
  important for EXAFS Analysis (some are {\Red{\texttt{required}}}, some {\Blue{\texttt{optional}}}):

  \begin{description}[POTENTIALSXX]
  \item[{\Red{\texttt{EDGE}}}]   which edge  absorbs  the x-ray ({\tt{K}}, {\tt{L3}}, etc.)

  \item[{\Red{\texttt{POTENTIALS}}}]   list of atomic potentials (0 = absorbing atom)

  \item[{\Red{\texttt{ATOMS}}}]   list of atomic $x$, $y$, $z$, Potential

  \item[{\Red{\texttt{CONTROL}}}]   which ``Modules'' to run.  Use  ``1 1 1 1 1 1''.

  \item[{\Red{\texttt{PRINT}}}]   which ``Outputs'' to write. Use  ``1 0 0  0 0 3''.


  \item[{\Blue{\texttt{RPATH}}}]   how far out (in {\AA}) to consider the  cluster of atoms.

  \item[{\Blue{\texttt{POLARIZATION}}}]   polarization vector of incident
    x-ray (in same coordinate system as atomic coordinates)

  \item[{\Blue{\texttt{EXCHANGE}}}]   which model to use for the exchange energy.
    Use the default (Hedin-Lundqvist model) unless you know why.

  \end{description}

\end{cenpage}

\end{slide}



\subsection{ Converting CIF data into {\feff} inputs}
\begin{frame}[fragile] \frametitle{Converting CIF data into {\feff} inputs}

\begin{cenpage}{95mm}

  {\larch} comes with about 9000 crystral structures from the 

\end{cenpage}

  \begin{cenpage}{75mm}
    American Mineralogist Crystal Structure Database 
  \end{cenpage}

\begin{cenpage}{95mm}


  and can convert these as {\RedEmph{Crystallographic Information Format}}
  (CIF) files to {\feff} input files.

  \vmm
  It handles fractional site occupancy with weighted substitutions -- use
  with caution!

  \vmm
  It can also read some CIF files from other sources (work-in-progress!).

  \vmm
  This is a convenient way to generate a {\feff} input, but not necessary!

\begin{postitbox}{84mm}
  The cluster used by {\feff} does not need to be a crystal!
\end{postitbox}

\vmm
Atomic clusters may also come from:

\vmm
Protein Data Bank, Molecular Dynamics Simulations, 
Density Functional Theory Calculations, \ldots.


\end{cenpage}
\end{frame}


\begin{slide}{{\feff}:  what {\em{should}} you worry about?}

  Using {\feff} in a real analysis, we'll be {\BlueEmph{refining}}
  bond lengths and coordination numbers.  For this to work, here are some
  ``rules of thumb'':

  \begin{cenpage}{90mm}

\begin{enumerate}
   \item refining distances by {\bf{more than $0.1\rm\,\AA$}} probably means
  the calculation should be re-run -- the overlap of atomic potentials may
  not be accurate.

   \item  refining energy origins $E_0$ by {\bf{more than $10\rm\,eV$}} may
  mean the calculation -- or the selection of $E_0$ in the data reduction
  -- is wrong or needs to be re-examined.

 \item  {\feff} can set $S_0^2$ and make a simple attempt at
  $\sigma^2$ values.  {\bf{don't use these}}.

 \item You {\bf{should set $S_0^2$}} for a particular group
  of data (central atom, beamline energy resolution, etc), based on an
  experimental standard, or have uncertain values for $N$.
  \end{enumerate}
  \end{cenpage}

\end{slide}
