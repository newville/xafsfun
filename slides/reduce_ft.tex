\begin{slide}{EXAFS Fourier Transform: ${\chi(R)}$}

  \begin{tabular}{ll}
    \begin{minipage}{65mm}  {\wgraph{65mm}{reduction/chir}}  \end{minipage}
    &
    \begin{minipage}{42mm}  \setlength{\baselineskip}{10pt}
      \vspace{1mm} 
      {\Red{${\chi(R)}$}}\vspace{0.5mm}      
      
      The Fourier Transform of ${k^2\chi(k)}$ has 2 main peaks for
      the  Fe-O and Fe-Fe shells.  \vspace{1mm}
      
      
      The Fe-O distance in ${\rm FeO}$ is 2.14{\AA}, but the first
      peak is at 1.6{\AA}.  This shift in the first peak is due to the
      {\RedEmph{phase-shift}}, ${{\bdelta(k)}}$: ${
      {\bsin[{2kR+ {\Red{\bdelta(k)}}}]} }$ .
      

    \end{minipage}\\
    \begin{minipage}{65mm} 
      \onslide+<2->
      {\wgraph{65mm}{reduction/chir_complex}}
    \end{minipage}
    &
    \begin{minipage}{42mm} \setlength{\baselineskip}{10pt}
        \vspace{-0.1mm}
        A shift of -0.5{\AA} is typical.
        
        \vspace{1mm}       
      \onslide+<2->{    \hrule        
        \vspace{2mm}
        
        {\Red{${\chi(R)}$ is complex:}}\vspace{0.5mm}      
        
        The FT makes ${\chi(R)}$ complex.  Usually only the amplitude
        is shown.

        \vmm

      In data modeling, both real and imaginary components are used.

      \vspace{3mm}}
    \end{minipage}
  \end{tabular}

\vfill
\end{slide} 
