\begin{slide}{Data Reduction: Post-Edge Background Subtraction}

  \begin{tabular}{ll}
    \begin{minipage}{65mm}  {\wgraph{65mm}{reduction/bkg}}  \end{minipage}
    &
    \begin{minipage}{37mm}  \setlength{\baselineskip}{10pt}
      {\Red{Post-Edge Background}}\vspace{0.5mm}
      
      We can't measure ${\mu_0(E)}$ (the absorption coefficient without
      neighboring atoms).
      
      \vspace{1mm} 

      We approximate ${\mu_0(E)}$ by an adjustable, smooth
      function: a {\BlueEmph{spline}}.
      
      \vspace{1mm} 

    \end{minipage}\\
    \begin{minipage}{65mm} 
            \onslide+<2->{\wgraph{65mm}{reduction/bkg_xanes}} \end{minipage}
    &
    \begin{minipage}{37mm} \setlength{\baselineskip}{10pt}
      \onslide+<2->{

      This can be somewhat dangerous -- a flexible enough
      spline could match the $\mu(E)$ and remove all the EXAFS!

      \vspace{1mm} 
      We want a spline that will match the {\BlueEmph{low frequency}}
      components of ${\mu_0(E)}$.
      }
    \end{minipage}
  \end{tabular}

\vfill
\end{slide} 
