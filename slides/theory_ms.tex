
%% Slide
\begin{slide}{Sum over Paths and Multiple Scattering}

  A sum over scattering paths allows {\RedEmph{multiple-scattering paths}}:
  the photo-electron scatters from {\BlueEmph{more than one atom}} before
  returning to the central atom:

    \vmm

    \begin{tabular}{ll}
      \begin{minipage}{57mm}
        \onslide+<2->
        \scalebox{1}{\rgraph{58mm}{mspaths}}
      \end{minipage}
      &
      \begin{minipage}{55mm}\setlength{\baselineskip}{10pt}
        \onslide+<2->
        For multi-bounce paths, the total amplitude depends on the
        {\Red{angles}} in the photo-electron path.

        \onslide+<3->
        \vspace{2mm} {\Blue{Triangle Paths}} with angles $ 45 < \theta <
        135^{\circ}$ aren't strong, but there can be a lot of them.

        \onslide+<4->
        \vspace{2mm}
        {\Blue{Linear paths}}, with angles $\theta \approx 180^{\circ}$,
        are very strong: the photo-electron can be {\Red{focused}} through
        one atom to the next.

        % \vspace{2mm} {\feff} calculates these effects and includes them
        % in $\Red{f(k)}$ and $\Red{\delta(k)}$ for the EXAFS equation so
        % that all paths look the same in the analysis.

      \end{minipage}
    \end{tabular}

    \onslide+<5->{
      \begin{center}
        \begin{postitbox}{80mm}
            Multiple Scattering is strongest when   $ \theta > 150^{\circ}$.
        \end{postitbox}
      \end{center}

      The strong angular dependence can be used  to measure bond angles.


\vmm   For first shell analysis, multiple scattering is hardly ever needed.
}
\end{slide}
