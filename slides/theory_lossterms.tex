%% Slide
\begin{slide}{What we left out}
  \vmm
    
    This simple description is qualitatively right, but for quantitative
    EXAFS calculations, it's important to include a few additional points:
    
    \begin{description}
    \item[\RedEmph{Curved Wave Effects}] proper scattering calculation
      including partial waves.
      
    \item[\RedEmph{Extrinsic Losses}] Photo-electron mean-free path,
      including self-energy and finite core-hole lifetime.
      
    \item[\RedEmph{Intrinsic Losses}] losses due to relaxation of excited
      atom to the sudden presence of the core hole.
      
    \item[\RedEmph{Multiple Scattering}] 

    \item[RedEmph{Polarization Effects}] 


    \item[RedEmph{Disorder Terms}]  thermal and static disorder should be
    properly accounted for
      
    \end{description}

    we need to take many-body exchange
    and loss terms into account.


    To account for many-body effects



    Another important {\Red{Amplitude Reduction Term}} is due to the relaxation
    of all the other electrons in the absorbing atom to the hole in the
    core level:

    \vspace{3mm}

    { \begin{center}
     $
     {\displaystyle
       S_0^2 =  {  |{\langle \Phi^{N-1}_f |\Phi^{N-1}_0 \rangle}|^2 }  }     $
     \end{center}
     }
     \vspace{1mm}
     
     where ${\langle \Phi^{N-1}_f|}$ accounts for the relaxation of
     the other ($N-1$) electrons relative to these electrons in the
     unexcited atom: ${| \Phi^{N-1}_0 \rangle }$.
     Typically, $\Red{S_0^2}$ is taken as a constant:

     \vspace{2mm}

     \begin{center}
       \fcolorbox{black}{lightyellow2}{
         \begin{minipage}{25mm}
           \begin{center}
           $ 0.7 < S_0^2 < 1.0 $ \end{center}
         \end{minipage}}
     \end{center}
    
     which is found for a given central atom, and simply multiplies the
     XAFS {$\chi$}.
     
     \vspace{3mm} 

     \begin{center}
       {\Blue{Note that $\Red{S_0^2}$ is Completely Correlated with $N$
           (!!!)}}
     \end{center}

     \vspace{3mm}      

     \begin{center}
       \fcolorbox{black}{lightyellow2}{
         \begin{minipage}{68mm}
           
           This, and other experimental and theoretical issues, make EXAFS
           amplitudes (and therefore {\Blue{$N$}}) less precise than EXAFS
           phases (and therefore {\Blue{$R$}}).

         \end{minipage}}
     \end{center}
     
     \vspace{3mm}


     Usually $\Red{S_0^2}$ is found from a ``standard'' (data from a sample
     with well-known structure) and applied to a set of unknowns as a scale
     factor.     

     
     \vspace{2mm}

\vfill
\end{slide} 
