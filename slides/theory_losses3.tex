%% Slide
\begin{slide}{$S_0^2$:  Amplitude Reduction Term}
  
\vmm
    
    {\RedEmph{Intrinsic Losses}}, due to the relaxation of all the other
    electrons in the absorbing atom to the hole in the core level, can be
    described as an {\Red{Amplitude Reduction Term}}:

    \vmm

    \[
    S_0^2 =  {  |{\langle \Phi^{N-1}_f |\Phi^{N-1}_0 \rangle}|^2}
    \]

     \vspace{1mm}
     
     where ${\langle \Phi^{N-1}_f|}$ accounts for the relaxation of the
     other ($N-1$) electrons relative to these electrons in the unexcited
     atom, ${| \Phi^{N-1}_0 \rangle }$
     

     \vmm

     ${S_0^2}$ is most often taken as a constant: $ 0.7 < S_0^2 < 1.0 $.
    
 
     and is used as a fitting parameter that multiplies {$\chi$}.
     
     \vmm

     \begin{center}
       {\Red{Note that ${S_0^2}$ is Completely Correlated with $N$      (!!!)}}
     \end{center}

     \vmm

     \begin{center}
       \highlightbox{
         \begin{minipage}{75mm}
           $ S_0^2 $  and experimental issues make EXAFS amplitudes
           ({\Blue{$N$}}) less precise than EXAFS phases ({\Blue{$R$}}).
         \end{minipage}}
     \end{center}
     

     \vmm

     In principle, $S_0^2$  includes {\RedEmph{multi-electron excitations}},
     which are often ignored for EXAFS.
     
     \vmm
     
     For improved calculations of $S_0^2(E)$, see Campbell, et al, PRB
     {\bf{65}} 064107.


     \vmm

\vfill
\end{slide} 
