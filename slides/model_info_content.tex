\begin{slide}{The Information Content of EXAFS}
    
    The number of parameters we can reliably measure from our data is
    limited:
    \vspace{1mm} 

    \begin{center} 
      $\displaystyle  N \approx { { 2 \Delta k \Delta R} \over{ \pi}}  $
    \end{center}
    
    \vspace{1mm} 
    
    where $\Red{ \Delta k}$ and $\Red{ \Delta R}$ are the $k$- and
    $R$-ranges of the usable data.  

    \onslide+<2->
    
    For a typical range of $k = [3.0, 12.5] \rm\,\AA^{-1}$ and $R = [1.0,
    3.0] \rm\,\AA$, there are $\sim 12$ parameters that can be determined
    from EXAFS.

    \onslide+<3->
    \vspace{2mm}

    The ``Goodness of Fit'' statistics, and confidence in the measured
    parameters need to reflect this limited amount of data.

    \vspace{3mm}
    
    \onslide+<3->
    It's often necessary to {\Red{constrain}} parameters $R$, $N$,
    $\sigma^2$ for different paths or even different data sets (different
    edge elements, temperatures, etc)

    \vspace{2mm}
    
    {\Red{Chemical Plausibility}} can also be incorporated, either to weed
    out obviously bad results or to use other knowledge of local
    coordination, such as the Bond Valence Model (relating valence,
    distance, and coordination number).
    
    \vspace{3mm}
 
    \begin{center}
      {\fcolorbox{black}{lightyellow}{
          \begin{minipage}{90mm}
            Use as much other information about the system as possible!
          \end{minipage}
        }}
      \end{center}
          
\vfill
\end{slide} 
