
\begin{slide}{Data Reduction: $\chi(k)$, $k$-weighting}
  \small \setlength{\baselineskip}{10pt}
  \begin{center}
  \begin{tabular}{ll}
    \begin{minipage}{65mm}  {\wgraph{65mm}{reduction/chi}}  \end{minipage}
    &
    \begin{minipage}{37mm}  \setlength{\baselineskip}{10pt}
      {\Red{${\chi(k)}$}}\vspace{0.5mm}
      
      The raw EXAFS ${\chi(k)}$ usually decays quickly with
      ${k}$, and difficult to assess or interpret by itself.

      \vspace{1mm} 

      It is customary to weight the higher-${k}$ portion of the spectra by
      multiplying by  ${k^2}$ or ${k^3}$.
      \vfill
    \end{minipage}\\
    \begin{minipage}{65mm} 
      \onslide+<2->
      {\wgraph{65mm}{reduction/chik}}   \end{minipage}
    &
    \begin{minipage}{37mm} \setlength{\baselineskip}{10pt}
      \onslide+<2->{
      {\Red{${k}$-weighted ${\bchi(k)}$:
      ${k^2\bchi(k)}$}}\vspace{0.5mm}
      
      ${\bchi(k)}$ is composed of sine waves, so we'll Fourier
      Transform from ${k}$ to ${R}$-space.  To avoid
      ``ringing'', we'll multiply by a {\BlueEmph{window function}}.
      \vfill}
    \end{minipage}
  \end{tabular}
  \end{center}    
\vfill
\end{slide} 
