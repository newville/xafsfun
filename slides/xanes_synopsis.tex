
\begin{slide}{XANES: Summary}
  \begin{cenpage}{110mm}
    \begin{itemize}
    \item  {\RedEmph{XANES is a much larger signal than  EXAFS}}
      \vspace{2mm}

      XANES can be done at lower concentrations, and
      less-than-perfect sample conditions (small beam/sample size, complex
      sample environments).

      \vmm

    \item {\RedEmph{XANES is harder to fully interpret than EXAFS}}
      \vspace{2mm}

      The exact physical and chemical interpretation of all spectral
      features is still difficult to do accurately, precisely, and
      reliably.

      \vmm

    \item {\RedEmph{XANES lends itself to linear spectroscopic methods}}
      \vspace{2mm}

      XANES is not highly affected by thermal disorder, so maps well to
      ``coordination geometry''.
      \vmm

      For many systems, the XANES analysis based on linear combinations of
      other measured spectra -- either from ``model compounds'' or from a
      large series of measured spectra -- is informative and valuable.


      \vmm These {\emph{linear analysis}} methods can be very easy to apply
      and can make use of more sophisticated statistical methods used in
      other spectroscopies.

    \end{itemize}
  \end{cenpage}
\end{slide}
