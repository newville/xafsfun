
\begin{slide}{Fluorescence Measurements: Filters}

  \vmm

  \begin{tabular}{ll}
    \begin{minipage}{35mm}\setlength{\baselineskip}{10pt}
      A typical fluorescence setup with a 'Z-1' filter uses a simple ion
      chamber which has no energy resolution, but high count rate and
      linearity.

      \vspace{4mm}

      Because the filter absorbs the scattered beam, it can itself
      re-radiate!!

    \end{minipage}
    &
    \begin{minipage}{65mm}
      \scalebox{1}{\wgraph{58mm}{experiment/soller}}
    \end{minipage}
  \end{tabular}
  \vspace{3mm} 

  A set of {\RedEmph{Soller slits}} can be used to see the sample, but
  absorb most of the re-radiate scatter from the filter.

\vspace{2mm}

This arrangement can be very effective especially when the signal is
dominated by {\BlueEmph{scatter}}, and when the concentration is at per
cent levels.

\vfill
\end{slide} 
