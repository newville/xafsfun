\section{Intermediate  FEFF}
%% Slide
\subsection{FEFF.DAT Files}
\begin{frame}[fragile] 
  \frametitle{Intermediate FEFF: Anatomy of a {\feffndat} file}

    \vmm
    \begin{tabular}{lr}
      \hspace{-5mm} \begin{minipage}{20mm} {\small
          
          \vmm \vmm 

          Header  \vmm

          {\tt{---------}}


          $N_{\rm leg}$, {\feffc{N}},  {\reff }

          {\Blue{Path Geometry}}
          
          \vmm \vmm \vmm
          
          columns of $k$-dependent data
          
        }
      \end{minipage}  
      & \begin{minipage}{84mm}
    \begin{block}{FEFF OUTPUT FILE}

      {\tiny{  \begin{semiverbatim}
 Cu  metal  fcc a=3.603                                           Feff XX 6.10
 Abs   Z=29 Rmt= 1.289 Rnm= 1.424 K shell
 Pot 1 Z=29 Rmt= 1.272 Rnm= 1.392
 Gam_ch=1.761E+00 H-L exch
 Mu=-5.535E+00 kf=1.806E+00 Vint=-1.796E+01 Rs_int= 2.008
 Path    1      icalc       2
 -----------------------------------------------------------------------
{\Red{   2  12.000   2.5478 }}   2.6300   -5.53502 {\Red{nleg, deg, reff}}, rnrmav(bohr), edge
{\Blue{        x         y         z   pot at#
      .0000     .0000     .0000  0  29 Cu       absorbing atom
      .0000   -1.8016    1.8016  1  29 Cu}}
{\Red{    k   real[2*phc]   mag[feff]  phase[feff] red factor   lambda     real[p]@#}}
   .000  3.5015E+00  0.0000E+00 -5.1320E+00  9.872E-01  1.5637E+01  1.8071E+00
   .100  3.5020E+00  3.3976E-02 -5.6318E+00  9.873E-01  1.5660E+01  1.8098E+00
   .200  3.5035E+00  6.7038E-02 -6.1121E+00  9.877E-01  1.5726E+01  1.8177E+00
   .300  3.5060E+00  9.8421E-02 -6.5731E+00  9.882E-01  1.5831E+01  1.8309E+00
   .400  3.5093E+00  1.2760E-01 -7.0153E+00  9.890E-01  1.5963E+01  1.8491E+00
   .500  3.5132E+00  1.5432E-01 -7.4387E+00  9.901E-01  1.6111E+01  1.8724E+00
    \end{semiverbatim}
  }}\end{block}
  \end{minipage}   \\
\end{tabular}  
{}
    \vmm \hrule\vmm

    Conversion from {\feffndat} to the EXAFS Equation:


    \begin{tabular}{lll}
      \feffc{N} : \Red{\bf{\tt{deg}}}  &  \hspace{9mm} \reff     : \Red{\bf{\tt{reff}}} &
      \feffc{k_{\rm FEFF}} : \Red{\bf{\tt{k}}}  \\
      \multicolumn{2}{l}{
      \feffc{f(k)} : \Red{\bf{\tt{mag[feff] * red factor}}}} \hspace{5mm} &       
      \feffc{\lambda(k)} : \Red{\bf{\tt{lambda}}}   \\
      \multicolumn{2}{l}{
      \feffc{\delta(k)} : \Red{\bf{\tt{real[2*phc]+phase[feff]}}} }& 
            \feffc{real\_p(k)} : \Red{\bf{\tt{real[p]}}}  \\
    \end{tabular}

\end{frame} 


\begin{slide}{Scattering Amplitude and Phase-Shift:
    ${f(k)}$ and ${\delta(k)}$ }

    
    The scattering amplitude ${\Red{f(k)}}$ and phase-shift
    ${{\Red{\delta(k)}}}$ depend on atomic number.

    \vspace{1mm}

    \begin{tabular}{ll}
      \begin{minipage}{55mm}
        \scalebox{1}{\wgraph{55mm}{theory/scatt_amp}}
      \end{minipage}
      & 
      \begin{minipage}{45mm}\setlength{\baselineskip}{10pt}
        The scattering amplitude ${{\Red{f(k)}}}$ peaks at
        different ${k}$ values and extends to
        higher-${k}$ for heavier elements.  For very heavy
        elements, there is structure in ${{\Red{f(k)}}}$.
        
      \end{minipage}
      \\
      \begin{minipage}{55mm}
        \scalebox{1}{\wgraph{55mm}{theory/scatt_pha}}
      \end{minipage}
      & 
      \begin{minipage}{45mm}\setlength{\baselineskip}{10pt}
        The phase shift ${{\Red{\delta(k)}}}$ shows sharp
        changes for very heavy elements.
        
        \vspace{2mm}
        
        \fcolorbox{black}{lightyellow2}{
          \begin{minipage}{42mm}
            These scattering functions can be accurately calculated (say with
            the program {\feff}, and used in the EXAFS modeling.
          \end{minipage}}
        
        
      \end{minipage}
    \end{tabular}
  
  \begin{center}
    \fcolorbox{black}{lightyellow}{
      \begin{minipage}{68mm}
        ${Z}$ can usually be determined to with 5 or so.  Fe and O can be
        distinguished, but Fe and Mn cannot be.
      \end{minipage}}
  \end{center}
  
  \vfill
\end{slide} 


%% Slide
\begin{slide}{FEFF and The EXAFS Equation: What's so hard???}
  
  \vmm
    
    {\feff} includes sophisticated techniqies to calculate of {\feffc{f(k)}}
    and \feffc{\delta(k)}:

    \begin{description}
    \item[\RedEmph{Curved Wave Effects}] the photo-electron goes out as
      spherical wave and scatters from an atom with finite size: A partial
      wave expansion is needed.

    \item[\RedEmph{Muffin-Tin Approximation:}] scattering calculation needs
      a real-space potential, and a muffin-tin approximation is most
      tractable, if approximate.

    \item[\RedEmph{Extrinsic Losses}] photo-electron mean-free path,
      including self-energy and finite core-hole lifetime.
      
    \item[\RedEmph{Intrinsic Losses}] relaxation of absorbing atom to the
      presence of the core hole.
      
    \item[\RedEmph{Multiple Scattering}] the photo-electron can scatter
      multiple times, which is important at low $k$, and can be important
      at high $k$ for some systems.  Luckily, all paths from {\feff} appear
      the same.
      
    \item[\RedEmph{Polarization Effects}] synchrotron beams are highly
      polarized, which needs to be taken into account.  This is simple for
      $K$-edges ($s\rightarrow p$ is dipole), and less so for $L$-edges
      (where both $p \rightarrow d$ and $p \rightarrow s$ contribute).

    \end{description}

\vfill
\end{slide} 


%% Slide
\begin{slide}{Intermediate FEFF: Extrinsic Loss Terms}
  
  \vmm
    The photo-electrons are quasiparticles, and can inelastically scatter.
    This leads to ``Extrinsic Losses'':

    \begin{description}
    \item[{\RedEmph{self-energy}}] $\Sigma$.  The potential for a one-particle Dyson
      equation becomes 
      
      \[ {\cal{H}}\psi = \Bigl[ -{\frac{1}{2}}\bigtriangledown^2 + V_{\rm
        coul} + V_{\rm core-hole} + \Sigma(E) \Bigr] \psi = E\psi
      \]

      
      The self-energy for the {\RedEmph{excited atom}}, goes beyond
      ``normal'' ground state DFT, and follows work of L. Hedin and S.
      Lundqvist J. Phys C {\bf{4}}, 2064 (1971).

      
    \item[{\RedEmph{core-hole width}}] $\Gamma$.  
      
      This ranges from ~0.1 to 10 eV for most hard x-ray core-levels, which
      corresponds to a {\RedEmph{core-hole lifetime}} $\tau$ on the order
      of 1 fs (fast!)  ($\Gamma \tau \approx \hbar$).

    \end{description}

    \vmm \hrule \vmm
    
    Conceptually, these two terms can be combined into a {\BlueEmph{mean
        free path}}, ${\lambda(k)}$, and the photo-electron wave
    function be changed from
    
    \[ \displaystyle{{
        \psi(k,r)  =  {\frac{e^{ikr}}{kr}} \Rightarrow   {\frac{e^{ikr}e^{-r/\lambda(k)}}{kr}}
        \Rightarrow {\frac{e^{ipr}}{pr}} }}
    \]

    which is why we use the complex wavenumber of the photo-electron:
    \[{p(k) = k + \Sigma + i / \lambda(k) } \]

\vfill
\end{slide} 

%% Slide
\begin{slide}{${p(k)}$: the complex momentum of the photo-electron }

    \[{p(k) = k + \Sigma + i / \lambda(k) } \]

    ${E = p(k)^2 \hbar^2 / 2m_e }$ is the ``kinetic energy'' of the
    photo-electron.

    \vmm 

    \begin{center} 
      \scalebox{1}{\wgraph{75mm}{theory/pvk}}
    \end{center}
      
    \vmm \vmm 
    
    Due to the finite, positive self-energy (${\Sigma(E)}$), and
    that ${p(k)}$ is referenced to the continuum level
    (${E_0}$) and ${k}$ is referenced to the Fermi level
    (${E_f}$):

    \[{Re[p(k)] \gtrsim k}\]

      
    \vmm \vmm
    

  \vfill
\end{slide} 

% %% Slide
% \begin{slide} \STitle{${\lambda(k)}$: The Photo-Electron Mean-Free Path}

%     
%     \begin{center} 
%       \scalebox{1}{\wgraph{75mm}{theory/scatt_lam}}
%     \end{center}
%       
%     \vmm \vmm 
% 
%     \begin{center}
%       \highlightbox{ \begin{minipage}{75mm}
%         ${\rm 5 \AA < \lambda < 25\,\AA}$ for the EXAFS ${k}$-range.
%         
%         \vspace{2mm} 
%         
%         The ${\lambda}$ and ${R^{-2}}$ terms make EXAFS a
%         {\RedEmph{local probe}}.
% 
%         \vspace{2mm} 
%         
%         XANES ( $k<3\rm\,\AA^{-1}$ ) is more sensitive to larger
%         distances.
%         
%       \end{minipage} }
%     \end{center}

% \vfill
% \end{slide} 
% 
% %% Slide
% \begin{slide}$S_0^2$:  Amplitude Reduction Term}
%     
%     {\RedEmph{Intrinsic Losses}}, due to the relaxation of all the other
%     electrons in the absorbing atom to the hole in the core level, can be
%     described as an {\Red{Amplitude Reduction Term}}:
% 
%     \vmm
% 
%     \[
%     S_0^2 =  {  |{\langle \Phi^{N-1}_f |\Phi^{N-1}_0 \rangle}|^2}
%     \]
% 
%      \vspace{1mm}
%      
%      where ${\langle \Phi^{N-1}_f|}$ accounts for the relaxation of the
%      other ($N-1$) electrons relative to these electrons in the unexcited
%      atom, ${| \Phi^{N-1}_0 \rangle }$
%      
% 
%      \vmm
% 
%      ${S_0^2}$ is most often taken as a constant: $ 0.7 < S_0^2 < 1.0 $.
%     
%  
%      and is used as a fitting parameter that multiplies {$\chi$}.
%      
%      \vmm
% 
%      \begin{center}
%        {\Red{Note that ${S_0^2}$ is Completely Correlated with $N$      (!!!)}}
%      \end{center}
% 
%      \vmm
% 
%      \begin{center}
%        \highlightbox{
%          \begin{minipage}{75mm}
%            $ S_0^2 $  and experimental issues make EXAFS amplitudes
%            ({\Blue{$N$}}) less precise than EXAFS phases ({\Blue{$R$}}).
%          \end{minipage}}
%      \end{center}
%      
% 
%      \vmm
% 
%      In principle, $S_0^2$  includes {\RedEmph{multi-electron excitations}},
%      which are often ignored for EXAFS.
%      
%      \vspace{1mm}
%      
%      For improved calculations of $S_0^2(E)$, see Campbell, et al, PRB
%      {\bf{65}} 064107.
% 
% 
%      \vmm
%      
%      
% \vfill
% \end{slide} 
% 
% 
% 
% %% Slide
% \begin{slide} \STitle{Multiple Scattering}
% 

%     \vmm
%     
%     The photo-electron can scatter multiple times:
% 
%     \vmm
% 
%     \begin{center}\scalebox{1}{\wgraph{85mm}{theory/mspaths2}}\end{center}
%     
%     
%     A {\Blue{Path Formalism}} based on Green's Functions is used in modern
%     EXAFS, for `''Real Space'' calculations on arbitrary clusters of atoms:
%     
%     \[ G = G^0 + G^0 t G^0 + G^0 t G^0 t G^0 + 
%     G^0 t G^0 t G^0 t G^0 + \ldots \]
%     
%     where ${\Blue{G^0}}$ describes the propagation of the electron, and
%     $\Blue{t}$ describes the scattering from the potential of the
%     neighboring atom.
% 
% 
%    \vmm
%        
%     \vmm {\Blue{Triangle Paths}} with angles $ 45 < \theta <
%     135^{\circ}$ are weak, but there are many of them.
%     
%     \vmm {\Blue{Linear paths}} with angles $\theta \approx 180^{\circ}$ 
%     are very strong: the photo-electron can be {\Red{focused}} through
%     one atom to the next.
%     This strong angular dependence  can be used to measure bond angles.
% 
%     
% 
%     \vmm  
% 
%     \highlightbox{\Red{{\feffndat} looks the same for Single- and Multiple-Scattering Paths!}}
% 
% 
%     \vfill
% 
% \end{slide} 
% 
