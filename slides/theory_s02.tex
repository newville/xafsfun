%% Slide
\begin{slide}{$S_0^2$:  Amplitude Reduction Term (intrinsic losses)}

  \vspace{1mm}
  \begin{cenpage} {125mm}

    Another important correction that we left out so far:\vmm

    The {\Red{Amplitude Reduction Term}} is due to the relaxation of the
    {\BlueEmph{other electrons in the absorbing atom}} to the hole in the
    core level:

   \vmm
   \[
   S_0^2 =  {  |{\langle \Phi^{N-1}_f |\Phi^{N-1}_0 \rangle}|^2}
   \]

   \vmm

   ${| \Phi^{N-1}_0 \rangle }$ = $(N-1)$ electrons in unexcited atom.

   ${\langle \Phi^{N-1}_f|}$   = $(N-1)$ electrons, relaxed by core-hole.

   \vmm

   \onslide+<2->
   ${S_0^2}$ is usually taken as a constant:
   \begin{postitbox}{25mm}   $ 0.7 < S_0^2 < 1.0 $ \end{postitbox}

  and is used as a Fitting Parameter that multiplies {$\chi$}:

  \vmm  \onslide+<3->

  \begin{center}
    {\Red{ ${S_0^2}$ is Completely Correlated with $N$      (!!!)}}
  \end{center}

  \onslide+<3->

  \begin{postitbox}{95mm} \justify{\vspace{-5mm} $S_0^2$ -- along with
      normalization of spectra -- makes EXAFS amplitudes (and therefore
      {\Blue{$N$}}) less precise than EXAFS phases (and therefore
      {\Blue{$R$}}).}
  \end{postitbox}

\end{cenpage}
\vfill
\end{slide}
