
%%%%%%%%%%%%%%%%%%%%%%
\begin{slide}{Restraints: Imprecise Prior Knowledge}
    
    A {\BlueEmph{Constraint}} expresses exact ``Prior Knowledge'' (e.g.,
    force 1 $E_0$ for all paths), and reduces the number of variables 
    $\vec{x}$ in the least-squares problem:

    \pause

    \[ 
    \chi^2  =  \sum_{i}
    \biggl[ { {  \chi_{i}^{\rm data} - \chi_{i}^{\rm model}(\vec{x})
      }\over{\epsilon_i}} 
    \biggr]^2  
    \]

    \hrule \vmm
    
    A {\BlueEmph{Restraint}} expresses imprecise ``Prior Knowledge'', and
    adds another term to the least-squares problem:
   
    \[ 
    \chi^2  =  \sum_{i}
    \biggl[ { {  \chi_{i}^{\rm data} - \chi_{i}^{\rm model}(\vec{x})
      }\over{\epsilon_i}} 
    \biggr]^2  
    + 
    {\Red{
     \biggl[\frac{\lambda_0 - \lambda(\vec{x})}{\delta\lambda}\biggr]^2
      }}
    \]
    
    \vmm
    
    We add ``Data'' $\lambda_0$ to the fit, with ``Model''
    $\lambda(\vec{x})$ and confidence level $\delta\lambda$:
    
    \vmm

    {\hspace{25mm}} \highlightbox{{\Red{ $ \Lambda = [\lambda_0 -
          \lambda(\vec{x})]/{\delta\lambda} $ }}} {\hspace{5mm}} 

    \vmm
    becomes another part of the ``Vector to Minimize'' in the least-squares
    sense.
    
    
    \vmm \hrule  \vmm 

    \begin{center}
      \fcolorbox{black}{lightyellow2}{
        \begin{minipage}{88mm}  \setlength{\baselineskip}{10pt}
          
    Restraints give us a {\RedEmph{Bayesian}} approach to analysis:
    if we can quantify our ``Prior Knowledge'', we can use it in our fit 
    along side our data.
        \end{minipage}
        }
    \end{center}


    \vmm
\vfill
\end{slide} 



%%%%%%%%%%%%%%%%%%%%%%
\begin{slide}{Restraints as ``Soft Limits'' on Parameters}
    
  \vmm
  For some parameters (say, \pthpar{s02}), there may be an acceptable
  range of values.   

  \vmm
  
  \begin{tabular}{ll}
    \begin{minipage}{35mm}
      You could use a {\BlueEmph{constraint}} as a ``hard wall'': \end{minipage} &
    \begin{minipage}{54mm}
      \begin{VerbSMBox}{50mm} \bfseries
   guess S02_Var = 0.8
   path(1,  S02 = {\Red{max(0.6, min(1.0, S02_Var)) }} )

      \end{VerbSMBox}
    \end{minipage}\\
  \end{tabular}
  
\vmm This complicates finding error bars when the free variable
{\tt{S02\_Var}} goes outside the limits $[0.6,1.0]$.  And we do want the
error bars!

   \vmm \hrule \vmm
    
   An alternative: use a Restraint to make an increasingly worse fit when a
   variable is ``out-of-bounds'':

   \begin{tabular}{lr}
     \begin{minipage}{45mm} \vspace{-5mm}
       \[ \Lambda(x) =  \left\{ 
       \begin{array}{ll}
         x - x_{\rm hi} & x > x_{\rm hi} \\
         0  &    x_{\rm hi} \ge x \ge x_{\rm lo} \\
         x_{\rm lo} - x &     x_{\rm lo} \ge x\\
       \end{array}
     \right.
     \]
     \end{minipage}
   & 
   \hspace{1mm}\begin{minipage}{45mm} \wgraph{45mm}{restraints/penalty} \end{minipage}
   \\ 
 \end{tabular}

 \vspace{-0.3mm}
 \begin{tabular}{ll}
   \begin{minipage}{80mm}
     \begin{VerbSMBox}{60mm} \bfseries
  guess S02_Var = 0.8
  path(1,  S02 = S02_Var)
  feffit(\ldots , restraint = {\Red{penalty(S02_Var,0.6,1.0)}} )
     \end{VerbSMBox}
   \end{minipage} &      \begin{minipage}{20mm}   \hspace{20mm}  \end{minipage}
   \\
 \end{tabular}
 
 \vmm

\vfill
\end{slide} 


%%%%%%%%%%%%%%%%%%%%%%
\begin{slide}{Example Restraint: Bond Valence Sums}
    
    
    The {\BlueEmph{Bond Valence}} expresses the Valence of an ion $i$ in terms
    of its ligands. It is commonly parametrized as [Altermatt and Brown,
    {\emph{Acta Cryst.} {\bf{B41}}} (1985)]
    
    \[  V_i =  \sum_{j=1}^N \exp[(R'_{ij} - R_{ij})/ 0.37]   \]
    
    where $R_{ij}$ is the interatomic distance and $R'_{ij}$ is an
    empirical parameter for atom types (Fe-S, Cu-O, etc).

    \vmm
    
    We usually can tell the formal valence from XANES, and this model
    correlates this with $N$ and $R$ to keep them {\RedEmph{Chemically
        Reasonable}}.
    
    \vmm \vmm
    \vmm \vmm
    
    The Bond Valence Sums work to $\sim$ 10\% -- inappropriate for a
    constraint.


\vfill
\end{slide} 


%%%%%%%%%%%%%%%%%%%%%%
\begin{slide}{Bond Valence Sums as Restraint}
    
    As a {\Red{Restraint}}, the Bond Valence Sum becomes a ``Model''
    that we fit to our  Valence ``Data'' ($V_{\rm XANES} = 2,3,\ldots$):

    \[ \Lambda   = \frac{V_{\rm XANES} - \sum_{j=1}^N \exp[(R'_{j} -
      R_{j})/ 0.37]}{\delta\lambda} \]

    
    \vmm \vmm
    
    $\delta\lambda$ is our ``Bayesian Parameter'', letting us control how
    much we prefer what our $\chi(k)$ data tells us about $N$ and $R$
    compared to our ``Prior Knowledge'' of what the XANES and Bond Valence
    Sum tells us about $N$ and $R$:

    \vmm \vmm

    \begin{center} 
      \begin{tabular}{lll} \hline
        ${\delta \lambda} \rightarrow 0$ & Force $V = V_{\rm XANES}$ &
        {\Blue{constraint}} \\

        ${\delta \lambda} \rightarrow \infty$ & $V$ determined by our
        data alone &  {\Blue{no prior knowledge}}  \\ 
        Finite ${\delta \lambda}$ & 

        $V$ influenced by our data and $V_{\rm XANES}$  & {\Blue{restraint}}\\
        \hline

      \end{tabular}
    \end{center} 

    \vmm \vmm \vmm\vmm
    
    \begin{cenpage}{85mm}
    Bond Valence Sums give a simple, flexible way to restrain the EXAFS
    results to agree with the qualitative XANES interpretation. 
    \end{cenpage}

\vfill
\end{slide} 


\begin{slide}{Bond Valence Restraint for {$\mathsf{Mn_2O_3}$}}
    
    \vmm 
    
    $\mathsf{Mn_2O_3}$ has two Mn(III)-O octahedra and ~3 different Mn-O
    distances
    \begin{itemize}
    \item one regular   (6 O @ 1.993\AA).
    \item one distorted ({2 O @ 1.985\AA, 2 O @ 1.899\AA,  and  2 O @ 2.248\AA}). 
    \end{itemize}
    
    The data I have (from Lytle) is limited to $12\rm\mathsf\,\AA^{-1}$ ($2\Delta k
    \Delta R/\pi \approx 7$) -- hard to measure all distances!! 
    
    \vmm 
    \begin{center}
      \fcolorbox{black}{lightyellow2}{
        \begin{minipage}[c]{87mm}
          Can the Bond Valence Sum help us see the splitting in the
          Mn-O shell?
        \end{minipage}
      }
    \end{center}

\ThinGreyLine

We assert the following ``prior information'':
\begin{itemize}
\item The Valence for both sites should be close to  +3.
\item one Mn site has 6 O at one distance 
\item the other Mn site has 2 bonds that are the same length as the regular 
  octahedron, 2 ``short'' O bonds, and 2 ``long'' O bonds.
\end{itemize}

We'll vary 3 distances, $N$, $E_0$ and $\sigma^2$, and set $S_0^2$ to 0.7
(6 variables).

\ThinGreyLine

And, we'll {\Red{\emph{restrain}}} both sites to have a Bond Valence Sum near 3:

\vspace{-3mm}

\begin{eqnarray*}  \hspace{10mm}
  \lambda_1 &=& \bigl[ {{({V(R_1) - {V_0}})}/ {{{\delta V}}}} \bigr]^2
  \hfill \\
  \lambda_2 &=& \bigl[ {{({V(R_1,R2,R3) - {V_0}})}/ {{{\delta V}}}} \bigr]^2
  \hspace{50mm} \hfill \\
\end{eqnarray*}
   

\vfill  
\end{slide} %%%--------------------------------------------



\begin{slide}{Bond Valence Restraint for {$\mathsf{Mn_2O_3}$}}

  \begin{cenpage}{74mm}{\wgraph{73mm}{restraints/fit_mn2o3_r}} \end{cenpage} 

\vspace{-3mm}

\vmm \hrule \vmm

    Using {\Red {$ {\delta V}$ = 1}}: 
 
    \scriptsize

    \begin{tabular}{cc}
      \begin{minipage}{27mm}
        \begin{tabular}{ll}
          Parameter      & Value \\ 
          \noalign{\smallskip} \hline
          $C(N,R_1)$      &   0.2 \\
          $V_1$ (regular)   &   3.40\\
          $V_2$ (distorted) &   3.00\\
          \noalign{\smallskip} \hline
          $\cal R$        & 0.007 \\
          $\chi^2$        & 91.8 \\
          $2\Delta k \Delta R/\pi$ &  7.2 \\ 
          \noalign{\smallskip} \hline
        \end{tabular}
      \end{minipage}
      &
      \begin{minipage}{48mm}
        \begin{tabular}{lllc}
          Fit Variable &  \multicolumn{2}{l}{Best-Fit Value} & Predicted \\
          \noalign{\smallskip} \hline
          $N$         & 5.86   &(0.7)     & 6 \\
          $E_0$       & 1.2    &(1.9)   & \\
          $\sigma^2$  & 0.005  &(0.004) & \\
          \noalign{\smallskip} \hline
          {$R_1\quad (8x)$}   & {1.96}  &(0.02) &  1.99 \\
          {$R_2\quad (2x)$}   & {1.85}  &(0.09) &  1.90 \\
          {$R_3\quad (2x)$}   & {2.28}  &(0.05) &  2.25 \\
          \noalign{\smallskip} \hline
        \end{tabular} 
      \end{minipage}
    \end{tabular}


\vfill  
\end{slide} %%%--------------------------------------------
