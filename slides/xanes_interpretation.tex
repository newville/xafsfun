
\begin{slide}{XANES Interpretation}

    The EXAFS Equation breaks down at low-$k$, as the mean-free-path goes
    up.  This complicates XANES interpretation:

    \vmm
    \begin{postitbox}{68mm}
      We do not have a simple equation for XANES.
    \end{postitbox}

    \vmm
    XANES can be described {\BlueEmph{qualitatively}} and
    {\BlueEmph{semi-quantitatively}} in terms of
    \vspace{2mm}
    \pause
%%    \begin{minipage}{95mm}\setlength{\baselineskip}{10pt}
      {
        \begin{tabular}{ll}
          {\Red{coordination chemistry}} &  regular, distorted octahedral,
          tetrahedral, \ldots\\
          {\Red{molecular orbitals}} &
          ${p}$-${d}$ orbital hybridization,
          crystal-field theory, \ldots \\
          {\Red{band-structure}} & the density of available electronic
          states. \\
          {\Red{multiple-scattering}} &  multiple bounces of the
          photo-electron.\\
        \end{tabular}
        }
%%    \end{minipage}

%     \vspace{2mm}
%
%     These interpretations all seek to answer:
    \vmm

    \pause
    \begin{postitbox}{72mm}
      What electronic states can the photo-electron fill?
    \end{postitbox}

    \vmm XANES calculations ({\sc{FEFF9}}, {\sc{FDMNES}}) are becoming
    reasonably accurate.  These can help interpret spectra in terms of
    {\BlueEmph{bonding orbitals}} and/or {\BlueEmph{density of states}}.

%     \vmm
%     Quantitative XANES analysis using first-principles calculations are still
%     rare, but becoming possible...
\vfill
\end{slide}
