\subsection{Constraints and Generalized Variables}
\begin{frame}[fragile] \frametitle{Constraints and Generalized Variables}

  Instead of varying the Path Parameters directly, we write them in terms of
  {\RedEmph{Generalized Variables}}.  This allows simple {\RedEmph{Constraints}} and model building:

\begin{columns}
  \begin{column}{53mm}

  \begin{CodeBlock}{50mm}{Parameter=Variable}
{\Blue{\# one variable e0 for 2 paths}}
{\Red{params = group(e0 = guess(1.0), \ldots)}}

path1  = feffpath('feo.dat', {\Red{e0='e0'}})
path2  = feffpath('fefe.dat', {\Red{e0='e0'}})
  \end{CodeBlock}
  \hspace{1mm}   \vmm

  \onslide+<2->
  \begin{CodeBlock}{50mm}{Mixed Coordination Shell}
{\Blue{\# mix O and S in 1st coordination shell}}
params = group(s02   = param(0.80, vary=False),
               sfrac = guess(0.5))

path1 = feffpath('feo.dat', {\Red{s02='s02*sfrac'}})
path2 = feffpath('fes.dat', {\Red{s02='s02*(1-sfrac)'}})
  \end{CodeBlock}

  \vmm

  \end{column}
  \begin{column}{57mm}
    \onslide+<3->
  \begin{CodeBlock}{55mm}{Einstein Temperature }
{\Blue{\# Use 1 ``theta'' to set sigma2 for multiple paths}}

params = group(amp=param(1, vary=True),
               theta=param(250, min=0, vary=True), \ldots)

path1_100K = feffpath('fefe.dat', s02='amp', \ldots,
                      sigma2='sigma2_eins(100, theta)')

path1_200K = feffpath('fefe.dat', s02='amp', \ldots,
                      sigma2='sigma2_eins(200, theta)')

path1_300K = feffpath('fefe.dat', s02='amp', \ldots,
                      sigma2='sigma2_eins(300, theta)')

  \end{CodeBlock}\\
\end{column}
\end{columns}

\vmm This allows us to use {\BlueEmph{Prior  Knowledge}} into the data analysis, and
consider more complicated problems:

\vfill

  \begin{itemize}
  \item force one $R$ for the same bond for data taken from different
    edges.

  \item model complex distortions (height of a sorbed atom above a surface).
  \end{itemize}


\end{frame}

% \begin{slide}{Constraints, Generalized Variables examples}


%   Constraints and Generalized Variables are one kind of ``Prior
%   Knowledge'',  allowing us to build and compare simple physical models for
%   our data.

%   \vmm   This allows us to consider more complicated problems:

%   \vmm

%   {\Red{
%       \begin{tabular}{clcl}
%         &multiple neighboring species  &&  multiple sites for central atom \\
%         &multiple scattering paths     &&  multiple polarizations \\
%         &multiple data sets            &\hspace{4mm}&  multiple edges to co-refine   \\
%       \end{tabular}
%     }}

%   \vmm Other constraints:

%   \begin{itemize}
%   \item force one $R$ for the same bond for data taken from different
%     edges.

%   \item model complex distortions (height of a sorbed atom above a surface).
%   \end{itemize}

% \vfill
% \end{slide}
