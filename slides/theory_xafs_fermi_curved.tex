
\begin{slide}{Photo-electron Scattering and $\chi(k)$}
 
    \begin{tabular}{lc}
      \begin{minipage}{49mm}
        Going back to our definition
        \[ {
        \chi(k[E]) =   {\displaystyle
          {{\mu(E) - \mu_0(E)}\over {\mu_0(E)}}} }
        \]
        we'll work out a simple form for $\chi(k)$ that we can use 
        for analysis.
      \end{minipage}
      &
      \begin{minipage}{52mm}
      \fcolorbox{verywhite}{white}{
        \scalebox{1}{\wgraph{51mm}{theory/absorb02}}
        }
      \end{minipage}
    \end{tabular}

    \vspace{2mm}
    \onslide+<2->

    DONT USE ME!! 

With a spherical wave  for the
    photo-electron, 
    \[ \psi(k,r) = {{e^{ikr}}/{kr}} \]

and a scattering atom at a distance $r=R$, we
    get
      
    {\large
      \[ {\displaystyle \chi(k) = {{e^{ikR}}\over{kR}} \> 
      [{\Red{2k f(k)e^{i\delta(k)}}} ] \> 
      {{e^{ikR}}\over{kR}}  + C.C. }
      \]
    }
      
    where the neighboring atom gives the amplitude ${\Red{f(k)}}$ and
    phase-shift ${\Red{d\bdelta(k)}}$ to the scattered photo-electron.

\vmm
\vfill
\end{slide} 


%%     \[\mathbf {  \bmu(E) \bsim | \langle i | {\cal{H}} | f \rangle |^2   } \] 
  
%%     \begin{cenpage}{83mm}
%%     \begin{description}
%%       \settowidth{\labelwidth}{10mm} \setlength{\itemindent}{0mm}
%%     \item[{$\mathbf{\langle i |}$}] the {\BlueEmph{initial state}}
%%       describes the core level (and the photon).  This {\Red{is not}}
%%       altered by the neighboring atom.

%%     \item[{$\mathbf{\cal{H}}$}] the {\BlueEmph{interaction}}. In the
%%       dipole approximation, $ \mathbf{{\cal{H}} = e^{ikr}} \approx 1$.
       
%%     \item[{$\mathbf{| f \rangle}$}] the {\BlueEmph{final state}} describes the
%%       photo-electron (and no photon).  This {\Red{is}} altered by the
%%       neighboring atom.
%%     \end{description}

