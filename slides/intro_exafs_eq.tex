\begin{slide}{The EXAFS Equation}


  \begin{cenpage}{120mm}

    To model the EXAFS, we use the {\BlueEmph{EXAFS Equation}}:
  \vspace{-1mm}

  \begin{center}
    \[ \chi(k) = \sum_j {\frac{{\Blue{N_j}} {\Red{f_j(k)}}
        e^{-2{\Blue{R_j}}/{\Red{\lambda(k)}}}
        e^{-2k^2{\Blue{\sigma_j^2}}}}{k{\Blue{R_j}}^2}
      {\sin[{2k{\Blue{R_j}} + {\Red{\delta_j(k)}}] }}} \]
  \end{center}

  \vmm

  where $\Red{f(k)}$ and $\Red{\delta(k)}$ are
  {\RedEmph{photo-electron scattering properties}} of the neighboring
  atom [and $ \Red{\lambda(k)} $ is the photo-electron mean-free-path].

  \vmm
  If we know these properties, we can determine:
  \onslide+<2->
    \begin{description}
      \settowidth{\labelwidth}{15mm}
      \setlength{\itemindent}{15mm}
      \setlength{\leftmargin}{15mm}
    \item[$R$] distance to neighboring atom.
    \item[$N$] coordination number of neighboring atom.
    \item[$\sigma^2$] mean-square disorder of neighbor distance.
    \end{description}

  \vmm
  \onslide+<3>
  ${\Red{f(k)}}$ and ${\Red{\delta(k)}}$ depend on atomic number
  {\BlueEmph{Z}} of the scattering atom, so we can also determine the
  species of the neighboring atom.


\end{cenpage}

\end{slide}

