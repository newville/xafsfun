
%% Slide
\begin{slide}{Photo-Electron Mean-Free Path}



\begin{cenpage}{100mm}

  To get to

    \[ \chi(k) = \sum_j {\frac{{\Blue{N_j}} {\Red{f_j(k)}}
        e^{-2k^2{\Blue{\sigma_j^2}}}}{k{\Blue{R_j}}^2}
      {\sin[{2k{\Blue{R_j}} + {\Red{\delta_j(k)}}] }}} \]

 we used a spherical wave for the photo-electron:

  \begin{cenpage}{45mm}
    {$\displaystyle        \psi(k,r) \sim  \frac{{e^{ikr}}}{kr}  $}
  \end{cenpage}


    {\Blue{But}}:  The photo-electron can also scatter {\BlueEmph{inelastically}}, and may
    not be able to get back the absorbing atom coherently.

 \vmm

 A {\RedEmph{mean free path}} ($\lambda$) describes how far the
 photo-electron can go before it scatters from and loses energy to other
 electrons, phonons, etc.

\vmm\vmm  Using a damped wave-function:

\begin{cenpage}{40mm}
  {$\displaystyle \psi(k,r) \sim \frac{e^{ikr} e^{-r/\lambda(k)}}{kr} $ }
\end{cenpage}

adds another term to the EXAFS equation.
\end{cenpage}
\end{slide}


\begin{frame} \frametitle{ $\lambda(k)$: The Photo-Electron Mean-Free Path }

\begin{cenpage}{108mm}

\hfsetfillcolor{yellow!60}
\hfsetbordercolor{yellow!60}

   \[ \chi(k) = \sum_j {\frac{{\Blue{N_j}} {\Red{f_j(k)}}
       {  \tikzmarkin{a}(-0.02,-0.08)(1.3,0.35){  e^{-2{\Blue{R_j}}/{\Red{\lambda(k)}}} }}
      e^{-2k^2{\Blue{\sigma_j^2}}}}{k{\Blue{R_j}}^2}
     {\sin[{2k{\Blue{R_j}} + {\Red{\delta_j(k)}}] }}} \]

 \vfill
  The $ e^{-2R/\lambda(k)} $ term in the XAFS Equation accounts for how far the
  photo-electron can travel and still return coherently) to the excited atom.


 \begin{columns}
   \begin{column}{54mm}
     \rgraph{58mm}{lambda.png}
   \end{column}
   \begin{column}{64mm}
     \begin{itemize}
     \item $\lambda$ is mostly independent of atom, molecule, environment.
     \item  can include effect of finite lifetime ($\lesssim$fs) of the core-hole.
     \item  $\lambda$ and $R^{-2}$ make EXAFS a  {\RedEmph{local probe}}.
     \end{itemize}
     \vmm
   \end{column}
 \end{columns}

\vmm

\begin{center}
  \begin{tabular}{lccl}
    Regime &  $k$ values  ( $\rm\AA^{-1}$)
   & $\lambda$ range ($\rm\AA$) & effect\\
    \noalign{\smallskip}
     \noalign{\hrule}
    \noalign{\smallskip}
    EXAFS  &   3 to 12 &  $< 20 \rm\,\AA$  &   {\RedEmph{local  atomic  probe}} \\
    \noalign{\smallskip}
    XANES &    $< 3$  &  $> 20 \rm\,\AA$  &   {\RedEmph{extended   probe}}    \\
    \noalign{\smallskip}
    \noalign{\hrule}
    \end{tabular}

\end{center}

  \end{cenpage}
 \vfill
 \end{frame}
