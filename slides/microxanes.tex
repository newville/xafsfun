
\begin{frame}\frametitle{XANES/EXAFS Spatial and Concentration Ranges}

  Sensitivities and time required for X-ray Spectroscopy
  measurements from a X-ray microprobe:

  \vmm

  \begin{tabular}{lllll}
    {\Red{Measurement}} & Result  & Concentration &   Resolution    &   {\Blue{Time}} \\
    \noalign{\medskip}    \hline   \noalign{\medskip}
    $\mu$-XRF     & Abundances      &  $<$ 1 ppm    & 1 $\rm{\mu}m$        & 10 sec \\
    XRF Mapping   & Relative abundances  &  $\sim$ 5 ppm  & 1 $\rm{\mu}m$   & 50 msec \\
    $\mu$-XANES   & oxidation state       & $\sim$ 10 ppm & 2 $\rm{\mu}m$   & 10 min \\
    $\mu$-XAFS    & $1^{\rm st}$ neighbor distance & $\sim$ 50 ppm & 5 $\rm{\mu}m$   & 60 min \\
    \noalign{\medskip}    \hline
  \end{tabular}

\vmm\vmm

The more the energy changes, and the more time required, the worse the
spatial resolution is.

\vmm

\begin{postitbox}{65mm}
  These are estimates of typical-to-best conditions for soil samples, and
  can vary greatly.
\end{postitbox}
\end{frame}

\begin{frame}\frametitle{${\mu}$XANES --  Primary vs. diagenetic sulfur phases in  bryozoan fossils}

  \begin{columns}[T]
    \begin{column}{60mm}
      {\wgraph{60mm}{microprobe/SFossil_Map}}

      \begin{minipage}{60mm}
        XRF map of 3mm section of a bryozoan fossil from the Ordovician era
        (450 Myears ago) - 5 $\rm{\mu}m$ pixels, 30 msec/pt, repeated at 2
        incident energies to emphasize sulfide (red), sulfate (blue) and
        silicon (green).
        \end{minipage}
      \vmm

      {\small{\Blue{David Fike, Catherine Rose, Jeff Catalano (Washington Univ)}}}
    \end{column}
    \begin{column}{55mm}

      \begin{minipage}{55mm}
        Carbonate fossils have up to 1000 ppm sulfate, carrying a record of
        the sulfur content of the ancient.  S isotope analysis are highly
        variable and suggest a late deposition of S.
        \end{minipage}

        {\wgraph{55mm}{microprobe/SFossil_XANES}}

      \begin{minipage}{55mm}
        S $\mu$XANES at several spots confirms localized sulfide grains -
        probably detrital sulfides - while most of the fossil is dominated by
        sulfate.
        \end{minipage}
    \end{column}
  \end{columns}
\end{frame}


\begin{frame}\frametitle{${\mu}$XAFS -- Sr in coral aragonite}


  \begin{columns}[T]
    \begin{column}{45mm}
      \hspace{2mm} {\wgraph{38mm}{microprobe/Coral_CaMap}}

      \hspace{2mm} {\wgraph{38mm}{microprobe/Coral_SrMap}}
      \begin{minipage}{45mm}
        XRF maps of 0.3mm section of a natural coral of Ca (top) and Sr
        (bottom) show diurnal variation in [Sr] / [Ca] ratio on a 10 micron
        scale.
      \end{minipage}
      \vmm

      {\small{\Blue{Nicola Allison, Adrian Finch (Univ St. Andrews)}}}

    \end{column}
    \begin{column}{58mm}

    \begin{minipage}{58mm}

      Sr concentration in corals has been used as a paleothermometer.
      Over what length-scale is this valid?

      \vmm \hspace{30mm}{\wgraph{15mm}{microprobe/Coral_SEM}}
      \vmm
    \end{minipage}

    {\wgraph{60mm}{microprobe/Coral_Sr_uXAFS_R}}

    \begin{minipage}{58mm}

      Sr $\mu$-XAFS shows that coral trap Sr in aragonite without
      precipitating $\rm SrCO_3$, even when [Sr] $>$ above the solubility
      limit of Sr in aragonite.

    \end{minipage}
  \end{column}
\end{columns}
\end{frame}


\begin{frame}\frametitle{High Energy-Resolution Fluorescence}

XAFS and XANES are heavily used in environmental sciences,

\begin{itemize}
\item {\RedEmph{element-specific}} probes of chemical state for any species.
\item Minimal sample requirements.
\item Work at relatively low concentrations (ppm).
\item Require tunable, monochromatic X-rays -- synchrotron.
\end{itemize}

\onslide+<2->

\begin{columns}[T]
  \begin{column}{48mm}
    {\wgraph{48mm}{microprobe/HighResFluor}}
  \end{column}
  \begin{column}{65mm}

    High (Energy) resolution X-ray Fluorescence ($K_\beta$)
    give chemical information comparable to XANES.

\begin{itemize}
\item Element-specific, minimal sample constraints, as XAS.
\item Scan/analyze emission energy with 1 eV resolution, {\bf{not incident energy}}!
\item Currently photon-starved measurements, so fairly high concentrations.
\item {\RedEmph{detector/analyzer-limited}}, not intrinsic.
\item Useful for environmental science?
\end{itemize}



  \end{column}
\end{columns}
\end{frame}
