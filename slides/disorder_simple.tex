
\section{Theory/Complications: Disorder Terms}

\begin{slide}{Structural Disorder and the Pair Distribution Function}


  \begin{cenpage}{120mm}

    
  An EXAFS measurement averages billions of {\BlueEmph{snapshots}} of the
  local structure:

  \onslide+<2->
  
    \begin{itemize}
    \item Each absorbed x-ray generates 1 photo-electron.
    \item the photo-electron / core-hole pair lives for about
      $10^{-15}$ s --  much faster than the timescale for thermal vibrations ($10^{-12}$ s).
    \item An EXAFS measurement samples $10^4$ (dilute fluorescence) to $10^{10}$
      absorbed x-rays for each energy point.
    \end{itemize}
  
  \vmm \onslide+<3->
  \vmm
  
  So far, we've put this in the EXAFS Equation as \hspace{2mm}
  $\chi \sim N \exp({-2k^2\sigma^2}) $
  
  \vmm \hrule \vmm \onslide+<4->
  
  \begin{columns}
  \begin{column}{50mm}
    More generally, EXAFS samples the
    
    \vmm
    {\RedEmph{Partial Pair Distribution Function}}
    
    \vmm
    
    {\RedEmph{$g(R)$}} =   probability that an
    atom is a distance $R$ away from the absorber.
    
    \vmm\vmm

    For now, we'll just note that this may need to be taken into account.
    
    
  \end{column}
  \begin{column}{70mm}
    
    \scalebox{1}{\wgraph{65mm}{errors/gnxas}}
  \end{column}
\end{columns}

\end{cenpage}

\end{slide}


