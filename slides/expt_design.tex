
%% Slide
\begin{slide}{X-ray Absorption Measurements: Experimental Design}

Important points to consider for measuring XAFS are:

\vmm

    \begin{description}
    \pause \item[{\RedEmph{Monochromatic x-rays:}}] Need x-rays with a small energy
      spread or bandwidth: ${\Delta E \approx 1 \rm\, eV}$ at 10keV.
      
    \pause \item[{\RedEmph{Linear Detectors:}}] The XAFS ${\chi(k) \sim
        10^{-2}}$ or smaller, so we need a lot of photons and detectors
      that are very linear in x-ray intensity (ion chambers).  This usually
      means using a synchrotron source.
      
    \pause \item[{\RedEmph{Well-aligned Beam:}}] The x-ray beam hitting the
      detectors has to be the same beam hitting the sample.
   
    \pause \item[{\RedEmph{Homogeneous sample:}}] For transmission measurements,
      we need a sample that is of uniform and appropriate sample thickness
      of {$\sim$}1 absorption lengths.  It should be free from pinholes.
      Powders need to be very fine-grained.

    \pause \item[{\RedEmph{Counting Statistics:}}]
    ${\mu(E)}$ should have a noise level of about
    ${10^{-3}}$.  That means we need to collect at least
    ${\sim 10^6}$ photons.   For very low concentration samples, this may
    require hours of counting time.
% 
% 
%       \begin{cenpage}{80mm}
%         \begin{description}
%         \item[{\BlueEmph{Transmission:}}] Fluxes at synchrotrons are  ${> 10^8\rm \,
%             photons/sec}$. Count rate is not much of an issue.
% 
%         \item[{\BlueEmph{Fluorescence:}}] Flux and Count rate may be a concern, especially when
%           concentrations are very low.
%         \end{description}
%         
%         
%       \end{cenpage}

    \end{description}

\vfill
\end{slide} 
