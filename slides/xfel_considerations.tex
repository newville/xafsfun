
\begin{frame} \frametitle{XAFS with an XFEL}
  \small
  \begin{cenpage}{100mm}\setlength{\baselineskip}{11pt}
    \vspace{1mm}
    
    \ldots ought to work, provided the energy is easily tunable and the
    sample survives.

    \vmm     \hrule     \vmm
    
    {\RedEmph{Multiple Excitations / non-linear x-ray absorption?}}
    
    The time-scale for XAFS is $\sim 1\rm\, fs$.  At $10^{17} \rm\,
    photons/sec$, 100 atoms in a sample would be excited at any one time.
        
    \vmm
    
    $10^{17} \rm\, photons/sec$ onto a $\rm 10nm \times 10 nm \times 10 nm$
    sample ($3\times 10^4\rm\, atoms$), $\sim$1\% of x-rays would be
    absorbed, and the photo-electrons might start interacting \ldots.
    
    \vmm {\Red{Would a sample (and optics) survive this flux?}} $10^{17}
    \rm\, Hz$ of $ 10 \rm \,keV$ is $160 \rm\, W $.  At the APS we
    routinely have $10^{11} \rm\, Hz$ of $ 10 \rm \,keV$ photons ($160
    \rm\, mW $) and can damage many samples.

    \vmm     \hrule     \vmm
    
    {\RedEmph{Experimental Opportunities:}}
    
    \begin{itemize}
    \item time-resolution pump-probe experiments  (storage rings have much large bunch lengths)
    \item brighter sources mean smaller beam sizes.

    \item flux limited ``related techniques'' such as RIXS and x-ray raman
      become more attractive.

    \end{itemize}

    \vmm     \hrule    

  \end{cenpage}

\vfill
\end{frame} 
