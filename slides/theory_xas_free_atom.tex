\begin{slide}{X-ray Absorption by a Free Atom}

  \vmm
  \begin{cenpage}{125mm}
    
  \Justify An atom absorbs an X-ray (energy $E$), destroying a core electron (energy
  ${E_0}$) and creating a photo-electron (energy ${E-E_0}$).  The
  {\Red{core hole}} is eventually filled, and a fluorescence X-ray or Auger
  electron is ejected from the atom.

    \vspace{3mm}

    \begin{columns}[T]
        \onslide+<1->
        \begin{column}{65mm}
          \rgraph{65mm}{xafscartoon_bare}
        \end{column}

        \onslide+<2->
        \begin{column}{55mm}    \setlength{\baselineskip}{10pt}
          \Justify

        X-ray absorption needs an available state for the
        photo-electron to go into:\par

        \vmm

        \begin{center}
          \begin{postitbox}{28mm}
            No available state:\par No absorption
          \end{postitbox}
          \end{center}
      \vspace{1mm}

      Once the X-ray energy is large enough to promote a core-level to the
      continuum, there is a sharp increase in absorption.

      \end{column}
    \end{columns}

    \vspace{1mm}

    {\onslide+<3->

    \begin{center}
      \begin{postitbox}{80mm}\Justify
        ${\mu(E)}$ has a sharp step at the core-level binding energy, and
        is a smooth function of energy above this absorption edge.
        \par
        \end{postitbox}
     \end{center}
     }

   \end{cenpage}
\end{slide}
