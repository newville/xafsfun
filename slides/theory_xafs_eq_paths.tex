%% Slide
\begin{slide}{The EXAFS Equation: Sum over Scattering Paths}

  \begin{cenpage}{120mm}

    We need some way to account for different neighbor species (Fe-O, Fe-Fe):

  \[ \chi(k) = {\Red{\sum_j}} \, \frac{ {\Blue{N_j}} S_0^2 {\Red{f_j(k)}}
      e^{-2{\Blue{R_j}}/{\Red{\lambda(k)}}}\,
      e^{-2k^2{\Blue{\sigma_j^2}}}}{k{\Blue{R_j}}^2}
    \sin{ [ 2k{\Blue{R_j}} + \Red{\delta_j(k)} ]}  \]

    \begin{cenpage}{85mm}
      This sum  could be over ``shells'' of atoms, but more generally it is over
      {\RedEmph{scattering paths}} for the photo-electron.  This allows
      {\RedEmph{multiple scattering}} to be taken into account.
    \end{cenpage}

  \vmm
  \onslide+<2->

    \begin{tabular}{ll}
      \begin{minipage}{50mm}
        \onslide+<2->
        \scalebox{1}{\rgraph{50mm}{mspaths}}
      \end{minipage}
      &
      \begin{minipage}{62mm}\setlength{\baselineskip}{10pt}
        \onslide+<2->
        \vspace{2mm} {\Blue{Triangle Paths}} with angles $ 45 < \theta <
        135^{\circ}$ aren't strong, but there can be a lot of them.

        \onslide+<2->
        \vspace{2mm}
        {\Blue{Linear paths}} with angles $\theta \approx 180^{\circ}$,
        are very strong: the photo-electron can be {\Red{focused}} through
        one atom to the next.

        \vspace{1mm}
        \hspace{1mm} Scattering is strongest when   $ \theta > 150^{\circ}$.

        \vspace{1mm}
        \hspace{1mm} This can be used  to measure bond angles.

      \end{minipage}
    \end{tabular}

\vmm   For first shell analysis, multiple scattering is hardly ever needed.

\end{cenpage}

\end{slide}
