
\begin{slide}{X-ray Absorption: Fermi's Golden Rule}

    \vmm
    \begin{cenpage}{125mm}
      
    \begin{columns}
      \begin{column}{55mm}
        Going back to our definition
        \[   \mu(E) =   \mu_0(E) [1 + \chi(E)]     \]
        we'll work out a form of the EXAFS Equation
        for $\chi(k)$ to use    in      analysis.
      \end{column}
      \begin{column}{53mm}
        \rgraph{53mm}{xafscartoon_scatter}
      \end{column}
    \end{columns}

    \vmm
    \onslide+<2->

    Fermi's Golden Rule describes $\mu(E)$ as a transition between quantum
    states:

    \[  \mu(E) \sim | \langle i | {\cal{H}} | f \rangle |^2   \]

    \onslide+<3->

    \begin{description} \settowidth{\labelwidth}{5mm} \setlength{\itemindent}{-5mm}

    \item[{${\langle i |}$ \hspace{1mm}}] the {\BlueEmph{initial state}} has a
      core level electron and the photon. \par This {\Red{is not}} altered by
      the neighboring atom.
    \item[{${\cal{H}}$ \hspace{1mm}}] the {\BlueEmph{interaction}}. In the
      dipole approximation, $ {{\cal{H}} = e^{ikr}} \approx 1$.
    \item[{${| f \rangle}$ \hspace{1mm}}] the {\BlueEmph{final state}} has
      a photo-electron, a {\Red{hole}} in the core, and no photon.  \par  This is
      altered by the neighboring atom: {\RedEmph{ the photo-electron scatters}}.
    \end{description}
  \end{cenpage}
\end{slide}
