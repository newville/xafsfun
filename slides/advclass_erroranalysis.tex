\section{Error Analysis}

\begin{slide}{Fitting Strategies in IFEFFIT }
    

    The general idea is to find a {\RedEmph{Model}} that best matches the {\BlueEmph{Data}}.

    \vmm
    
    We use $\bchi^2$ (don't confuse with EXAFS $\bchi$!!) to describe our
    fit:

    \[ \displaystyle{\mathbf{
        \chi^2  =   \sum_i^{N_{\rm fit}} \frac{[\chi_i^{\rm data} - \chi_i^{\rm
            model}({x})]^2}{\epsilon^2} 
      }} \]

    
    \vmm $\mathbf{N_{\rm fit}} = $ number of points to fit, 
    $\mathbf{x} = $  the set of variables, and $\mathbf{\epsilon} =$ the estimated noise
    level in the data.

    \vmm \hrule \vmm
    
    But we usually oversample our data (that is, $\mathbf{N_{\rm fit} >
      N_{\rm idp} } $), so we renormalize this to $ \mathbf{N_{\rm idp}}$,
    and assert that $\epsilon$ is a constant:

    \begin{center} \highlightbox{
        $ \displaystyle{\mathbf{
            \chi^2  =  \frac{ N_{\rm idp}}{\epsilon^2 N_{\rm fit}}
            \sum_i^{N_{\rm fit}} [\chi_i^{\rm data} - \chi_i^{\rm model}({x})]^2
          }} 
        $
      } 
    \end{center}

    \vmm

    \begin{center}
       {\RedEmph{ The Best Fit is that with lowest $\bchi^2$. }}
    \end{center}

    \vmm \hrule \vmm

    \vmm
    
    The fitting algorithm finds the set of variables $\mathbf{x} $ which
    minimizes $[\mathbf{\chi_i^{\rm data} - \chi_i^{\rm model}({x})}]$ in
    this ``least-squares'' sense.


\vfill
\end{slide} 

\begin{slide}{Other Fitting Statistics}
\small

    Other ``goodness-of-fit statistics'':
    \begin{description}
    \item[{\Red{chi-square}}] As before: \vmm

      $ \displaystyle{\mathbf{
          \chi^2  =  \frac{ N_{\rm idp}}{\epsilon^2 N_{\rm fit}}
          \sum_i^{N_{\rm fit}} [\chi_i^{\rm data} - \chi_i^{\rm model}({x})]^2
        }} 
      $
      
      
    \item[{\Red{reduced chi-square}}] scale $\mathbf{N_{\rm varys}}$ by the 
      ``degrees of freedom'' : \vmm

      $ \chi^2_\nu =  \chi^2 / (N_{\rm idp}-N_{\rm varys}) $


      \vmm

      
      For a ``Good Fit'', $\chi^2_\nu$ should be $\sim$ 1 (This
      {\RedEmph{never}} happens!!).

    \item[{\Red{R-factor}}] $\mathbf{\cal{R}}$  gives a ``fractional misfit'': \vmm

      $ \displaystyle{\mathbf{
          {\cal{R}} = 
          {\sum_i^{N_{\rm fit}}[\chi_i^{\rm data} - \chi_i^{\rm model}({x})]^2 }
          / 
           { \displaystyle{\sum_i^{N_{\rm fit}} [{\chi_i^{\rm data}}]^2}}
        }}
      $

      \vmm $\mathbf{\cal{R}}$ is useful because it is not scaled to the data
    uncertainty $\epsilon$. 

    \end{description}
\vfill
\end{slide} 


\begin{slide}{Fitting in $R$-space, $k$-space, etc.}
    
    The $\bchi^2$ definition didn't say anything about what our data
    {\Blue{$\chi_i^{\rm data}$}} is.

    \vmm  We usually fit in $R$-space, because 
    we can throw out unwanted ``shells'':
    
    \vmm \vmm \vmm 

    {
      \begin{tabular}{ll}
        \hspace{-10mm} \begin{minipage}{54mm}{\wgraph{53mm}{reduction/chik}}  \end{minipage} &
        \hspace{-5mm}  \begin{minipage}{54mm}{\wgraph{53mm}{reduction/chir_win}}\end{minipage}
    \end{tabular}
    }

    \vmm 
    Fitting in $R$-space gives more meaningful fit statistics when we know that
    we're not fitting all the spectral features.
    
    \vmm 
    {\RedEmph{AND:}} We can also have the data {\Blue{$\chi_i^{\rm data}$}}
    extend over {\Red{multiple data sets}}, {\Red{multiple $k$-weightings}},
    etc, as long as we generate {\Red{$\mathbf{\chi_i^{\rm model}(x)}$}} to
    match these data.

\vfill
\end{slide} 


%%%%%%%%%%%%%%%%%%%%%%
\begin{slide}{$\mathbf\epsilon$: The Noise Levels in Data }

    
    Here are some typical EXAFS spectra (both transmission, 1 sec/point), 
    and their estimated noise $\epsilon$ in $\bchi(k)$:


    \vmm

      \begin{tabular}{lcl}\setlength{\baselineskip}{2pt}
        0.2 mM Zn nitrate solution & & Cu foil Room Temperature   \\
        \hspace{-7mm} \wgraph{48mm}{errors/noise_data01} & \hspace{2mm} & 
        \hspace{-5mm} \wgraph{48mm}{errors/noise_data02} \\
        $\epsilon \approx 6.6 \times 10^{-4}$ ``Normal Data''& &
        $\epsilon \approx 1.6 \times 10^{-4}$ ``Good Data''\\
      \end{tabular}

      \vmm

      For the Zn solution, this noise level is consistent with counting
      statistics for ``Number of Photons in Ion Chambers''! 

\vfill
\end{slide} 


%%%%%%%%%%%%%%%%%%%%%%
\begin{slide}{$\epsilon$: Noise Levels in Data}
\small
\hspace{-.1mm}
\begin{center}
  \begin{minipage}{95mm}\setlength{\baselineskip}{10pt}
    
    $\epsilon$ is estimated from the high-$R$ (15
    to 25 \AA) components of the data:
    
    \vmm
      \begin{tabular}{lcl}
        $|\chi(R)|$ &    &  $ \hspace{4mm} \log_{10}(|\chi(R)|)$ \\
        \vspace{-3mm}\hspace{-10mm} \wgraph{50mm}{errors/znnoise01} &      & 
        \hspace{-5mm}  \wgraph{50mm}{errors/znnoise02} \\
      \end{tabular}
      
      \begin{itemize}
        \item The Zn data really shows ``white noise''.
          
        \item The Cu data has signal well above the noise level past 10\AA!

      \item Using the range $R=[15,25]\rm\, \AA$ may 
        {\RedEmph{overestimate}} $\epsilon$ for good data.
      \end{itemize}
      
      A general property of Fourier transforms (Parseval's Theorem) translates
      the noise in $\bchi(R)$ to the noise in $\bchi(k)$. 
      

      \begin{center}
        \highlightbox{ \begin{minipage}{70mm} When fitting in $R$-space, we want
            $\epsilon_R$, the noise in $\bchi(R)$, but $\epsilon_k$, the noise
            in $\bchi(k)$, is easier to interpret.
      \end{minipage}}\end{center}

  \end{minipage}
\end{center}
\vspace{1mm}
\vfill
\end{slide} 




%%%%%%%%%%%%%%%%%%%%%%
\begin{slide}{Error Bars: the uncertainities in the fit variables}
\small
\hspace{-.1mm}
\begin{center}
  \begin{minipage}{95mm}\setlength{\baselineskip}{10pt}
    
    A fit finds $\Blue{\mathbf{x_0}}$, the ``best fit values'' of the variables
    $\Blue{\mathbf{x}}$ by minimizing $\bchi^2$.

    
    \vmm \vmm The uncertaintes in $\Blue{\mathbf{x}}$ are found by increasing
    the best $\bchi^2$ by 1:

    \vmm
    \begin{center} \wgraph{60mm}{errors/ellipse} 
    \end{center}
      
    \vmm

    {\RedEmph{increase by 1}} implies that the fit is ``Good'' ($\chi^2_\nu
    \approx 1$).
    
    \vmm Sadly, we typically get {\highlightbox{$\chi^2_\nu \gtrsim 50$!! }}
    even for very good fits.
    So, we increase the best $\bchi^2$ by $\bchi^2_\nu$.
    
    \vmm 
    \begin{center}{\highlightbox{
          The ``$\bchi^2 + 1$'' error bars $\delta x$ are multiplies by
          $\mathbf{\sqrt{\bchi^2_\nu}}$.
        }}
    \end{center}
    
  \end{minipage}
\end{center}
\vspace{1mm} \vfill
\end{slide} 

%%%%%%%%%%%%%%%%%%%%%%
\begin{slide}{Error Bars: correlations between fit variables}
\small
\hspace{-.1mm}
\begin{center}
  \begin{minipage}{95mm}\setlength{\baselineskip}{10pt}
    
    Pairs of variables can be {\RedEmph{correlated}}: changing one variable away
    from its optimum value can be compensated by changing another variable away
    from its best value.  The uncertainties needs to take correlations into
    account.

    \vmm
    \begin{center} \wgraph{60mm}{errors/ellipse2}   \end{center}
      

    \vmm 
    The uncertainty in $x$ is $\delta x$, NOT  $\delta x'$!
    
    \vmm 
    
    The correlations values are useful statistics, measuring the ellipse shape.

    \vmm
    
    In EXAFS, ($R$, $E_0$) and ($N$, $\sigma^2$) are usually very highly
    correlated ($>0.85$).


  \end{minipage}
\end{center}
\vspace{1mm}
\vfill
\end{slide} 



\begin{slide}
  \small
  \begin{cenpage}{80mm}\setlength{\baselineskip}{10pt}
    \vfill
    \begin{center}
      {\Huge\BlueEmph{ Some Fit Examples}   }
  
    \end{center}
    \vfill
  \end{cenpage}
\end{slide} 


%%%%%%%%%%%%%%%%%%%%%%
\begin{slide}{Room Temperature Cu Fit }
\small
\hspace{-.1mm}
\begin{center}
  \begin{minipage}{100mm}\setlength{\baselineskip}{10pt}
    
    Simple fit to first shell of Cu foil (300K): $k = [2,16] \rm\,
    \AA^{-1}$, $R = [1.7,2.6] \rm\, \AA$, $k$-weight=2, $N_{\rm idp} = 8.4
    $.  Fit results and statistics:
    

    {
      \hspace{0.1mm}\begin{tabular}{lll}
        $R = 2.548(0.007) \, \rm\AA$ 
        &     
        $\Delta E_0 = 4.5(0.6)$ 
        &  
        $C_3      = 9(9) \times10^{-5} \rm\, \AA^3$ 
        \\
        $\epsilon_k = 1.6 \times 10^{-4}$ 
        &
        $S_0^2 = 0.96(0.04)$  
        &
        $\sigma^2 = 8.5(0.3) \times10^{-3} \rm\, \AA^2$ 
        \\
        $\chi^2 = 678$ &
        $\chi^2_\nu = 196.7$   & ${\cal{R}} = 0.00107 $\\
      \end{tabular}
    }

    \vmm
      \begin{tabular}{lcl}
        \hspace{-10mm} \wgraph{49mm}{errors/cufit02} & \hspace{2mm} & 
        \hspace{-3mm}  \wgraph{49mm}{errors/cufit01} \\
      \end{tabular}

      \begin{itemize}
      \item ${\cal{R}} = 0.1\% $ -- a good fit!  But like $\chi^2_\nu$,
        ${\cal{R}}$ is larger than the $\epsilon_k$ suggests.
      \item These error bars account for correlations.  They increase
        $\chi^2$ by $\chi^2_\nu$ (not 1), which scales them by
        $\sqrt{\chi^2_\nu}\approx 14$ over ``increase $\chi^2$ by 1''.
      \end{itemize}

      \vmm

  \end{minipage}
\end{center}
\vspace{1mm}
\vfill
\end{slide} 


