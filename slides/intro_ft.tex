\begin{slide}{EXAFS Fourier Transform: $\chi(R)$}

  \begin{center}
  \begin{tabular}{ll}
    \begin{minipage}{65mm}  {\wgraph{65mm}{reduction/chir}}  \end{minipage}
    &
    \begin{minipage}{37mm}  \setlength{\baselineskip}{10pt}
      \vspace{1mm} 
      {\Red{$\mathbf{\bchi(R)}$}}\vspace{0.5mm}      
      
      A Fourier Transform of $\mathbf{k^2\bchi(k)}$ shows peaks
      corresponding to shells of atoms around the absorbing atom. 
      
      \vspace{3mm}      
      
      $|\mathbf{\bchi(R)}|$ looks like a radial distribution function, but
      is more complicated than that.

    \end{minipage}\\
    \begin{minipage}{65mm}  {\wgraph{65mm}{reduction/chir_complex}}
    \end{minipage}
    &
    \begin{minipage}{37mm} \setlength{\baselineskip}{10pt}

       \vspace{1mm}       \hrule
       \vspace{3mm}

       
       {\Red{$\chi(R)$ is complex:}}\vspace{0.5mm}      

       \vspace{2mm}
       
      Usually only the amplitude of $\chi(R)$  is shown (as above).


      \vspace{1mm}
      
      Both both Phase and Amplitude (Real/Imaginary) parts of
      $\chi(R)$ are important, and used in modeling.

      \vspace{3mm}
    \end{minipage}
  \end{tabular}
  \end{center}    

\vfill
\end{slide} 
