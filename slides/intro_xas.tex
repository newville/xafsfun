
% Slide
\subsection{X-Ray Absorption}
\begin{frame} \frametitle{X-Ray Absorption and the Photo-Electric Effect}

  \begin{cenpage}{89mm}
  X-rays are
%% light with wavelength $0.025{\rm\,\AA} \lesssim \lambda \lesssim 25
%% {\rm\,\AA}$ and energy $ 500{\rm\, eV} \lesssim E \lesssim 500
%%  \rm\,keV$.  They are
  absorbed by all matter through the {\RedEmph{photo-electric effect}}:
  \end{cenpage}

  \vspace{-3mm}

  \begin{columns}[T]
    \begin{column}{50mm}

      \vspace{3mm} {\ }

       \begin{postitbox}{50mm}
        An atom absorbs an x-ray when the x-ray energy is transferred to a
        core-level electron ({\sl{K}}, {\sl{L}}, or {\sl{M}} shell).

          \vmm\vmm

          The atom is left in an {\RedEmph{excited state}} with a {\RedEmph{core
              hole}} -- an empty electronic level.

          \vmm
          Any excess energy from the x-ray is given to an
          ejected {\RedEmph{photo-electron}}.
        \end{postitbox}


        \begin{columns}
          \begin{column}{8mm}
            \rgraph{8mm}{Einstein}
          \end{column}
          \begin{column}{40mm}

            {\tiny{A. Einstein, Nobel Prize, 1921 ``For his services to Theoretical Physics, and
               especially for his discovery of the law of the photoelectric
               effect''.}}

          \end{column}
        \end{columns}

      \end{column}
      \begin{column}{62mm}
        \rgraph{62mm}{photoelectric}
      \end{column}
    \end{columns}

\vfill
\end{frame}
