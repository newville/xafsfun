\section{Information Theory,  Fitting Constraints}

 \begin{slide}
   \small
   \begin{cenpage}{80mm}\setlength{\baselineskip}{10pt}
     \vfill
     \begin{center}
     {\Huge\BlueEmph{ Information Theory, Generalized Variables   }}
 
     \end{center}
     \vfill
   \end{cenpage}
 \end{slide} 

\begin{slide}{EXAFS Analysis and Information Theory}
    
    The basic difficulties with EXAFS Analysis are:

    \begin{enumerate}
    \item The basis functions (Paths, Shells, \ldots) are not very well
      resolved, and their number grows exponentially with $R$.

    \item The scattering factors \feffc{f(k)}, \feffc{\delta(k)} are not simple to
      calculate.
      
    \item There's not much information in a real measurement:
      \[ N_{\rm idp} \approx \frac{2\Delta k\Delta R}{\pi} \]
      
    \end{enumerate}
    
    \hrule \vmm
    
    Quantitative EXAFS analysis procedures address these with methods to:
    
    \begin{enumerate}
    \item {reduce} the number of Paths to consider (Fourier analysis).

    \item re-use or parametrize {\emph{ab initio}} calculations of
      \feffc{f(k)}, \feffc{\delta(k)} (using {\scshape{FEFF}} or something similar).
      
      
    \item recycle parameters to cut down the number of independent
      variables in the fit, while keeping a meaningful analysis.
      
    \end{enumerate}

    \hrule \vmm
    
    Generally, we parametrize the EXAFS with a physical model, and then
    put {\RedEmph{Constraints}} and {\RedEmph{Restraints}} on the
    parameters in a least-squares analysis.

    \vmm
\vfill
\end{slide} 

%%%%%%%%%%%%%%%%%%%%%%
\begin{slide}{Constraints: why we need them}
    
    The EXAFS Equation (slightly simplified here): 
    
%     \[
%     \displaystyle{{\chi_j(k) = \sum_j
%         \frac{\feffc{f(k)} \feffc{N}\pthpar{S02} e^{(-2p''\reff - 2p^2\pthpar{sigma2})}}
%         {k(\reff + \pthpar{delR})^2} 
%         \bsin[ 2k\reff + \feffc{\delta(k)}
%           + 2p\pthpar{delR} ]  
%         }}
%     \]
    
    \vmm has 7 {\Blue{Path Parameters}} that could be refined per Path $j$:
%     \begin{center} 

%        \pthpar{S02}, \pthpar{e0}, \pthpar{ei}, 
%        \pthpar{delR}, \pthpar{sigma2}, \pthpar{third}, and \pthpar{fourth}.
%     \end{center}
    
    
    \vmm 
    
    But the information content of EXAFS is low:
%     \begin{center} \[ N_{\rm idp} = 8 \> \> \> 
%       {\rm for} \> \Delta R = 1 \, {\rm \AA} \>\> {\rm and} \>\> \Delta k = 12.5\,
%     {\rm \AA^{-1} } \] \end{center}
 
  
    \vmm \hrule \vmm

    For well-isolated single path problems, it's fine to just fit $N$, $R$,
    $\sigma^2$, and $E_0$.
    
    \vmm 
    But for more complicated problems, including all the ``multiples'':

    \vmm

    {\Red{
        \begin{tabular}{clcl}
          \hspace{3mm}
          &multiple neighboring species  &&  multiple sites for central atom \\
          &multiple scattering paths     &&  multiple polarizations \\
          &multiple data sets            &\hspace{4mm}&  multiple edges to co-refine   \\
        \end{tabular}
        }}

    \vmm
    we need to limit the number of parameters, and want to impose a
    physical model to get more meaningful results.

\vfill
\end{slide} 


%%%%%%%%%%%%%%%%%%%%%%
\begin{slide}{Constraints, continued}
    
    {\BlueEmph{Constraints}} allow us to impose relationships between the
    parameters, and create physical models about our data.  For example: 

    \begin{itemize} 
    \item force the same $E_0$ for all Paths (may not always \ldots).
      
    \item allow mixed coordination shell: $x$ (Fe-O) + (1-$x$)(Fe-Fe).
      
    \item make $\sigma^2$ follow a simple temperature law (e.g,  an
      Einstein model).
      
    \item force one $R$ for the same bond for data taken from different
      edges.

    \item model complex distortions (height of a sorbed atom above a surface).

    \end{itemize}
  
    \hrule \vmm \vmm
    
    {\ifeffit} uses {\RedEmph{Generalized Variables}} for Constraints in
    the Fit:

    \vmm\vmm

    
    The {{Path Parameters}} \pthpar{delR}, \pthpar{S02}, \pthpar{sigma2},
    \pthpar{e0}, \ldots are not adjusted directly.
    
    \vmm Instead, the Path Parameters are written as mathematical
    expressions of the Generalized Variables.  This makes it easy to share
    Variables across:

    \begin{itemize}
    \item different Path Parameters for a Path
    \item different Scattering Paths (including Multiple Scattering)
    \item different data sets (different temperature, edge, polarization 
      \ldots)
    \end{itemize}
      


    

\vfill
\end{slide} 


% %%%%%%%%%%%%%%%%%%%%%%
% \begin{slide}{Generalized Variables and DEF in IFEFFIT }
    
%     {\ifeffit} has user-defined Generalized Variables for easy
%     model-building (ie, putting constraints on Path Parameters).


%      \vmm
   
%      \begin{tabular}{ll}
%        \hspace{-5mm} \begin{minipage}{44mm}
%             \begin{VerbSBox}{40mm} \bfseries
%   {{set   S02     = 0.80 }}
%   {\Blue{guess N_FeO   = 5.0 }}

%   {\Red{def    N_FeS     = 6 - N_FeO  }}
%   {\Red{def    Amp_Path1 = N_FeO * S02 }}
%   {\Red{def    Amp_Path1 = N_FeS * S02 }}

%   path(1,  feff=feff_feo.dat, degen=1,
%        s02= Amp_Path1 )

%   path(2,  feff=feff_fes.dat, degen=1,
%        s02= Amp_Path2)

%   feffit(1,2, chi=data.chi,....)
%     \end{VerbSBox}

%     \end{minipage}
%     & 

%     \begin{minipage}{56mm}
     
%       The $S_0^2$ parameter is set to 0.80
%       \vmm 
%       {\Blue{\tt{N\_FeO}}} will be varied in the fit, adjusting the
%       amplitude for Paths 1 (Fe-O) and 2 (Fe-S).
%       \vmm
 
%      Both {\Blue{\tt{N\_FeO}}} and {\Red{\tt{N\_FeS}}} can change, but not
%      independently: they must sum to 6.
     
%      \vmm 
     
%      {\Red{\tt{N\_FeS}}}, {\Red{\tt{Amp\_Path1}}} and
%      {\Red{\tt{Amp\_Path2}}} are {\RedEmph{defined}} parameters: their
%      values will adjust in the fit according to the formula provided.
%
%   \end{minipage}\\
%  \end{tabular}    
    

%   \vmm

     
%   {\Blue{\tt{N\_FeO}}}, {\Red{\tt{N\_FeS}}}, {\Red{\tt{Amp\_Path1}}}, etc
%   are {\RedEmph{Generalized Variables}}.  These can be:

  
%   \begin{description}
%   \item[{\Blue{Normal variables}}] ({\tt{S02}}), that are set to take a fixed value
%   \item[{\Blue{Fitting variables}}] ({\tt{N\_FeO}}), that are freely adjusted in the
%     fit.
%   \item[{\Blue{Defined variables}}] ({\tt{N\_FeS}}), that are functions of other
%     Variables whose values will adjust according to the formula provided. 
  
%   \end{description}


%   \vmm
  
%   \vmm 
    
%  \end{slide} 




% %%%%%%%%%%%%%%%%%%%%%%
% \begin{slide}{Some Examples of Generalized Variables}
    
    
%     Examples of Path Parameters written as functions of the Generalized
%     Variables:

%     \vmm
    
%     \begin{tabular}{ll}
%       \hspace{-5mm} \begin{minipage}{40mm}
%            \begin{VerbSBox}{38mm} \bfseries
%  {\Blue{# simplest case: Parameter=Variable}}

%  {\Red{guess e0    = 1.0 }}

%  path(1,  e0 = e0)
%  path(2,  e0 = e0)
%    \end{VerbSBox}

%    \vspace{-2.5mm}

%    \begin{VerbSBox}{38mm} \bfseries
% {\Blue{# mixed coordination shell}}
%  set S02   = 0.80

%  {\Red{guess x   = 0.5}}
  
%  path(1,  S02 = S02 * x )

%  path(2,  S02 = S02 * (1-x) )
%    \end{VerbSBox}
%    \end{minipage}
%    & 
%    \begin{minipage}{59mm}
%    \begin{VerbSBox}{58mm} \bfseries
% {\Blue{# Einstein Temperature }}

%  set   factor  = 24.254337   {\Blue{#= (hbar*c)^2/(2 k_boltz)}}
% {\Blue{# mass and reduced mass in amu}}
%  set   mass1 = 63.54,  mass2 = 63.54
%  set   r_mass =  1/ (1/mass1 +  1/mass2)

% {\Blue{# the Einstein Temp will be adjusted in the fit!}}
%  {\Red{guess thetaE = 200}}
% {\Blue{# use for data set 1, T=77}}
%  set   temp1 = 77
%  {\Red{def ss2_path1 = factor*coth(thetaE/(2*temp1))/r_mass )}}
%  path(101,  sigma2 = ss2_path1   )

% {\Blue{# use for data set 2, T=300}}
%  set   temp2 = 300
%  {\Red{def ss2_path2 = factor*coth(thetaE/(2*temp2))/r_mass )}}
%  path(201,  sigma2 = ss2_path2   )
%     \end{VerbSBox} 
%     \end{minipage}\\
    
%   \end{tabular}    
    

% %% \sigma^2 = {{(\hbar c)^2} \over{2k_{B}}} { {\coth(\theta /2T) }
% %% \over{M_{\rm R}\theta}}

%   \vmm\vmm
  
%   Constraints and Defined Variables are one kind of ``Prior Knowledge'',
%   allowing us to build and compare simple physical models for our data.

%   \vmm 
    
% \vfill
% \end{slide} 


