

\begin{slide}{Scattering Amplitude and Phase-Shift}

  The scattering amplitude ${\Red{f(k)}}$ and phase-shift
  ${{\Red{\delta(k)}}}$ depend on atomic number.

  \vmm

    \begin{tabular}{ll}
      \onslide+<1->
      \begin{minipage}{57mm}   \rgraph{57mm}{scatt_amp}       \end{minipage}
      &
      \begin{minipage}{57mm}  \rgraph{57mm}{scatt_pha}    \end{minipage}
      \\

      \onslide+<1->
      \begin{minipage}{56mm}\setlength{\baselineskip}{10pt}
        ${{\Red{f(k)}}}$ extends to higher ${k}$ values for higher $Z$
        elements.  For very heavy elements, there is structure in
        ${{\Red{f(k)}}}$.

      \end{minipage}
      &

      \begin{minipage}{56mm}\setlength{\baselineskip}{10pt}
        ${{\Red{\bdelta(k)}}}$ shows sharp changes for very heavy elements.
        These functions can be calculated  for modeling EXAFS.
      \end{minipage}
    \end{tabular}


\vmm \hrule\vmm\vmm

\onslide+<2->

    These complex factors allow EXAFS to distinguish the species of
    neighboring atom:

\begin{columns}
\begin{column}{90mm}{\ }

      \begin{postitbox}{70mm}
        ${Z}$ can usually be determined to $\pm 5$.

        Fe and O can be  distinguished, but not Fe and Mn.
      \end{postitbox}
\end{column}

\end{columns}


  \vfill
\end{slide}
