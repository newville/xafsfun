
\begin{slide}{Fluorescence Measurements: Solid-State Detectors}
 
    \begin{tabular}{lr}
      \begin{minipage}{64mm} \setlength{\baselineskip}{10pt}
        \vspace{3mm}

        An alternative is to use a {\RedEmph{solid-state detector}} with Ge
        or Si as the x-ray absorber. This uses electronic energy
        discrimination.
        \vspace{1mm}
     
        This has an advantage of being able to measure the {\RedEmph{Full
            XRF Spectra}}, for identifying other elements.
     
        \vspace{1mm} This can be used for XAFS measurements with
        concentrations down to 10's of ppm.
        \vfill
      \end{minipage}
      &
      \begin{minipage}{30mm}
        \scalebox{1}{\wgraph{25mm}{experiment/med_det}}
      \end{minipage}
    \end{tabular}

  \vspace{3mm}

  Though this has many advantages, it has a few drawbacks:

  \begin{description}
    \settowidth{\labelwidth}{25mm}    \setlength{\itemindent}{0mm}
  \item[{\RedEmph{Dead time}}] The electronic discrimination saturates at
    ${\bsim 10^5 \rm\,Hz}$ or so.\par Ten (or more) detectors are often used
    in parallel, but XAFS measurements are still often limited by these
    detectors.

  \item[{\RedEmph{Complicated}}] Maintaining, setting up, and using one of
    these is more work than an ion chamber.  
  \end{description}
  \vfill

\end{slide} 
