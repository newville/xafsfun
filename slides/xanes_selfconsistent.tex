
\begin{frame} \frametitle{Self-Consistent Potentials for XANES calculations}
  \small
  \begin{cenpage}{100mm}\setlength{\baselineskip}{10pt}

    \vmm
   
    For XANES, we should use a  {\RedEmph{Self-Consistent Potential}}.   

    \vmm
    
    The electron density is modified by the neighboring atoms just as
    $\mu$.  Since the electron density of states gives $\mu$, is must be
    consistent with it:

    \vspace{-1mm}

    \begin{eqnarray*} 
      \mu(E)      &=&  \mu_0(E) [ 1 + \chi(E) ]  \qquad   \qquad   \qquad   \qquad  \\
      \rho_l(E)   &=&  \rho_{0,l}(E) [ 1 + \chi_l(E) ]  \,\qquad l = 0,1,2,3,...\\
    \end{eqnarray*}

    \vspace{-4mm} \hrule \vmm

    A self-consistent loop is used to refine the potential for the XANES:
    
    \vmm

    \begin{tabular}{ll}
      \hspace{-4mm}  \begin{minipage}{60mm}

        math here
        
        % \begin{equation*}
        %     \rho_0(E) @>>> \psi(E),\Phi(E) @>>> \chi(E) \\
        %     && @AAA @VVV \\
        %     && \mathrm{converged?} @<<< \rho(E)
        % \end{equation*}
      \end{minipage} & 
      \begin{minipage}{42mm}
        convergence in $\sim 10$ iterations.
        
        
        Gives the DOS for each $l$ ($s$, $p$, $d$, $f$), and improves
        the calculation of $E_{\rm Fermi}$
      \end{minipage} \\
    \end{tabular}
    \vmm
      
    Note: self-consistency  wasn't needed for EXAFS because $\chi \ll 1 $.
    
    
    \vmm 
    
    With a SCF and Full Multiple Scattering, a cluster of $\sim 100$
    atoms  is usually sufficient for a good qualitative match to experiment.
    
  \end{cenpage}
\vfill
\end{frame} 
