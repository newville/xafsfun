%% Slide
\section{X-ray Absorption Measurements}
\begin{frame} \frametitle{X-ray Absorption Measurements}

  \begin{tabular}{ll}
    {\onslide+<1->{\wgraph{42mm}{general/kl_edges}}} &
    {\onslide+<2->{\wgraph{60mm}{experiment/cartoon}} \vmm \vmm}\\
    {\onslide+<1-> $K$ and $L_{\rm III}$ Edge Energies } & 
    {\onslide+<2-> $\mu(E)$ can be measured two modes:} \\
  \end{tabular}
  

  \vmm
  {\onslide+<3->

  \begin{description}
  \item[Transmission]  measure what is transmitted through the sample:  
    \[    I = I_0 e^{-\mu(E)t}  \]
    Appropriate for concentration samples:  $ \gtrsim 10 \rm\, wt.\%$.
    %% \pause
  \item[Fluorescence] measure fluorescent x-rays from the re-filling the core hole:
    \[      \mu(E)  \propto I_f/I_0     \]

    Appropriate for dilute elements:  $ \lesssim 2 \rm\, wt.\%$ (to ppm, or so).
  \end{description}
  \pause
  \vmm
}

  {\onslide+<4->  
  \begin{postitbox}{85mm}
    We need a measurement of $\mu(E)$ to $\sim 0.1\%$, but with
    an energy-tunable x-ray source, the measurements are fairly easy.
  \end{postitbox}
}

\vfill
\end{frame}

