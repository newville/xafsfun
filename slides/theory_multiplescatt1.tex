
%% Slide
\begin{slide}{Multiple Scattering}
    \vmm
    
    The photo-electron can scatter multiple times:

    \vmm

    \begin{center}\scalebox{1}{\wgraph{85mm}{theory/mspaths2}}\end{center}
    
    
    A {\Blue{Path Formalism}} based on Green's Functions is used in modern
    EXAFS, for `''Real Space'' calculations on arbitrary clusters of atoms:
    
    \[ G = G^0 + G^0 t G^0 + G^0 t G^0 t G^0 + 
    G^0 t G^0 t G^0 t G^0 + \ldots \]
    
    where ${\Blue{G^0}}$ describes the propagation of the electron, and
    $\Blue{t}$ describes the scattering from the potential of the
    neighboring atom.

    \vmm
   
    
    \vmm {\Blue{Triangle Paths}} with angles $ 45 < \theta <
    135^{\circ}$ are weak, but there are many of them.
    
    \vmm {\Blue{Linear paths}} with angles $\theta \approx 180^{\circ}$ 
    are very strong: the photo-electron can be {\Red{focused}} through
    one atom to the next.
    

    \vmm This strong angular dependence  can be used to measure bond angles.

    
        


    \vfill

\end{slide} 

