%% Slide
\begin{slide}{X-ray Absorption with Photo-Electron Scattering}

  \Justify With another atom nearby, the ejected photo-electron can
  {\RedEmph{scatter}} from a neighboring atom.  The amplitude of the
  photo-electron scattered back to {\RedEmph{the absorbing atom}} will
  cause oscillations in $\mu(E)$.

    \vspace{3mm}

    \begin{columns}[T]
      \begin{column}{62mm}

        \begin{overprint}[62mm]
          \onslide+<1| handout: 0>  \rgraph{62mm}{xafscartoon_noscatter}
          \onslide+<2| handout: 0> \rgraph{62mm}{xafscartoon_scatter}
          \onslide+<3| handout: 1> \rgraph{62mm}{xafscartoon_scatter}
        \end{overprint}
      \end{column}

      \begin{column}{44mm} \setlength{\baselineskip}{10pt}
          \justify

          {\onslide+<2->
            The photo-electron scattered back will interfere with itself.

            \vmm\vmm

            $\mu$ depends on the presence of an electron state with energy
            ${(E-E_0)}$, at the absorbing atom.

            \vmm\vmm

            The scattered photoelectron partially fills that state.
          }
      \end{column}
    \end{columns}

    \vspace{-2mm}

    {\onslide+<3->

    \begin{center}
      \begin{postitbox}{105mm}\Justify
        XAFS oscillations are due to the interference of the
        outgoing  photo-electron with the
        photo-electron scattered from neighboring atoms.
        \end{postitbox}
      \end{center}
    }

\vmm\vmm
\end{slide}
