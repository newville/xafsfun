
%% Slide
\begin{slide}{Photo-Electron Mean-Free Path (extrinsic losses)}
  To get to

    \[ \chi(k) = \sum_j {\frac{{\Blue{N_j}} {\Red{f_j(k)}}
        e^{-2k^2{\Blue{\sigma_j^2}}}}{k{\Blue{R_j}}^2}
      {\sin[{2k{\Blue{R_j}} + {\Red{\delta_j(k)}}] }}} \]

 we used a spherical wave for the photo-electron:

  \begin{center}
      \begin{minipage}{40mm}
        {$\displaystyle        \psi(k,r) \sim   {{e^{ikr}}/{kr}}  $}
      \end{minipage}
  \end{center}


 But \ldots   \vmm \pause

  \begin{minipage}{100mm}

    The photo-electron can also scatter {\BlueEmph{inelastically}}, and may
    not be able to get back the absorbing atom in tact.

  \vmm

  A {\RedEmph{mean free path}} ($\lambda$) describes how far the
  photo-electron can go before it scatters, losing energy to other
  electrons, phonons, etc.

  \pause \vmm

  {\BlueEmph{Plus}}: the core-level has a {\RedEmph{finite lifetime}},
  also limiting how far the photo-electron can go out and make it back to
  ``the same'' absorbing atom.

  \end{minipage}


  \vfill
\end{slide}

%% Slide
\begin{slide}{Photo-Electron Mean-Free Path}

  To account for the mean-free-path,  we can
replace the
  spherical photo-electron wavefunction:


  \begin{center}
      \begin{minipage}{40mm}
        {$\displaystyle        \psi(k,r) \sim   {{e^{ikr}}/{kr}}  $}
      \end{minipage}
  \end{center}


\vmm  with a damped wave-function: \vmm

  \begin{center}
      \begin{minipage}{40mm}
        {$\displaystyle \psi(k,r) \sim \frac{e^{ikr} e^{-r/\lambda(k)}}{kr} $ }
      \end{minipage}
  \end{center}

which simply adds another term to the EXAFS equation:
\vmm
  \onslide+<2->

    \[ \chi(k) = \sum_j {\frac{{\Blue{N_j}} {\Red{f_j(k)}}
        e^{-2{\Blue{R_j}}/{\Red{\lambda(k)}}}
        e^{-2k^2{\Blue{\sigma_j^2}}}}{k{\Blue{R_j}}^2}
      {\sin[{2k{\Blue{R_j}} + {\Red{\delta_j(k)}}] }}} \]

  \vfill
\end{slide}

\begin{slide}{The Photo-Electron Mean-Free Path}

  \vspace{-1mm}

  \begin{cenpage}{70mm} \rgraph{70mm}{lambda}   \end{cenpage}

  $\lambda$ is mostly independent of the system, and depends strongly on $k$:

  \begin{itemize}
  \item For $3\rm \,\AA^{-1} <k<15 \rm \,\AA^{-1}$ , $\lambda \rm<
    30\,\AA$
  \item This (and the $R^{-2}$ term) makes EXAFS a {\RedEmph{local
        atomic probe}}.
  \item For XANES  ($k < 3\rm\,\AA^{-1}$)    Both $\lambda$ and $R^{-2}$ get
    large:  \par XANES is not really a {\RedEmph{local probe}}.
  \end{itemize}

\vfill
\end{slide}
