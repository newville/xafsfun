\section{The EXAFS Equation}

\begin{slide}{The EXAFS Equation}  % XAFS Analysis with {\feff}}

  \begin{cenpage}{135mm}

  The XAFS Equation used with {\feff}:

  \[
  \chi(k) = \sum_j {{ S_0^2 {\Blue{N_j}} {\Red{f_j(k)}}  e^{-2R_j/{\Red{\lambda(k)}}}
      e^{-2k^2{\Blue{\sigma_j^2}}}}\over{k{\Blue{R_j}}^2}}
  {\sin[{2k{\Blue{R_j}} + {\Red{\delta_j(k)}}} ]}
   \]

   \begin{itemize}
  \item $\Red{f(k)}$ and $\Red{\delta(k)}$ are  {\emph{photo-electron scattering
        amplitude and phase}}:
    \begin{itemize}
    \item Energy dependent    \hspace{3mm}  $k \sim \sqrt{(E-E_0)} $.
    \item Depend on $Z$ of the scattering atom(s).
    \item Non-trivial: must be calculated or carefully extracted from  measured spectra.
    \end{itemize}

\item $\Red{\lambda(k)}$ tells how far the photo-electron can travel.

\item The sum is over {\RedEmph{Scattering Paths}} of the photo-electron,
  from absorbing atom to neighboring atom(s) and back.  May include
  {\BlueEmph{multiple scattering}}!

\end{itemize}

   \begin{postitbox}{64mm}
     If we know $\Red{f(k)}$,  $\Red{\delta(k)}$, and $\Red{\lambda(k)}$, we can get:
     \begin{itemize}
     \item ${\Blue{R}}$ --  near neighbor distance.
     \item ${\Blue{N}}$ -- coordination number.
     \item ${\Blue{\sigma^2}}$ -- mean-square  disorder in ${\Blue{R}}$.
     \end{itemize}
   \end{postitbox}

\end{cenpage}  \end{slide}
