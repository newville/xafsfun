
%% Slide
\begin{slide}{Calculating  ${f(k)}$ and ${\delta(k)}$ with {\feff} }

    
    These days, we can calculate ${\Red{f(k)}}$ and
    ${{\Red{\delta(k)}}}$ with computer programs: {\feff}.
    
    Though not necessarily ``User-Friendly'', this program takes as input:

    \begin{enumerate}
    \item a list of atomic x,y,z coordinates for a physical structure (
      tools %% such as {\Program{atoms}} and {\Program{crystalff}} 
      exist to
      use convert crystallographic data to the required list).  
    \item a selected central atom.
    \end{enumerate}
    
    \pause It can be more painful than that, but is still pretty easy \ldots

    \vmm \pause
    The result is a set of files: 

    {\Red{\file{feff0001.dat}}},
    {\Red{\file{feff0002.dat}}}, \ldots, each containing the
    ${\Red{f(k)}}$, ${{\Red{\delta(k)}}}$,
    ${{\Red{\lambda(k)}}}$ for a particular scattering ``shell'' or
    ``scattering path'' for that cluster of atoms.
    
    \vmm \vmm
    
    Many analysis programs use these {\feff} files directly to model EXAFS
    data.

    \vmm
    
    \begin{cenpage}{89mm}
      
      A structure that is close to the expected structure can be used to
      generate a {\feff} model, and used in the analysis programs to refine
      distances and coordination numbers.
      
    \end{cenpage}

  \vfill
\end{slide} 
