%% Slide
\begin{slide}{What we left out: Muffin-Tin and Curved Wave Scattering}

    \vmm
    
    {\Red{Curved-Wave Scattering:}}

    A plane-wave scattering calculation for $f(k)$ isn't good enough for
    EXAFS.     We need to replace 

    \[ f = \Sigma_l (-1)^l (2l+1) e^{i\delta_l} \sin\delta_l \]

    with 
    \[ f \approx \Sigma_l (-1)^l (2l+1) e^{i\delta_l} \sin\delta_l \Red{ e^{il(l+1)/2kR} } \] 
    
  
    This gives large scattering phase shifts, and requires the calculation
    to $l\approx 25$. 

    \vmm  \hrule \vmm

    {\RedEmph{Muffin-Tin Approximation:}}

    The muffin-tin approximation allows spherically symmetric charge
    densities to be used, greatly simplifying the calculations (and relying
    on a wealth of literature).  
  
    \vmm  \hrule \vmm

    {\Red{Green's Functions:}}

    More efficient than evaluating wave-functions $\psi$ is to use Green's
    functions:
    
    \[ -{\frac{1}{\pi}}{\rm Im} G(dr',r,E) = \Sigma_f | f \rangle>\delta(E-E_f) \langle f| \]

    \[ {  \mu(E) \propto -{\frac{1}{\pi}}{\rm Im} \langle i | 
      \hat\epsilon\cdot{r'} G({r'},{r},E) \hat\epsilon\cdot {r} |
      i \rangle \Theta(E-E_f)  } 
    \] 


\end{slide} 
